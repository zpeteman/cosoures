\documentclass[12pt]{article}                                                                                                                                                                                                            
\usepackage[utf8]{inputenc}                                                                                                                                                                                                              
\usepackage[T1]{fontenc}                                                                                                                                                                                                                 
\usepackage{amsmath, amssymb}                                                                                                                                                                                                            
\usepackage{geometry}                                                                                                                                                                                                                    
\usepackage{xcolor}                                                                                                                                                                                                                      
\usepackage{lipsum}                                                                                                                                                                                                                      
\usepackage{titlesec}                                                                                                                                                                                                                    
\usepackage{fancyhdr}                                                                                                                                                                                                                    
\usepackage{tcolorbox}                                                                                                                                                                                                                   
\usepackage{graphicx}                                                                                                                                                                                                                    
\usepackage{float} % For precise figure placement                                                                                                                                                                                        
\usepackage{tikz}                                                                                                                                                                                                                        
\usetikzlibrary{arrows.meta, decorations.markings}                                                                                                                                                                                       
                                                                                                                                                                                                                                         
\geometry{a4paper, margin=2.5cm}                                                                                                                                                                                                         
\pagestyle{fancy}                                                                                                                                                                                                                        
\fancyhf{}                                                                                                                                                                                                                               
\rhead{Problem Set I}                                                                                                                                                                                                                    
\lhead{Algebra II}                                                                                                                                                                                                                      
\cfoot{\thepage}                                                                                                                                                                                                                         
                                                                                                                                                                                                                                         
% Fancy titles                                                                                                                                                                                                                           
\titleformat{\section}{\normalfont\Large\bfseries}{Exercise \thesection:}{1em}{}                                                                                                                                                         
\titleformat{\subsection}{\normalfont\bfseries}{Correction:}{1em}{}                                                                                                                                                                      
                                                                                                                                                                                                                                         
% Custom box for correction                                                                                                                                                                                                              
\tcbuselibrary{listingsutf8}                                                                                                                                                                                                             
\newtcolorbox{correctionbox}{                                                                                                                                                                                                            
  colback=gray!5,                                                                                                                                                                                                                        
  colframe=black,                                                                                                                                                                                                                        
  fonttitle=\bfseries,                                                                                                                                                                                                                   
  title=Correction,                                                                                                                                                                                                                      
  breakable,                                                                                                                                                                                                                             
  before skip=10pt,                                                                                                                                                                                                                      
  after skip=10pt                                                                                                                                                                                                                        
}                                                                                                                                                                                                                                        
                                                                                                                                                                                                                                         
\begin{document}                                                                                                                                                                                                                         
                                                                                                                                                                                                                                         
% Cover info                                                                                                                                                                                                                             
\begin{center}                                                                                                                                                                                                                           
	\Large\textbf{UNIVERSITY IBN TOFAIL} \\[1em]                                                                                                                                                                                           
	\large\textit{Algebra II} \\[2em]                                                                                                                                                                                                     
	\large\textit{Problem Set I} \\[0.5em]                                                                                                                                                                                                 
\end{center}                                                                                                                                                                                                                             
                                                                                                                                                                                                                                         
\vspace{1cm}                                                                                                                                                                                                                             
                                                                                                                                                                                                                                         
% ----------- EXERCISE 1 -------------                                                                                                                                                                                                   
\section{}                                                                                                                                                                                                                               
Study the following propositions. Prove those that are true and provide counterexamples for those that are false.
\begin{enumerate}
    \item[a)] $\mathbb{R}^2$ with the usual addition and the external law: $\lambda \cdot (x, y) = (\lambda x, 0)$ where $\lambda \in \mathbb{R}$, $(x, y) \in \mathbb{R}^2$ is a vector space over $\mathbb{R}$.
    
    \item[b)] $\mathbb{C}^3$ with the usual addition and the external law over $\mathbb{C}$ defined by $\lambda \cdot (x, y, z) = (\lambda x, y, z)$ where $\lambda \in \mathbb{C}$, $(x, y, z) \in \mathbb{C}^3$ is a $\mathbb{C}$-vector space.
    
    \item[c)] The set of polynomials with real coefficients divisible by $X^3 + 1$, with the usual addition of polynomials and multiplication of a polynomial by a scalar, is an $\mathbb{R}$-vector space.
\end{enumerate}
                                                                                                                                                                                                                                         
\begin{correctionbox}                                                                                                                                                                                                                    
\begin{enumerate}                                                                                                                                                                                                                                 
    \item[a)] 
              
    \item[b)] 
              
    \item[c)] 
\end{enumerate}                                                                                                                                                                                                                                    
\end{correctionbox}                                                                                                                                                                                                                      
                                                                                                                                                                                                                                         
% ----------- EXERCISE 2 -------------                                                                                                                                                                                                   
\section{}                                                                                                                                                                                                                               
Consider the following sets:
\begin{align*}
E_1 &= \{(x, y, z) \in \mathbb{R}^3 \mid 3x - 2y + 5z = 0\}\\
E_2 &= \{v \in \mathbb{R}^3 \mid v = (a - b, 2b, a + 3b), a, b \in \mathbb{R}\}\\
E_3 &= \{(x, y, z) \in \mathbb{R}^3 \mid x \cdot y = 0\}
\end{align*}

\begin{enumerate}
    \item Among these sets, which ones are vector subspaces of the vector space $\mathbb{R}^3$ over $\mathbb{R}$?
    \item Give a basis for each vector subspace.
\end{enumerate}
                                                                                                                                                                                                                                         
\begin{correctionbox}                                                                                                                                                                                                                    
 \begin{enumerate}                                                                                                 
     \item 
     \item 
 \end{enumerate}                                                                                                   
\end{correctionbox}                                                                                                                                                                                                                      
                                                                                                                                                                                                                                         
% ----------- EXERCISE 3 -------------                                                                                                                                                                                                   
\section{}                                                                                                                                                                                                                               
In the vector space $\mathbb{R}^4$ with its canonical basis, consider the vectors:
\begin{align*}
e'_1 &= (1, 2, -1, -2)\\
e'_2 &= (2, 3, 0, -1)\\
e'_3 &= (1, 3, -1, 0)\\
e'_4 &= (1, 2, 1, 4)
\end{align*}

\begin{enumerate}
    \item[a)] Show that the family $B' = (e'_1, e'_2, e'_3, e'_4)$ is a basis of $\mathbb{R}^4$.
    \item[b)] Calculate the coordinates of the vector $v = (7, 14, -1, 2)$ in the basis $B'$.
\end{enumerate}
                                                                                                                                                                                                                                         
\begin{correctionbox}                                                                                                                                                                                                                    
 \begin{enumerate}                                                                                
     \item[a)] 
     \item[b)] 
 \end{enumerate}                                                                                  
\end{correctionbox}                                                                                                                                                                                                                      
                                                                                                                                                                                                                                         
% ----------- EXERCISE 4 -------------                                                                                                                                                                                                   
\section{}                                                                                                                                                                                                                               
Consider the set:
\begin{align*}
F = \{(x, y, z) \in \mathbb{C}^3 \mid x + y + z = 0 \text{ and } 2x + iy - z = 0\}
\end{align*}

\begin{enumerate}
    \item[a)] Show that $F$ is an $\mathbb{R}$-vector space.
    \item[b)] Give a basis for $F$ and deduce its dimension.
\end{enumerate}

\begin{correctionbox}                                                                                                                                                                                                                    
 \begin{enumerate}                                            
     \item[a)] 
     \item[b)] 
 \end{enumerate}                                              
\end{correctionbox}                                                                                                                                                                                                                      
                                                                                                                                                                                                                                         
% ----------- EXERCISE 5 -------------                                                                                                                                                                                                   
\section{}                                                                                                                                                                                                                               
Let $F$ be the vector subspace of $\mathbb{R}_4[X]$ generated by the following vectors (polynomials):
\begin{align*}
P_1 &= X^2\\
P_2 &= (X - 1)^2\\
P_3 &= (X + 1)^2
\end{align*}

\begin{enumerate}
    \item[a)] Show that $(P_1, P_2, P_3)$ is a basis of $F$.
    \item[b)] Complete the family $(P_1, P_2, P_3)$ into a basis of $\mathbb{R}_4[X]$ and deduce a supplementary subspace of $F$ in $\mathbb{R}_4[X]$.
\end{enumerate}
                                                                                                                                                                                                                                         
\begin{correctionbox}                                                                                                                                                                                                                    
	\begin{enumerate}                                                                                                                                      
	    \item[a)] 
	    \item[b)] 
	\end{enumerate}                                                                                                                                        
\end{correctionbox}                                                                                                                                                                                                                      
                                                                                                                                                                                                                                         
% ----------- EXERCISE 6 -------------                                                                                                                                                                                                   
\section{}                                                                                                                                                                                                                               
In the $\mathbb{R}$-vector space $F(\mathbb{R}, \mathbb{R})$, consider the functions:
\begin{align*}
f_n(x) = \sin(nx), n \geq 1
\end{align*}

\begin{enumerate}
    \item[a)] Show that for all $n \in \mathbb{N}^*$, the family $(f_1, \ldots, f_n)$ is linearly independent.
    \item[b)] Deduce that $F(\mathbb{R}, \mathbb{R})$ is an $\mathbb{R}$-vector space of infinite dimension.
\end{enumerate}
                                                                                                                                                                                                                                         
\begin{correctionbox}                                                                                                                                                                                                                    
 \begin{enumerate}                                                                                              
     \item[a)] 
     \item[b)] 
 \end{enumerate}                                                                                                
\end{correctionbox}                                                                                                                                                                                                                      
                                                                                                                                                                                                                                         
% ----------- EXERCISE 7 -------------                                                                                                                                                                                                   
\section{}                                                                                                                                                                                                                               
\begin{enumerate}
    \item Show that the application $f: \mathbb{R}^2 \to \mathbb{R}^3$ defined by $f(x, y) = (x - y, x, x + y)$ is linear.
    \item Show that $f$ is injective but not surjective.
    \item Determine a basis for $\text{Im}f$ (the image of $f$).
\end{enumerate}
                                                                                                                                                                                                                                         
\begin{correctionbox}                                                                                                                                                                                                                    
 \begin{enumerate}                                                                                                          
     \item 
     \item 
     \item 
 \end{enumerate}                                                                                                            
\end{correctionbox}                                                                                                                                                                                                                      
                                                                                                                                                                                                                                         
% ----------- EXERCISE 8 -------------                                                                                                                                                                                                   
\section{}                                                                                                                                                                                                                               
Let $f \in L(\mathbb{R}^3)$ defined by $f(x, y, z) = (2y + z, x - 4y, 3x)$.
\begin{enumerate}
    \item Determine the matrix $A$ of $f$ with respect to the canonical basis $B = \{e_1, e_2, e_3\}$ of $\mathbb{R}^3$.
    \item Let $v_1 = (1, 1, 1)$, $v_2 = (1, 1, 0)$, and $v_3 = (1, 0, 0)$ be vectors in $\mathbb{R}^3$.
    \begin{enumerate}
        \item[a)] Show that the family $B' = \{v_1, v_2, v_3\}$ is a basis of $\mathbb{R}^3$.
        \item[b)] Calculate $f(v_1)$, $f(v_2)$, and $f(v_3)$.
        \item[c)] Determine $A'$, the matrix of $f$ in the basis $B'$.
    \end{enumerate}
    \item \begin{enumerate}
        \item[a)] Determine the matrices $P$ and $P^{-1}$ where $P$ is the change of basis matrix from basis $B$ to basis $B'$.
        \item[b)] Using the change of basis formula, recalculate the matrix $A'$.
    \end{enumerate}
\end{enumerate}

\begin{correctionbox}                                                                                                                                                                                                                    
 \begin{enumerate}                                                                                                               
     \item 
     \item 
     \begin{enumerate}                                                                                                           
         \item[a)] 
         \item[b)] 
         \item[c)] 
     \end{enumerate}                                                                                                             
     \item \begin{enumerate}                                                                                                     
         \item[a)] 
         \item[b)] 
     \end{enumerate}                                                                                                             
 \end{enumerate}                                                                                                                 
\end{correctionbox}                                                                                                                                                                                                                      
                                                                                                                                                                                                                                         
% ----------- the documents finishes here body :D -------------                                                                                                                                                                          
\end{document}                                                                                                                                                                                                                           

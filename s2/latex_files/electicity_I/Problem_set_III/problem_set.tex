\documentclass[12pt]{article}                                                                                                           
\usepackage[utf8]{inputenc}                                                                                                             
\usepackage[T1]{fontenc}                                                                                                                
\usepackage{amsmath, amssymb}                                                                                                           
\usepackage{geometry}                                                                                                                   
\usepackage{xcolor}                                                                                                                     
\usepackage{lipsum}                                                                                                                     
\usepackage{titlesec}                                                                                                                   
\usepackage{fancyhdr}                                                                                                                   
\usepackage{tcolorbox}                                                                                                                  
\usepackage{graphicx}                                                                                                                   
\usepackage{float} % For precise figure placement                                                                                       
\usepackage{tikz}                                                                                                                       
\usepackage{circuitikz}
\usetikzlibrary{arrows.meta, decorations.markings}                                                                                      
                                                                                                                                        
\geometry{a4paper, margin=2.5cm}                                                                                                        
\pagestyle{fancy}                                                                                                                       
\fancyhf{}                                                                                                                              
\rhead{Problem Set III}                                                                                                                   
\lhead{Electricity I}                                                                                                                     
\cfoot{\thepage}                                                                                                                        
                                                                                                                                        
% Fancy titles                                                                                                                          
\titleformat{\section}{\normalfont\Large\bfseries}{Exercise \thesection:}{1em}{}                                                        
\titleformat{\subsection}{\normalfont\bfseries}{Correction:}{1em}{}                                                                     
                                                                                                                                        
% Custom box for correction                                                                                                             
\tcbuselibrary{listingsutf8}                                                                                                            
\newtcolorbox{correctionbox}{                                                                                                           
  colback=gray!5,                                                                                                                       
  colframe=black,                                                                                                                       
  fonttitle=\bfseries,                                                                                                                  
  title=Correction,                                                                                                                     
  breakable,                                                                                                                            
  before skip=10pt,                                                                                                                     
  after skip=10pt                                                                                                                       
}                                                                                                                                       
                                                                                                                                        
\begin{document}

% Cover info                                                                                                                            
\begin{center}
	\Large\textbf{UNIVERSITY IBN TOFAIL} \\[1em]
	\large\textit{Electricity I} \\[2em]
	\large\textit{Problem Set III} \\[0.5em]
\end{center}

\vspace{1cm}

% ----------- EXERCISE 1 -------------                                                                                                  
\section{}
A solid conducting sphere $ S_1 $ of radius $ R_1 $ is brought to a potential $ V_1 $. A second hollow conducting sphere $ S_2 $, with radius $ R_2 > R_1 $, is concentric with $ S_1 $. The sphere $ S_2 $ is brought to a potential $ V_2 $.
\begin{enumerate}
	\item Determine the expressions for:


	      - The charge $ Q_1 $ on the sphere $ S_1 $,

	      - The charge $ Q'_2 $ on the inner surface of $ S_2 $,

	      - The charge $ Q''_2 $ on the outer surface of $ S_2 $.

	\item Deduce the capacitance and influence coefficients. Verify that $ C_{11} > 0 $, $ C_{22} > 0 $, $ C_{12} < 0 $, and that $ C_{11} + C_{12} = 0 $.
	\item What happens if both spheres are brought to the same potential $ V_2 $?
\end{enumerate}


\begin{correctionbox}

\end{correctionbox}

% ----------- EXERCISE 2 -------------                                                                                                  
\section{}
A cylindrical capacitor of length $ L $ is formed by two coaxial cylinders $ A_1 $ and $ A_2 $, with radii $ R_1 $ and $ R_2 $ respectively ($ R_1 < R_2 $). The capacitor carries a charge $ Q $. The potentials of $ A_1 $ and $ A_2 $ are $ V_1 $ and $ V_2 $, respectively. Assuming $ L \gg R_2 $ to neglect edge effects, determine the capacitance $ C $ of this capacitor.


\begin{correctionbox}

\end{correctionbox}

% ----------- EXERCISE 3 -------------                                                                                                  
\section{}
Three capacitors are connected as shown in the figure below.
\begin{enumerate}
	\item What value must $ C_2 $ have so that the equivalent capacitance of the system equals $ C_2 $, given that $ C_1 = 3 \, \mu F $?
	\item A voltage $ U_0 = 400 \, \text{V} $ is applied between points $ A $ and $ B $. Determine the charge and voltage across each capacitor in the case where $ C_1 $ and $ C_2 $ have the values found in part (1).
\end{enumerate}

% Circuit Diagram
\begin{center}
	\begin{circuitikz}[scale=1.5]
		\ctikzset{bipoles/thickness=1}
		\ctikzset{bipoles/length=1.2cm}

		\draw (0,0) node[left] {$A$} to[short, o-] (1,0)
		to[C, l=$C_1$] (3,0)
		-- (5,0)
		to[short, -o] (6,0) node[right] {$B$};

		\draw (3,0) to[C, l=$C_2$, *-*] (5,0);
		\draw (3,0) -- (3,-1) to[C, l=$C_1$] (5,-1) -- (5,0);
	\end{circuitikz}
\end{center}

\begin{correctionbox}

\end{correctionbox}

% ----------- EXERCISE 4 -------------                                                                                                   
\section{}
Determine the electrostatic energy of a sphere of radius $ R $ charged with a uniform volumetric charge density $ \rho $ using two different methods:
\begin{enumerate}
	\item By using the expression for energy in terms of the potential.
	\item By using the expression for local energy density.
\end{enumerate}


\begin{correctionbox}

\end{correctionbox}

% ----------- EXERCISE 5 -------------                                                                                                   
\section{}
A capacitor is formed by two horizontal circular plates of surface area $ S $, parallel to each other, with radius $ R $ and separated by a distance $ e $. The capacitor is charged using a voltage generator $ V $. Express all results in terms of $ R $.
\begin{enumerate}
	\item Determine the charge $ Q $ acquired by the capacitor (its capacitance is $ C = \frac{\varepsilon_0 S}{e} $).
	\item Determine the energy $ W_c $ stored in the capacitor.
	\item What is the energy density $ W $? Deduce the intensity $ E $ of the electric field.
	\item Determine the energy $ W_G $ supplied by the generator. Compare it with $ W_c $ and interpret the result.
\end{enumerate}


\begin{correctionbox}

\end{correctionbox}

\end{document}

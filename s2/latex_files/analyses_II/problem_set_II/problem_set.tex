\documentclass[12pt]{article}                                                                                                           
\usepackage[utf8]{inputenc}                                                                                                             
\usepackage[T1]{fontenc}                                                                                                                
\usepackage{amsmath, amssymb}                                                                                                           
\usepackage{geometry}                                                                                                                   
\usepackage{xcolor}                                                                                                                     
\usepackage{lipsum}                                                                                                                     
\usepackage{titlesec}                                                                                                                   
\usepackage{fancyhdr}                                                                                                                   
\usepackage{tcolorbox}                                                                                                                  
\usepackage{graphicx}                                                                                                                   
\usepackage{float} % For precise figure placement                                                                                       
\usepackage{tikz}                                                                                                                       
\usetikzlibrary{arrows.meta, decorations.markings}                                                                                      
                                                                                                                                        
\geometry{a4paper, margin=2.5cm}                                                                                                        
\pagestyle{fancy}                                                                                                                       
\fancyhf{}                                                                                                                              
\rhead{Problem Set II}                                                                                                                   
\lhead{Analyses II}                                                                                                                     
\cfoot{\thepage}                                                                                                                        
                                                                                                                                        
% Fancy titles                                                                                                                          
\titleformat{\section}{\normalfont\Large\bfseries}{Exercise \thesection:}{1em}{}                                                        
\titleformat{\subsection}{\normalfont\bfseries}{Correction:}{1em}{}                                                                     
                                                                                                                                        
% Custom box for correction                                                                                                             
\tcbuselibrary{listingsutf8}                                                                                                            
\newtcolorbox{correctionbox}{                                                                                                           
  colback=gray!5,                                                                                                                       
  colframe=black,                                                                                                                       
  fonttitle=\bfseries,                                                                                                                  
  title=Correction,                                                                                                                     
  breakable,                                                                                                                            
  before skip=10pt,                                                                                                                     
  after skip=10pt                                                                                                                       
}                                                                                                                                       
                                                                                                                                        
\begin{document}

% Cover info                                                                                                                            
\begin{center}
	\Large\textbf{UNIVERSITY IBN TOFAIL} \\[1em]
	\large\textit{Analyses II} \\[2em]
	\large\textit{Problem Set II} \\[0.5em]
\end{center}

\vspace{1cm}

% ----------- EXERCISE 1 -------------                                                                                                  
\section{}
Let $n$ be a fixed positive integer. Consider the function $f: [0, 1] \rightarrow \mathbb{R}$ defined for all integers $i$ such that $0 \leq i \leq n-1$ by:

\[
	f(x) =
	\begin{cases}
		\frac{i^2}{n^2} & \text{if } x \in \left[\frac{i}{n}, \frac{i+1}{n}\right) \\
		1               & \text{if } x = 1.
	\end{cases}
\]

\begin{enumerate}
	\item Show that $f$ is a step function.
	\item Calculate, as a function of $n$, the integral $\int_0^1 f(x) \, dx$ of $f$ over $[0, 1]$.
\end{enumerate}

\begin{correctionbox}
	\begin{enumerate}
		\item $f$ is a constant fuction in all the intervals \left[\frac{i}{n}, \frac{i+1}{n}\right) and it is constant for $1$ and we know that the functin is $[0, 1] \rightarrow \mathbb{R}$.
		\item we have :
		      $$
			      \begin{align*}
				      \int_0^1 f(x) dx & = \sum_{0}^{1}  \int_{\frac{i}{n}}^{\frac{i+1}{n}} f(x) dx \\
				                       & = \sum_0^1 \frac{i^2}{n^2} \frac{1}{n^3}                   \\
				                       & = \frac{n(n-1)(2n-1)}{6} \frac{1}{n^3}                     \\
				                       & = \frac{(n-1)(2n-1)}{6n^2}
			      \end{align*}
		      $$
	\end{enumerate}
\end{correctionbox}

% ----------- EXERCISE 2 -------------                                                                                                  
\section{}
Let $a, b \in \mathbb{R}$ such that $a < b$. Consider the function $f: [a, b] \rightarrow \mathbb{R}$ defined by:

\[
	f(x) =
	\begin{cases}
		1 & \text{if } x \in \mathbb{Q}     \\
		0 & \text{if } x \notin \mathbb{Q}.
	\end{cases}
\]

\begin{enumerate}
	\item Show that if $\phi$ and $\psi$ are two step functions such that $\phi \leq f \leq \psi$, then $\phi \leq 0$ and $1 \leq \psi$.
	\item Deduce that $f$ is not integrable (in the Riemann sense) on $[a, b]$.
\end{enumerate}

\begin{correctionbox}
	the solutions for this exercise does exists with the course.
\end{correctionbox}

% ----------- EXERCISE 3 -------------                                                                                                  
\section{}
Consider the functions $f, g: \mathbb{R} \rightarrow \mathbb{R}$ defined by:

\[
	\forall x \in \mathbb{R}, \, f(x) = x^2 \text{ and } g(x) = e^x.
\]

\begin{enumerate}
	\item Show that the function $f$ is integrable on any closed bounded interval of $\mathbb{R}$.
	\item Using the definition, calculate the integral $\int_0^1 f(x) \, dx$.
	\item Repeat questions 1. and 2. for the function $g$.
	\item Study the integrability of the functions $h, k: [0, 2] \rightarrow \mathbb{R}$ defined by:
	      \[
		      h(x) = x - [x] \text{ and } k(x) =
		      \begin{cases}
			      \frac{1}{x} & \text{if } x \in (0, 2] \\
			      0           & \text{if } x = 0.
		      \end{cases}
	      \]
\end{enumerate}

\begin{correctionbox}
	\begin{enumerate}
		\item $f$ is continuous function cuz it is a polynomial, then by the Riemann sense it is integrable function.
		\item we have this :
		      $$
			      \begin{align*}
				      S_n               & = \sum_{1}^{n} f(\frac{i}{n}) \Delta x                \\
				                        & = \sum_{1}^{n} \left(\frac{i}{n}\right)^2 \frac{1}{n} \\
				                        & = \frac{(n-1)(2n-1)}{6n}                              \\
				                        & = \frac{2n^2 + 3n +1}{6n}                             \\
				                        & = \frac{2+\frac{3}{n}+\frac{1}{n^2}}{6n}              \\
				      \lim_{\infty}	S_n & = \frac{1}{3}
			      \end{align*}
		      $$ then we have $\int_{0}^{1}f(x) dx = \frac{1}{3}$.
		\item  $g$ is continuous function cuz it is and expenential functions, then by the Riemann sense it is integrable function.
		      \begin{align*}
			      S_n                & = \sum_{1}^{n} g(\frac{i}{n}) \Delta x                  \\
			                         & = \sum_{1}^{n} e^{\left(\frac{i}{n}\right)} \frac{1}{n} \\
			      \int_{0}^{1}e^x dx & = e -1
		      \end{align*}
		\item $h$ is integrabale in each interval of the form $[n,n+1]$.
	\end{enumerate}
\end{correctionbox}

% ----------- EXERCISE 4 -------------                                                                                                   
\section{}
Let $a, b \in \mathbb{R}$ such that $a < b$ and $f: [a, b] \rightarrow \mathbb{R}$ a bounded function.

\begin{enumerate}
	\item Show that if $f$ is zero except at a finite number of points in $[a, b]$, then $f$ is integrable on $[a, b]$, and $\int_a^b f(x) \, dx = 0$.
	\item Deduce that if $f$ is integrable and if we change the values of $f$ at a finite number of points in $[a, b]$, then $f$ is still integrable and the value of $\int_a^b f(x) \, dx$ does not change.
	\item Show that if $f$ is integrable on $[a, b]$, then its restriction to any interval $[c, d] \subset [a, b]$ is still integrable on $[c, d]$.
\end{enumerate}

\begin{correctionbox}
	\begin{enumerate}
		\item
		\item
		\item
	\end{enumerate}
\end{correctionbox}

% ----------- EXERCISE 5 -------------                                                                                                   
\section{}
Consider the function $f: [0, 1] \rightarrow \mathbb{R}$ defined by:
\[
	f(x) = \frac{1}{1 + x^2}.
\]

\begin{enumerate}
	\item Calculate the Darboux sums (lower and upper) of $f$ with respect to the following subdivision $S_0 = \{0, 1/2, 1\}$ of $[0, 1]$.
	\item Same question for the following subdivision $S_1 = \{0, 1/4, 1/2, 3/4, 1\}$ of $[0, 1]$.
	\item Assuming that $\frac{\pi}{4} = \int_0^1 \frac{dx}{1 + x^2}$, give bounds for $\pi$ using rational numbers.
	\item For a regular (uniform) subdivision with step $1/n$, what value of $n$ ensures an approximate value of $\pi$ by excess to within $10^{-3}$?
\end{enumerate}

\begin{correctionbox}
	\begin{enumerate}
		\item
		\item
		\item
		\item
	\end{enumerate}
\end{correctionbox}

% ----------- EXERCISE 6 -------------                                                                                                    
\section{}
Let $a, b \in \mathbb{R}$ such that $a < b$ and $f: [a, b] \rightarrow \mathbb{R}$ a continuous function.

\begin{enumerate}
	\item Show that if $\int_a^b f(x) \, dx = 0$, then $f$ vanishes at least once on $[a, b]$.
	\item Deduce that if $\int_a^b f(x) \, dx = \frac{b^2 - a^2}{2}$, then $f$ has at least one fixed point on $[a, b]$.
	\item Show that if $f$ is non-negative, then $\int_a^b f(x) \, dx = 0 \Leftrightarrow \forall x \in [a, b], f(x) = 0$.
	\item Deduce that if $P$ is a real polynomial, then $\int_a^b P^2(x) \, dx = 0 \Rightarrow P = 0$.
\end{enumerate}

\begin{correctionbox}
	\begin{enumerate}
		\item
		\item
		\item
		\item
	\end{enumerate}
\end{correctionbox}

% ----------- EXERCISE 7 -------------                                                                                                     
\section{}
Using Riemann sums, calculate the limit of the following sequences:

\begin{enumerate}
	\item $R_n = \sum_{k=1}^{n} \frac{n}{n^2 + k^2}$
	\item $S_n = \frac{\pi}{2n} \sum_{k=1}^{n} \sin\left(\frac{k\pi}{2n}\right)$
	\item $T_n = \frac{1}{n^3} \sum_{k=1}^{n} k^2 \sin\left(\frac{k\pi}{n}\right)$
	\item $U_n = \sum_{k=1}^{n} \frac{n+k}{n^2 + k}$
	\item $V_n = \frac{1}{n\sqrt{n}} \sum_{k=1}^{n} E(\sqrt{k})$
	\item $W_n = \sum_{k=1}^{2n} \frac{1}{n+k}$
	\item $X_n = \sum_{k=n}^{2n-1} \frac{1}{2k}$
	\item $Y_n = \left(\prod_{k=1}^{n} (n+k)^{1/n}\right)$
	\item $Z_n = \left(\frac{(2n)!}{n!n^n}\right)^{1/n}$
\end{enumerate}

\begin{correctionbox}
	\begin{enumerate}
		\item
		\item
		\item
		\item
		\item
		\item
		\item
		\item
		\item
	\end{enumerate}
\end{correctionbox}


% ----------- the documents finishes here body :D -------------                                                                         
\end{document}

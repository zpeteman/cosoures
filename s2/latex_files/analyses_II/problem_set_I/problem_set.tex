\documentclass[12pt]{article}                                                                                                           
\usepackage[utf8]{inputenc}                                                                                                             
\usepackage[T1]{fontenc}                                                                                                                
\usepackage{amsmath, amssymb}                                                                                                           
\usepackage{geometry}                                                                                                                   
\usepackage{xcolor}                                                                                                                     
\usepackage{lipsum}                                                                                                                     
\usepackage{titlesec}                                                                                                                   
\usepackage{fancyhdr}                                                                                                                   
\usepackage{tcolorbox}                                                                                                                  
\usepackage{graphicx}                                                                                                                   
\usepackage{float} % For precise figure placement                                                                                       
\usepackage{tikz}                                                                                                                       
\usetikzlibrary{arrows.meta, decorations.markings}                                                                                      
                                                                                                                                        
\geometry{a4paper, margin=2.5cm}                                                                                                        
\pagestyle{fancy}                                                                                                                       
\fancyhf{}                                                                                                                              
\rhead{Problem Set I}                                                                                                                   
\lhead{Analyses II}                                                                                                                     
\cfoot{\thepage}                                                                                                                        
                                                                                                                                        
% Fancy titles                                                                                                                          
\titleformat{\section}{\normalfont\Large\bfseries}{Exercise \thesection:}{1em}{}                                                        
\titleformat{\subsection}{\normalfont\bfseries}{Correction:}{1em}{}                                                                     
                                                                                                                                        
% Custom box for correction                                                                                                             
\tcbuselibrary{listingsutf8}                                                                                                            
\newtcolorbox{correctionbox}{                                                                                                           
  colback=gray!5,                                                                                                                       
  colframe=black,                                                                                                                       
  fonttitle=\bfseries,                                                                                                                  
  title=Correction,                                                                                                                     
  breakable,                                                                                                                            
  before skip=10pt,                                                                                                                     
  after skip=10pt                                                                                                                       
}                                                                                                                                       
                                                                                                                                        
\begin{document}

% Cover info                                                                                                                            
\begin{center}
	\Large\textbf{UNIVERSITY IBN TOFAIL} \\[1em]
	\large\textit{Analyses II} \\[2em]
	\large\textit{Problem Set I} \\[0.5em]
\end{center}

\vspace{1cm}

% ----------- EXERCISE 1 -------------                                                                                                  
\section{}
let $a,b \in \mathbb{R}$ such that $a < b$ and $f: [a,b] \to \mathbb{R}$ a funciton of class $C^n$ in $[a,b]$ and $n+1$ time derivable in $]a,b[$ such that :
$$
	f(a) = f(b) \text{ and } f'(a) = \dots = f^n(a) = 0
$$
Prove that there is $c \in ]a,b[$ such that $f^{(n+1)}(c) = 0$


\begin{correctionbox}
	usign the Rolls theorem we can prove that there is $c_1$ such that $f'(c_1)=0$.
	usign the theorem we cna prove it for a $c_n$ in a the interval $[a;c_{n-1}]$.
	then for $n$ we have $f^{(n)}(a) = f^{(n)}(c) = 0$ then usign the Rolls theorem
	we have $c_{n+1} \in [a,c_n]$ that $f^{n+1}(c_{n+1}=0$
\end{correctionbox}

% ----------- EXERCISE 2 -------------                                                                                                  
\section{}
\begin{enumerate}
	\item using the inequality of Taylor-Lagrange, prove that for all $n \in \mathbb{N}$
	      $$\forall x \in \mathbb{R} \left|e^x - \sum_{k =0}^{n} \frac{x^k}{k!} \right| \le \frac{|x|^{n+1}}{(n+1)!} e^{|x|}$$
	\item we deduce taht the sequence $(U_n)_n>= 0$ defined by :
	      $$u_n = 1 + \frac{1}{1!}+ \frac{1}{2!}+\dots+ \frac{1}{n!}$$
	      converges to $e$, $\lim_{\infty} \sum_{k=0}^{n} \frac{1}{k!}=e$
	\item prove by the same way taht the sequence $(v_n)$ converges to ln(2).
	      $$v_n = 1 - \frac{1}{2}+ \frac{1}{3}+\dots+ \frac{(-1)^{n-1}}{n}$$
\end{enumerate}

\begin{correctionbox}
	\begin{enumerate}
		\item we do have from the taylor Taylor-Lagrange formula :
		      $$\forall x \in \mathbb{R}, e^x - \sum_{k =0}^{n} \frac{x^k}{k!} = \frac{x^{n+1}}{(n+1)!} e^{c}$$
		      what gives us :
		      \begin{align*}
			               & \frac{x^{n+1}}{(n+1)!} e^{x} -  \frac{x^{n+1}}{(n+1)!} e^{c}  \ge 0 \\
			      \implies & x  - c                                              \ge 0
		      \end{align*}
		      and this is true because we have $c \in [0,x]$,
		      then :
		      $$\forall x \in \mathbb{R} \left|e^x - \sum_{k =0}^{n} \frac{x^k}{k!} \right| \le \frac{|x|^{n+1}}{(n+1)!} e^{|x|}$$
		\item for $x = 1$ we have :
		      $$\left|e - \sum_{k =0}^{n} \frac{1}{k!} \right| \le \frac{e}{(n+1)!}$$
		      when gettig closer to infinity we get :
		      \begin{align*}
			               & \left|e - \sum_{k =0}^{n} \frac{1}{k!} \right| \le 0 \\
			      \implies & \lim_{\infty} \sum_{k =0}^{n} \frac{1}{k!} = e
		      \end{align*}
		\item long story short :
		      \begin{align*}
			               & \left|\ln(2) - \sum_{k =0}^{n} \frac{(-1)^{k+1}}{k} \right| \le \frac{1}{n+1} \\
			      \implies & \lim_{\infty} \sum_{k =0}^{n} \frac{(-1)^{k+1}}{k} = \ln(2)
		      \end{align*}
	\end{enumerate}
\end{correctionbox}

% ----------- EXERCISE 3 -------------                                                                                                  
\section{}
find the local extremums in teh domaines of those functions:
\begin{itemize}
	\item $f(x) = x^3 - 3x^2 - 9x +2$
	\item $g(x) = e^x + (ln(x) - e - 1)x$
\end{itemize}

\begin{correctionbox}
	\begin{itemize}
		\item $x = 3, x = -1$
		\item $ 0.2 < x < 1.6   $
	\end{itemize}
\end{correctionbox}

% ----------- EXERCISE 4 -------------                                                                                                   
\section{}
\begin{enumerate}
	\item prove that the function $f(x) = ln(e^x +1)$ is convexe.
	\item we deduce :
	      $$\forall (a,b) \in \mathbb{R}\times \mathbb{R}, 1 + \sqrt{ab}\le (\sqrt{1+a})(\sqrt{1+b})$$
\end{enumerate}

\begin{correctionbox}
	\begin{enumerate}
		\item $f"(x)$ is always postitive then the function is convexe.
		\item we know that if $f$ is convexe then : $ f\left(\frac{a+b}{2}\right) = \frac{f(a)+f(b)}{2}$ \\
		      \text{let $a= \ln(a)$ and $b = \ln(b)$}
		      \begin{align*}
			      \\
			               & f\left(\frac{ln(a) + ln(b)}{2}\right) = \frac{f(ln(a))+f(ln(b))}{2} \\
			      \implies & \ln(\sqrt{ab} + 1) \le \ln(\sqrt{a +1}) \ln(\sqrt{b+1})             \\
			      \implies & \sqrt{ab} + 1 \le \sqrt{a +1} \sqrt{b+1}                            \\
		      \end{align*}
	\end{enumerate}
\end{correctionbox}

% ----------- EXERCISE 5 -------------                                                                                                   
\section{}
\begin{enumerate}
	\item prove that the function Ln doenst admet a $DL$ near 0.
	\item prove that the function $f: \mathbb{R} \to \mathbb{R}$ defined by:
	      $$
		      f(x)  =
		      \begin{cases}
			      x^3 ln x & ,x > 0   \\
			      0        & ,x \le 0
		      \end{cases}
	      $$ admet a $DL_2(0)$, but doesnt admet a $DL_3(0)$.
\end{enumerate}

\begin{correctionbox}
	\begin{enumerate}
		\item the $\ln$ doesnt admet a $DL$ near $0$ because it divergece and also its derivates diverges near $0$.
		\item the function does have a $DL_2$ near $0$ but not a $DL_3$, because from the $f(x)$ to the $f"(x)$ we do have $xln$ wich gives us $0$ is $0$ but the $f^{(3)}(x)$ contains $\ln$ and this function doesnt have a $DL$ near $0$.
	\end{enumerate}
\end{correctionbox}

% ----------- EXERCISE 6 -------------                                                                                                    
\section{}
calculate the limited developement at 0 of order n of the funcions:
\begin{enumerate}
	\item $f(x) = e^x + \frac{1}{1-x}, n = 3$
	\item $f(x) = cos(x) ln(1+x), n=4$
	\item $f(x) = \frac{sin x }{\sqrt{1+x}}, n = 4$
	\item $f(x) = tan x, n = 5$
	\item $f(x) = e^{sin x} , n =4$
	\item $f(x) = arctan x, n=5$
\end{enumerate}

\begin{correctionbox}
	\begin{enumerate}
		\item $f(x) = \frac{7}{6}x^3 + \frac{3}{4}x^2 + 2x +2 + \frac{f^{4}(c) x^4}{24}$
	\end{enumerate}
	pretty much all of them the same.
\end{correctionbox}

% ----------- EXERCISE 7 -------------                                                                                                     
\section{}
calculate teh developement of the function at n :
\begin{enumerate}
	\item $f(x) = cos x \text{ ,in } \frac{\pi}{4}$
	\item $f(x) = \frac{\sqrt{x+1}}{x} \text{ ,in } \infty \text{ with } n = 3$
	\item $f(x) = ln(x+ \sqrt{x^2+1} - ln x) \text{ ,in } \infty \text{ with } n = 5$
\end{enumerate}

\begin{correctionbox}
	\begin{enumerate}
		\item  $f(x) =  $
		\item
		\item
	\end{enumerate}
\end{correctionbox}

% ----------- EXERCISE 8 -------------                                                                                                     
\section{}
caldulate the limites:
\begin{enumerate}
	\item $\lim_0 \frac{sin x - x}{x^3}$
	\item $\lim_0 \left(\frac{a^x + b^x}{2}\right)^{\frac{1}{x}} \text{ ,with }  a, b \in \mathbb{R}^*_+$
	\item $\lim_0 \frac{ln(1+x) - sin x}{x}$
	\item $\lim_0 \frac{e^{x^2}- cos x}{x^2}$
	\item $\lim_{- \infty} (\sqrt{x^2 + 3x + 2} + x)$
	\item $\lim_0 \frac{ln(1+x) + 1 - e^x}{1 - cos x}$
\end{enumerate}

\begin{correctionbox}
	\begin{enumerate}
		\item
		\item
		\item
		\item
		\item
		\item
	\end{enumerate}
\end{correctionbox}

% ----------- EXERCISE 9 -------------                                                                                                      
\section{}
we considere the funtion $f: \mathbb{R} \to \mathbb{R}$ defined by :
$$f(x) = \sqrt{1 + x + x^3}$$

\begin{enumerate}
	\item calculate $DL_2(0)$ of the function $f$.
	\item we deduce the position of the tangent at teh point $x = 0$.
	\item Deletermen the equation of the asymprote in $\infty$.
\end{enumerate}


\begin{correctionbox}
	\begin{enumerate}
		\item
		\item
		\item
	\end{enumerate}
\end{correctionbox}


% ----------- the documents finishes here body :D -------------                                                                         
\end{document}

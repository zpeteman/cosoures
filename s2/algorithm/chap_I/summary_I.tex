\documentclass{article}
\usepackage[utf8]{inputenc}
\usepackage{enumitem}

\title{Summary of the PDF on Algorithmics}
\author{}
\date{}

\begin{document}

\maketitle

\section*{Introduction to Algorithmics}
\begin{description}[leftmargin=!,labelwidth=\widthof{\bfseries Key Concept}]
	\item[Key Concept] Introduction to algorithmics and general structure of an algorithm.
	\item[Content] An algorithm is a set of ordered instructions designed to solve a problem. It includes a header, a declarative part, and the body.
	\item[Notes] Algorithmics forms the foundation of computer programming.
\end{description}

\section*{Variables and Constants}
\begin{description}[leftmargin=!,labelwidth=\widthof{\bfseries Key Concept}]
	\item[Key Concept] Declaration and use of variables and constants.
	\item[Content] Variables store modifiable data, while constants hold fixed values. Examples include integers, real numbers, characters, strings, and booleans.
	\item[Notes] A variable must be declared with a specific type.
\end{description}

\section*{Data Types}
\begin{description}[leftmargin=!,labelwidth=\widthof{\bfseries Key Concept}]
	\item[Key Concept] Understanding different types of data.
	\item[Content] Common data types are integers (e.g., 1, -5), real numbers (e.g., 3.14), characters (e.g., 'A'), strings (e.g., "Hello"), and booleans (True/False).
	\item[Notes] Choosing the correct data type is essential for efficient memory usage and program functionality.
\end{description}

\section*{Basic Operations}
\begin{description}[leftmargin=!,labelwidth=\widthof{\bfseries Key Concept}]
	\item[Key Concept] Performing operations on data.
	\item[Content] Basic operations include arithmetic (e.g., addition, subtraction), relational (e.g., equality, inequality), and logical (e.g., AND, OR).
	\item[Notes] Operations must be compatible with the data type being used.
\end{description}

\section*{Control Structures}
\begin{description}[leftmargin=!,labelwidth=\widthof{\bfseries Key Concept}]
	\item[Key Concept] Use of control structures to manage program flow.
	\item[Content] Control structures include conditional statements (e.g., IF-THEN-ELSE) and loops (e.g., FOR, WHILE).
	\item[Notes] Proper use of control structures ensures that algorithms run efficiently and produce correct results.
\end{description}

\end{document}

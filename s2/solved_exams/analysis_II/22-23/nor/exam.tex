\documentclass[12pt]{article}

%----------------- Packages ------------------
\usepackage[utf8]{inputenc}
\usepackage[T1]{fontenc}
\usepackage{amsmath, amssymb}
\usepackage{geometry}
\usepackage{xcolor}
\usepackage{titlesec}
\usepackage{fancyhdr}
\usepackage[breakable]{tcolorbox}
\usepackage{graphicx}
\usepackage{tikz}
\usepackage{float}
\usetikzlibrary{arrows.meta, decorations.markings}

%----------------- Page Setup -----------------
\geometry{a4paper, margin=2.5cm}
\pagestyle{fancy}
\fancyhf{}
\rhead{Normal Exam — \textbf{22-23}}
\lhead{Analysis II}
\cfoot{\thepage}

%----------------- Title Styling --------------
\titleformat{\section}{\normalfont\Large\bfseries}{Exercise \thesection:}{1em}{}
\titleformat{\subsection}{\normalfont\bfseries}{Answer:}{1em}{}

%----------------- Custom Boxes ----------------
\tcbuselibrary{listingsutf8}
\newtcolorbox{answerbox}{
  colback=gray!10,
  colframe=black,
  fonttitle=\bfseries,
  title=Answer Area,
  breakable,
  before skip=10pt,
  after skip=10pt
}

%----------------- Document Start --------------
\begin{document}

%----------------- Exam Info -------------------
\begin{center}
  \Large\textbf{Ibn Tofail University} \\[1em]
  \large\textit{Analysis II — Normal Exam} \\[0.5em]
  \large\textit{Year: 22-23} \\[2em]
\end{center}

\vspace{0.5cm}

%----------------- EXERCISE 1 ------------------
\section{}
Consider the function $f: [1, 3] \to \mathbb{R}$ defined by:
$$f(x) = \frac{1}{x}$$

\begin{enumerate}
    \item Justify that $f$ is integrable (in the Riemann sense) on $[1, 3]$.
    \item Calculate the Darboux sums (lower and upper) $D^-_S(f)$ and $D^+_S(f)$ of $f$ with respect to the subdivision $S$ of $[1, 3]$ defined by $S = \{1, 2, 3\}$.
    \item State (without proving) the inequalities between $D^-_S(f)$, $D^+_S(f)$ and $\int_1^3 f(x) dx$.
    \item Deduce an approximation of $\ln 3$ by rational numbers.
\end{enumerate}

\newpage

\begin{answerbox}
\begin{enumerate}
  \item The function $ f(x) = \frac{1}{x} $ is continuous on $[1,3]$ because it is a rational function and the denominator does not vanish on this interval. Since every continuous function on a closed and bounded interval is Riemann integrable, $ f $ is integrable on $[1,3]$.

  \item We are given the subdivision $ S = \{1, 2, 3\} $. This divides $[1,3]$ into two subintervals: $[1,2]$ and $[2,3]$. On each subinterval, we compute the infimum and supremum of $ f(x) = \frac{1}{x} $:
  $$
  \begin{aligned}
      &\text{On } [1, 2]: \quad m_1 = \min_{x \in [1,2]} f(x) = \frac{1}{2}, \quad M_1 = \max_{x \in [1,2]} f(x) = 1 \\
      &\text{On } [2, 3]: \quad m_2 = \min_{x \in [2,3]} f(x) = \frac{1}{3}, \quad M_2 = \max_{x \in [2,3]} f(x) = \frac{1}{2}
  \end{aligned}
  $$

  Compute the lower Darboux sum:
  $$
  D_{-}(S, f) = m_1(2 - 1) + m_2(3 - 2) = \frac{1}{2} \cdot 1 + \frac{1}{3} \cdot 1 = \frac{5}{6}
  $$

  Compute the upper Darboux sum:
  $$
  D_{+}(S, f) = M_1(2 - 1) + M_2(3 - 2) = 1 \cdot 1 + \frac{1}{2} \cdot 1 = \frac{3}{2}
  $$

  \item The Darboux sums satisfy the following inequality:
  $$
  D_{-}(S, f) \leq \int_1^3 f(x)\,dx \leq D_{+}(S, f)
  $$
  That is:
  $$
  \frac{5}{6} \leq \int_1^3 \frac{1}{x}\,dx \leq \frac{3}{2}
  $$

  \item Since $ \int_1^3 \frac{1}{x}\,dx = \ln 3 $, we deduce:
  $$
  \frac{5}{6} \leq \ln 3 \leq \frac{3}{2}
  $$

  A reasonable approximation can be obtained by taking the average:
  $$
  \ln 3 \approx \frac{\frac{5}{6} + \frac{3}{2}}{2} = \frac{\frac{5}{6} + \frac{9}{6}}{2} = \frac{14}{12} \cdot \frac{1}{2} = \frac{7}{6}
  $$

  So, $ \ln 3 $ lies between $ \frac{5}{6} $ and $ \frac{3}{2} $, and one rational approximation is $ \frac{7}{6} $.
\end{enumerate}
\end{answerbox}

\newpage

%----------------- EXERCISE 2 ------------------
\section{}
Consider the function $G: \mathbb{R} \to \mathbb{R}$ defined by:
$$G(x) = \int_x^{2x} \frac{dt}{\sqrt{t^2 + 1}}$$

\begin{enumerate}
    \item Justify that $G$ is defined on $\mathbb{R}$. Also show that $G$ is an odd function.
    \item Verify that $G$ is differentiable on $\mathbb{R}$, and calculate its derivative $G'(x)$. (Hint: use any primitive $F$ of the function $t \mapsto \frac{1}{\sqrt{t^2 + 1}}$).
    \item Deduce that $G$ is strictly increasing on $\mathbb{R}$.
    \item Verify that $t^2 \leq t^2 + 1 \leq (t + 1)^2$ for all $t > 0$. Deduce the following inequality:
    $$\forall x > 0, \ln(2x + 1) - \ln(x + 1) \leq G(x) \leq \ln 2$$
    \item Deduce the limit $\lim_{x\to +\infty} G(x)$.
    \item Solve the equation $G(x) = 0$.
\end{enumerate}

\newpage

\begin{answerbox}
  \begin{enumerate}
    \item The function $ G(x) = \int_{x}^{2x} \frac{dt}{\sqrt{t^2 + 1}} $ is defined for all $ x \in \mathbb{R} $ because the integrand $ \frac{1}{\sqrt{t^2 + 1}} $ is continuous on $ \mathbb{R} $. Therefore, the integral over any finite interval exists. \\
    To show that $ G $ is odd, we compute:
    $$
    G(-x) = \int_{-x}^{-2x} \frac{dt}{\sqrt{t^2 + 1}}
    $$
    Perform the substitution $ u = -t $, so $ du = -dt $, and the limits become:
    $$
    G(-x) = \int_{x}^{2x} \frac{-du}{\sqrt{u^2 + 1}} = -\int_{x}^{2x} \frac{du}{\sqrt{u^2 + 1}} = -G(x)
    $$
    Hence, $ G $ is an odd function.

    \item Since the integrand $ f(t) = \frac{1}{\sqrt{t^2 + 1}} $ is continuous on $ \mathbb{R} $, by the Fundamental Theorem of Calculus, $ G(x) $ is differentiable on $ \mathbb{R} $. Let $ F $ be an antiderivative of $ f $, then:
    $$
    G(x) = F(2x) - F(x)
    $$
    Differentiating using the chain rule:
    $$
    G'(x) = 2F'(2x) - F'(x) = 2f(2x) - f(x) = \frac{2}{\sqrt{(2x)^2 + 1}} - \frac{1}{\sqrt{x^2 + 1}}
    $$

    \item From the previous part:
    $$
    G'(x) = \frac{2}{\sqrt{4x^2 + 1}} - \frac{1}{\sqrt{x^2 + 1}}
    $$
    We analyze the sign of $ G'(x) $. For $ x > 0 $, clearly:
    $$
    \frac{2}{\sqrt{4x^2 + 1}} > \frac{1}{\sqrt{x^2 + 1}} \quad \Rightarrow \quad G'(x) > 0
    $$
    For $ x < 0 $, since $ G $ is odd, $ G' $ is even (you can verify this), so $ G'(x) > 0 $ also holds. Thus, $ G $ is strictly increasing on $ \mathbb{R} $.

    \item First, observe that for $ t > 0 $,
    $$
    t^2 \leq t^2 + 1 \leq (t + 1)^2
    $$
    Taking square roots:
    $$
    t \leq \sqrt{t^2 + 1} \leq t + 1
    $$
    Inverting (and reversing inequalities):
    $$
    \frac{1}{t + 1} \leq \frac{1}{\sqrt{t^2 + 1}} \leq \frac{1}{t}
    $$
    Now integrate from $ x $ to $ 2x $:
    $$
    \int_x^{2x} \frac{dt}{t + 1} \leq G(x) \leq \int_x^{2x} \frac{dt}{t}
    $$
    Compute both sides:
    $$
    \ln(2x + 1) - \ln(x + 1) \leq G(x) \leq \ln(2x) - \ln(x) = \ln 2
    $$

    \item From the inequality:
    $$
    \ln(2x + 1) - \ln(x + 1) \leq G(x) \leq \ln 2
    $$
    As $ x \to +\infty $, the left-hand side tends to:
    $$
    \ln\left(\frac{2x + 1}{x + 1}\right) \to \ln 2
    $$
    So by the Squeeze Theorem:
    $$
    \lim_{x \to +\infty} G(x) = \ln 2
    $$

    \item We solve $ G(x) = 0 $, i.e.,
    $$
    \int_x^{2x} \frac{dt}{\sqrt{t^2 + 1}} = 0
    $$
    This implies $ x = 0 $, since the integrand is positive for all $ t $, and the only way the integral is zero is if the lower and upper limits are equal. Therefore:
    $$
    x = 0
    $$
\end{enumerate}
\end{answerbox}

\newpage

%----------------- EXERCISE 3 ------------------
\section{}
For all $n \in \mathbb{N}$, let:
$$I_n = \int_0^1 (1 - t^2)^n dt$$

\begin{enumerate}
    \item Justify the existence of the integral $I_n$ for all $n \in \mathbb{N}$.
    \item Show that $\forall n \in \mathbb{N}, I_{n+1} = \frac{2n + 2}{2n + 3} \cdot I_n$.
    \item Deduce that $\forall n \in \mathbb{N}, I_n = \frac{2^n(n!)^2}{(2n + 1)!}$.
    \item Using Newton's binomial formula, show that $\forall n \in \mathbb{N}, I_n = \sum_{k=0}^n \binom{n}{k} \cdot \frac{(-1)^k}{2k + 1}$.
\end{enumerate}

\newpage

\begin{answerbox}
  \begin{enumerate}
    \item The function $ (1 - t^2)^n $ is continuous on the closed interval $[0, 1]$ for all $ n \in \mathbb{N} $, because it is a composition and power of continuous functions. Since any continuous function on a closed and bounded interval is Riemann integrable, the integral
    $$
    I_n = \int_0^1 (1 - t^2)^n \, dt
    $$
    exists for all $ n \in \mathbb{N} $.

    \item We use integration by parts to prove the recurrence relation. Let:
    $$
    I_{n+1} = \int_0^1 (1 - t^2)^{n+1} dt
    $$
    Write:
    $$
    (1 - t^2)^{n+1} = (1 - t^2)(1 - t^2)^n
    $$
    Now integrate by parts. Set:
    $$
    u = (1 - t^2)^{n+1}, \quad dv = dt \quad \Rightarrow \quad du = -2(n+1)t(1 - t^2)^n dt, \quad v = t
    $$
    Then:
    $$
    I_{n+1} = \left. t(1 - t^2)^{n+1} \right|_0^1 + 2(n+1) \int_0^1 t^2 (1 - t^2)^n dt
    $$
    The boundary term vanishes at both ends. So:
    $$
    I_{n+1} = 2(n+1) \int_0^1 t^2 (1 - t^2)^n dt
    $$
    Now observe that:
    $$
    t^2 = 1 - (1 - t^2)
    $$
    Hence:
    $$
    I_{n+1} = 2(n+1) \left[ \int_0^1 (1 - t^2)^n dt - \int_0^1 (1 - t^2)^{n+1} dt \right]
    = 2(n+1)(I_n - I_{n+1})
    $$
    Solving for $ I_{n+1} $:
    $$
    I_{n+1}(1 + 2(n+1)) = 2(n+1)I_n \quad \Rightarrow \quad I_{n+1} = \frac{2(n+1)}{2n + 3} I_n
    $$

    \item We now deduce the general formula:
    $$
    I_n = \frac{2^n (n!)^2}{(2n + 1)!}
    $$
    This can be proved by induction using the recurrence:
    $$
    I_{n+1} = \frac{2(n+1)}{2n + 3} I_n
    $$
    The base case $ n = 0 $:
    $$
    I_0 = \int_0^1 (1 - t^2)^0 dt = \int_0^1 1 dt = 1
    $$
    Also:
    $$
    \frac{2^0 (0!)^2}{1!} = \frac{1}{1} = 1
    $$
    Assume the formula holds for $ n $. Then:
    $$
    I_{n+1} = \frac{2(n+1)}{2n + 3} \cdot \frac{2^n (n!)^2}{(2n + 1)!}
    = \frac{2^{n+1} (n+1)(n!)^2}{(2n + 3)(2n + 1)!}
    = \frac{2^{n+1} ((n+1)!)^2}{(2n + 3)!}
    $$
    Thus, the formula holds for all $ n \in \mathbb{N} $.

    \item Using Newton's binomial formula:
    $$
    (1 - t^2)^n = \sum_{k=0}^n \binom{n}{k} (-1)^k t^{2k}
    $$
    Integrate term by term from 0 to 1:
    $$
    \begin{aligned}
    I_n & = \int_0^1 (1 - t^2)^n dt = \int_0^1 \sum_{k=0}^n \binom{n}{k} (-1)^k t^{2k} dt \\
    & = \sum_{k=0}^n \binom{n}{k} (-1)^k \int_0^1 t^{2k} dt \\
    & = \sum_{k=0}^n \binom{n}{k} \frac{(-1)^k}{2k + 1}
    \end{aligned}
    $$
\end{enumerate}
\end{answerbox}

\newpage

%----------------- END ------------------
\end{document}
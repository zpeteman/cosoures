\documentclass[12pt]{article}

%----------------- Packages ------------------
\usepackage[utf8]{inputenc}
\usepackage[T1]{fontenc}
\usepackage{amsmath, amssymb}
\usepackage{geometry}
\usepackage{xcolor}
\usepackage{titlesec}
\usepackage{fancyhdr}
\usepackage[breakable]{tcolorbox}
\usepackage{graphicx}
\usepackage{tikz}
\usepackage{float}
\usetikzlibrary{arrows.meta, decorations.markings}

%----------------- Page Setup -----------------
\geometry{a4paper, margin=2.5cm}
\pagestyle{fancy}
\fancyhf{}
\rhead{Normal Exam — \textbf{22-23}}
\lhead{Analysis II}
\cfoot{\thepage}

%----------------- Title Styling --------------
\titleformat{\section}{\normalfont\Large\bfseries}{Exercise \thesection:}{1em}{}
\titleformat{\subsection}{\normalfont\bfseries}{Answer:}{1em}{}

%----------------- Custom Boxes ----------------
\tcbuselibrary{listingsutf8}
\newtcolorbox{answerbox}{
  colback=gray!10,
  colframe=black,
  fonttitle=\bfseries,
  title=Answer Area,
  breakable,
  before skip=10pt,
  after skip=10pt
}

%----------------- Document Start --------------
\begin{document}

%----------------- Exam Info -------------------
\begin{center}
  \Large\textbf{Ibn Tofail University} \\[1em]
  \large\textit{Analysis II — Normal Exam} \\[0.5em]
  \large\textit{Year: 22-23} \\[2em]
\end{center}

\vspace{0.5cm}

%----------------- EXERCISE 1 ------------------
\section{}
Consider the function $f: [1, 3] \to \mathbb{R}$ defined by:
$$f(x) = \frac{1}{x}$$

\begin{enumerate}
    \item Justify that $f$ is integrable (in the Riemann sense) on $[1, 3]$.
    \item Calculate the Darboux sums (lower and upper) $D^-_S(f)$ and $D^+_S(f)$ of $f$ with respect to the subdivision $S$ of $[1, 3]$ defined by $S = \{1, 2, 3\}$.
    \item State (without proving) the inequalities between $D^-_S(f)$, $D^+_S(f)$ and $\int_1^3 f(x) dx$.
    \item Deduce an approximation of $\ln 3$ by rational numbers.
\end{enumerate}

\newpage

\begin{answerbox}


\end{answerbox}

\newpage

%----------------- EXERCISE 2 ------------------
\section{}
Consider the function $G: \mathbb{R} \to \mathbb{R}$ defined by:
$$G(x) = \int_x^{2x} \frac{dt}{\sqrt{t^2 + 1}}$$

\begin{enumerate}
    \item Justify that $G$ is defined on $\mathbb{R}$. Also show that $G$ is an odd function.
    \item Verify that $G$ is differentiable on $\mathbb{R}$, and calculate its derivative $G'(x)$. (Hint: use any primitive $F$ of the function $t \mapsto \frac{1}{\sqrt{t^2 + 1}}$).
    \item Deduce that $G$ is strictly increasing on $\mathbb{R}$.
    \item Verify that $t^2 \leq t^2 + 1 \leq (t + 1)^2$ for all $t > 0$. Deduce the following inequality:
    $$\forall x > 0, \ln(2x + 1) - \ln(x + 1) \leq G(x) \leq \ln 2$$
    \item Deduce the limit $\lim_{x\to +\infty} G(x)$.
    \item Solve the equation $G(x) = 0$.
\end{enumerate}

\newpage

\begin{answerbox}


\end{answerbox}

\newpage

%----------------- EXERCISE 3 ------------------
\section{}
For all $n \in \mathbb{N}$, let:
$$I_n = \int_0^1 (1 - t^2)^n dt$$

\begin{enumerate}
    \item Justify the existence of the integral $I_n$ for all $n \in \mathbb{N}$.
    \item Show that $\forall n \in \mathbb{N}, I_{n+1} = \frac{2n + 2}{2n + 3} \cdot I_n$.
    \item Deduce that $\forall n \in \mathbb{N}, I_n = \frac{2^n(n!)^2}{(2n + 1)!}$.
    \item Using Newton's binomial formula, show that $\forall n \in \mathbb{N}, I_n = \sum_{k=0}^n \binom{n}{k} \cdot \frac{(-1)^k}{2k + 1}$.
\end{enumerate}

\newpage

\begin{answerbox}


\end{answerbox}

\newpage

%----------------- END ------------------
\end{document}
\documentclass[12pt]{article}

%----------------- Packages ------------------
\usepackage[utf8]{inputenc}
\usepackage[T1]{fontenc}
\usepackage{amsmath, amssymb}
\usepackage{geometry}
\usepackage{xcolor}
\usepackage{titlesec}
\usepackage{fancyhdr}
\usepackage[breakable]{tcolorbox}
\usepackage{graphicx}
\usepackage{tikz}
\usepackage{float}
\usetikzlibrary{arrows.meta, decorations.markings}

%----------------- Page Setup -----------------
\geometry{a4paper, margin=2.5cm}
\pagestyle{fancy}
\fancyhf{}
\rhead{Make-up Exam — \textbf{22-23}}
\lhead{Analysis II}
\cfoot{\thepage}

%----------------- Title Styling --------------
\titleformat{\section}{\normalfont\Large\bfseries}{Exercise \thesection:}{1em}{}
\titleformat{\subsection}{\normalfont\bfseries}{Answer:}{1em}{}

%----------------- Custom Boxes ----------------
\tcbuselibrary{listingsutf8}
\newtcolorbox{answerbox}{
  colback=gray!10,
  colframe=black,
  fonttitle=\bfseries,
  title=Answer Area,
  breakable,
  before skip=10pt,
  after skip=10pt
}

%----------------- Document Start --------------
\begin{document}

%----------------- Exam Info -------------------
\begin{center}
  \Large\textbf{Ibn Tofail University} \\[1em]
  \large\textit{Analysis II — Make-up Exam} \\[0.5em]
  \large\textit{Year: 22-23} \\[2em]
\end{center}

\vspace{0.5cm}

%----------------- EXERCISE 1 ------------------
\section{}
Let $a, b \in \mathbb{R}$ such that $a < b$ and $f: [a, b] \rightarrow \mathbb{R}$ a non-zero continuous function on $[a, b]$ such that
\begin{align*}
\int_a^b f(x) , dx = 0 \quad \text{and} \quad \int_a^b x f(x) , dx = 0.
\end{align*}
\begin{enumerate}
\item Using the mean value theorem, show that there exists at least one $c \in [a, b]$ such that $f(c) = 0$.
\item Verify that $f$ necessarily changes sign on $[a, b]$. (Hint: use the fact that $f$ is continuous and not identically zero on $[a, b]$).
\item Suppose that $c$ is the only point in $[a, b]$ such that $f(c) = 0$. In this case, we have
\begin{align*}
\forall x \in [a, c[, f(x) < 0 \quad \text{and} \quad \forall x \in ]c, b], f(x) > 0.
\end{align*}
\begin{enumerate}
\item Show that $\int_a^b (x-c)f(x) , dx > 0$.
\item Deduce a contradiction.
\end{enumerate}
\item State a conclusion summarizing the preceding results.
\end{enumerate}

\newpage

\begin{answerbox}
\begin{enumerate}
  \item \textbf{Using the mean value theorem, show that there exists at least one $ c \in [a, b] $ such that $ f(c) = 0 $.}
  
  Since $ f $ is continuous on $[a, b]$ and $\int_a^b f(x)\,dx = 0$, by the Mean Value Theorem for integrals, there exists a point $ c \in [a, b] $ such that:
  $$
  \int_a^b f(x)\,dx = f(c)(b - a).
  $$
  Given that $ \int_a^b f(x)\,dx = 0 $ and $ b - a > 0 $, it follows that:
  $$
  f(c)(b - a) = 0 \Rightarrow f(c) = 0.
  $$
  Therefore, there exists at least one $ c \in [a, b] $ such that $ f(c) = 0 $.

  \item \textbf{Verify that $ f $ necessarily changes sign on $[a, b]$.}

  Suppose $ f $ does not change sign on $[a, b]$. Then either $ f(x) \geq 0 $ or $ f(x) \leq 0 $ for all $ x \in [a, b] $. Since $ f $ is continuous and not identically zero, then the integral $ \int_a^b f(x)\,dx \neq 0 $, which contradicts the given condition $ \int_a^b f(x)\,dx = 0 $. Hence, $ f $ must change sign on $[a, b]$.

  \item \textbf{Suppose that $ c $ is the only point in $[a, b]$ such that $ f(c) = 0 $. In this case, we have:}
  $$
  \forall x \in [a, c),\ f(x) < 0 \quad \text{and} \quad \forall x \in (c, b],\ f(x) > 0.
  $$

  \begin{enumerate}
      \item \textbf{Show that $ \int_a^b (x - c)f(x)\,dx > 0 $.}

      Consider the function $ g(x) = (x - c)f(x) $. Note that:
      
          
- On $ [a, c) $, $ x - c < 0 $ and $ f(x) < 0 $, so $ g(x) > 0 $.
          
- At $ x = c $, $ g(c) = 0 $.
          
- On $ (c, b] $, $ x - c > 0 $ and $ f(x) > 0 $, so $ g(x) > 0 $.
      
      Thus, $ g(x) \geq 0 $ on $[a, b]$, and $ g(x) > 0 $ on a set of positive measure. Since $ g $ is continuous and non-negative with positive values on a subset of $[a, b]$, we conclude:
      $$
      \int_a^b (x - c)f(x)\,dx > 0.
      $$

      \item \textbf{Deduce a contradiction.}

      Now expand the integral:
      $$
      \int_a^b (x - c)f(x)\,dx = \int_a^b x f(x)\,dx - c \int_a^b f(x)\,dx.
      $$
      But from the problem statement:
      $$
      \int_a^b f(x)\,dx = 0 \quad \text{and} \quad \int_a^b x f(x)\,dx = 0,
      $$
      so the entire expression becomes:
      $$
      \int_a^b (x - c)f(x)\,dx = 0 - c \cdot 0 = 0.
      $$
      This contradicts the earlier result that the integral is strictly positive. Therefore, our assumption that $ c $ is the only zero of $ f $ must be false.

  \end{enumerate}

  \item \textbf{State a conclusion summarizing the preceding results.}

  From the above, since assuming that $ f $ has only one zero leads to a contradiction, we conclude that $ f $ must have at least two distinct zeros in $[a, b]$. Additionally, since $ f $ changes sign and is continuous, it must vanish at least twice in the interval $[a, b]$.

\end{enumerate}
\end{answerbox}

\newpage

%----------------- EXERCISE 2 ------------------
\section{}
Consider the two integrals $I$ and $J$ defined by:
\begin{align*}
I = \int_0^{\pi/2} \frac{\cos x}{\cos x + \sin x} , dx \quad \text{and} \quad J = \int_0^{\pi/2} \frac{\sin x}{\cos x + \sin x} , dx.
\end{align*}
\begin{enumerate}
\item Using an appropriate change of variable, show that $I = J$.
\item Calculate $I + J$. Deduce the common value of $I$ and $J$.
\item Deduce (using an appropriate change of variable) the integral $\int_0^1 \frac{1}{\sqrt{1-t^2} + t} , dt$.
\end{enumerate}

\newpage

\begin{answerbox}
  \begin{enumerate}
    \item \textbf{Using an appropriate change of variable, show that $ I = J $.}

    Recall:
    $$
    I = \int_0^{\frac{\pi}{2}} \frac{\cos x}{\cos x + \sin x}\,dx, \quad J = \int_0^{\frac{\pi}{2}} \frac{\sin x}{\cos x + \sin x}\,dx.
    $$

    Consider the substitution $ u = \frac{\pi}{2} - x $. Then when $ x = 0 $, $ u = \frac{\pi}{2} $; and when $ x = \frac{\pi}{2} $, $ u = 0 $. Also, $ dx = -du $.

    Apply this substitution to $ I $:
    $$
    \begin{aligned}
    I & = \int_0^{\frac{\pi}{2}} \frac{\cos x}{\cos x + \sin x}\,dx \\ 
    & = \int_{\frac{\pi}{2}}^{0} \frac{\cos\left(\frac{\pi}{2} - u\right)}{\cos\left(\frac{\pi}{2} - u\right) + \sin\left(\frac{\pi}{2} - u\right)}(-du) \\
    & = \int_0^{\frac{\pi}{2}} \frac{\sin u}{\sin u + \cos u}\,du \\ 
    & = J.
    \end{aligned}
    $$
    Therefore, $ I = J $.

    \item \textbf{Calculate $ I + J $. Deduce the common value of $ I $ and $ J $.}

    We compute:
    $$
    \begin{aligned}
    I + J & = \int_0^{\frac{\pi}{2}} \frac{\cos x}{\cos x + \sin x}\,dx + \int_0^{\frac{\pi}{2}} \frac{\sin x}{\cos x + \sin x}\,dx \\
    & = \int_0^{\frac{\pi}{2}} \frac{\cos x + \sin x}{\cos x + \sin x}\,dx \\
    & = \int_0^{\frac{\pi}{2}} 1\,dx \\
    & = \frac{\pi}{2} \\
    \end{aligned}
    $$

    Since $ I = J $, we have:
    $$
    2I = \frac{\pi}{2} \Rightarrow I = J = \frac{\pi}{4}.
    $$

    \item \textbf{Deduce (using an appropriate change of variable) the integral $ \int_0^1 \frac{1}{\sqrt{1 - t^2} + t}\,dt $.}

    Consider the substitution $ t = \sin x $. Then $ dt = \cos x\,dx $, and when $ t = 0 $, $ x = 0 $; when $ t = 1 $, $ x = \frac{\pi}{2} $. The integral becomes:
    $$
    \begin{aligned}
    \int_0^1 \frac{1}{\sqrt{1 - t^2} + t}\,dt  &= \int_0^{\frac{\pi}{2}} \frac{\cos x}{\sqrt{1 - \sin^2 x} + \sin x}\,dx \\
    &= \int_0^{\frac{\pi}{2}} \frac{\cos x}{\cos x + \sin x}\,dx \\ 
    &= I \\ 
    &= \frac{\pi}{4} \\.
    \end{aligned} 
    $$

\end{enumerate}
\end{answerbox}

\newpage

%----------------- EXERCISE 3 ------------------
\section{}
For all $n \in \mathbb{N}$, we define:
\begin{align*}
I_n = \int_0^1 \frac{x^n}{x+1} , dx.
\end{align*}
\begin{enumerate}
\item Justify the existence of $I_n$ for all $n \in \mathbb{N}$. Then calculate $I_0$.
\item Verify that for all $n \in \mathbb{N}$, we have the following inequality:
\begin{align*}
\forall x \in [0, 1], \frac{x^n}{2} \leq \frac{x^n}{x+1} \leq x^n.
\end{align*}
\item Deduce that $\lim_{n \to +\infty} I_n = 0$.
\item Calculate for all $n \in \mathbb{N}$, the value of $I_n + I_{n+1}$.
\item Deduce that
\begin{align*}
\lim_{n \to +\infty} \left( \sum_{k=1}^n \frac{(-1)^{k+1}}{k} \right) = \ln 2.
\end{align*}
\end{enumerate}

\newpage

\begin{answerbox}
  \begin{enumerate}
    \item \textbf{Justify the existence of $ I_n $ for all $ n \in \mathbb{N} $. Then calculate $ I_0 $.}

    The function $ f_n(x) = \frac{x^n}{x + 1} $ is continuous on $[0, 1]$ for all $ n \in \mathbb{N} $, since both numerator and denominator are continuous and the denominator does not vanish on $[0, 1]$. Therefore, the integral
    $$
    I_n = \int_0^1 \frac{x^n}{x + 1}\,dx
    $$
    exists for all $ n \in \mathbb{N} $.

    For $ n = 0 $, we have:
    $$
    I_0 = \int_0^1 \frac{1}{x + 1}\,dx = \Big[\ln|x + 1|\Big]_0^1 = \ln(2) - \ln(1) = \ln(2).
    $$

    \item \textbf{Verify that for all $ n \in \mathbb{N} $, we have the inequality: $ \forall x \in [0, 1],\ \frac{x^n}{x + 1} \leq x^n $.}

    On $[0, 1]$, we know that $ x + 1 \geq 1 $, so $ \frac{1}{x + 1} \leq 1 $. Multiplying both sides by $ x^n \geq 0 $, we get:
    $$
    \frac{x^n}{x + 1} \leq x^n, \quad \forall x \in [0, 1].
    $$

    \item \textbf{Deduce that $ \lim_{n \to +\infty} I_n = 0 $.}

    Since $ \frac{x^n}{x + 1} \leq x^n $, integrating both sides over $[0, 1]$ gives:
    $$
    0 \leq I_n = \int_0^1 \frac{x^n}{x + 1}\,dx \leq \int_0^1 x^n\,dx = \frac{1}{n + 1}.
    $$
    As $ n \to \infty $, $ \frac{1}{n + 1} \to 0 $, so by the squeeze theorem:
    $$
    \lim_{n \to +\infty} I_n = 0.
    $$

    \item \textbf{Calculate for all $ n \in \mathbb{N} $, the value of $ I_n + I_{n+1} $.}

    We compute:
    $$
    \begin{aligned}
    I_n + I_{n+1} &= \int_0^1 \frac{x^n}{x + 1}\,dx + \int_0^1 \frac{x^{n+1}}{x + 1}\,dx \\
    &= \int_0^1 \frac{x^n(1 + x)}{x + 1}\,dx \\ 
    &= \int_0^1 x^n\,dx = \frac{1}{n + 1} \\.
    \end{aligned}
    $$

    \item \textbf{Deduce that $ \lim_{n \to +\infty} \sum_{k=1}^n \frac{(-1)^{k+1}}{k} = \ln 2 $.}

    From the recurrence:
    $$
    I_k + I_{k+1} = \frac{1}{k + 1},
    $$
    summing from $ k = 0 $ to $ n - 1 $, we get:
    $$
    \sum_{k=0}^{n-1} (I_k + I_{k+1}) = \sum_{k=0}^{n-1} \frac{1}{k + 1} = \sum_{k=1}^{n} \frac{1}{k}.
    $$
    But the left-hand side telescopes:
    $$
    \sum_{k=0}^{n-1} (I_k + I_{k+1}) = I_0 + 2(I_1 + I_2 + \cdots + I_{n-1}) + I_n.
    $$
    Alternatively, consider the partial sums of the alternating harmonic series:
    $$
    \sum_{k=1}^n \frac{(-1)^{k+1}}{k} = 1 - \frac{1}{2} + \frac{1}{3} - \cdots + \frac{(-1)^{n+1}}{n}.
    $$
    It is known that this converges to $ \ln 2 $ as $ n \to \infty $. Therefore:
    $$
    \lim_{n \to +\infty} \sum_{k=1}^n \frac{(-1)^{k+1}}{k} = \ln 2.
    $$

\end{enumerate}
\end{answerbox}

%----------------- END ------------------
\end{document}
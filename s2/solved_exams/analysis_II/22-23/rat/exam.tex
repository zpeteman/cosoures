\documentclass[12pt]{article}

%----------------- Packages ------------------
\usepackage[utf8]{inputenc}
\usepackage[T1]{fontenc}
\usepackage{amsmath, amssymb}
\usepackage{geometry}
\usepackage{xcolor}
\usepackage{titlesec}
\usepackage{fancyhdr}
\usepackage[breakable]{tcolorbox}
\usepackage{graphicx}
\usepackage{tikz}
\usepackage{float}
\usetikzlibrary{arrows.meta, decorations.markings}

%----------------- Page Setup -----------------
\geometry{a4paper, margin=2.5cm}
\pagestyle{fancy}
\fancyhf{}
\rhead{Make-up Exam — \textbf{22-23}}
\lhead{Analysis II}
\cfoot{\thepage}

%----------------- Title Styling --------------
\titleformat{\section}{\normalfont\Large\bfseries}{Exercise \thesection:}{1em}{}
\titleformat{\subsection}{\normalfont\bfseries}{Answer:}{1em}{}

%----------------- Custom Boxes ----------------
\tcbuselibrary{listingsutf8}
\newtcolorbox{answerbox}{
  colback=gray!10,
  colframe=black,
  fonttitle=\bfseries,
  title=Answer Area,
  breakable,
  before skip=10pt,
  after skip=10pt
}

%----------------- Document Start --------------
\begin{document}

%----------------- Exam Info -------------------
\begin{center}
  \Large\textbf{Ibn Tofail University} \\[1em]
  \large\textit{Analysis II — Make-up Exam} \\[0.5em]
  \large\textit{Year: 22-23} \\[2em]
\end{center}

\vspace{0.5cm}

%----------------- EXERCISE 1 ------------------
\section{}
Let $a, b \in \mathbb{R}$ such that $a < b$ and $f: [a, b] \rightarrow \mathbb{R}$ a non-zero continuous function on $[a, b]$ such that
\begin{align*}
\int_a^b f(x) , dx = 0 \quad \text{and} \quad \int_a^b x f(x) , dx = 0.
\end{align*}
\begin{enumerate}
\item Using the mean value theorem, show that there exists at least one $c \in [a, b]$ such that $f(c) = 0$.
\item Verify that $f$ necessarily changes sign on $[a, b]$. (Hint: use the fact that $f$ is continuous and not identically zero on $[a, b]$).
\item Suppose that $c$ is the only point in $[a, b]$ such that $f(c) = 0$. In this case, we have
\begin{align*}
\forall x \in [a, c[, f(x) < 0 \quad \text{and} \quad \forall x \in ]c, b], f(x) > 0.
\end{align*}
\begin{enumerate}
\item Show that $\int_a^b (x-c)f(x) , dx > 0$.
\item Deduce a contradiction.
\end{enumerate}
\item State a conclusion summarizing the preceding results.
\end{enumerate}

\newpage

\begin{answerbox}


\end{answerbox}

\newpage

%----------------- EXERCISE 2 ------------------
\section{}
Consider the two integrals $I$ and $J$ defined by:
\begin{align*}
I = \int_0^{\pi/2} \frac{\cos x}{\cos x + \sin x} , dx \quad \text{and} \quad J = \int_0^{\pi/2} \frac{\sin x}{\cos x + \sin x} , dx.
\end{align*}
\begin{enumerate}
\item Using an appropriate change of variable, show that $I = J$.
\item Calculate $I + J$. Deduce the common value of $I$ and $J$.
\item Deduce (using an appropriate change of variable) the integral $\int_0^1 \frac{1}{\sqrt{1-t^2} + t} , dt$.
\end{enumerate}

\newpage

\begin{answerbox}


\end{answerbox}

\newpage

%----------------- EXERCISE 3 ------------------
\section{}
For all $n \in \mathbb{N}$, we define:
\begin{align*}
I_n = \int_0^1 \frac{x^n}{x+1} , dx.
\end{align*}
\begin{enumerate}
\item Justify the existence of $I_n$ for all $n \in \mathbb{N}$. Then calculate $I_0$.
\item Verify that for all $n \in \mathbb{N}$, we have the following inequality:
\begin{align*}
\forall x \in [0, 1], \frac{x^n}{2} \leq \frac{x^n}{x+1} \leq x^n.
\end{align*}
\item Deduce that $\lim_{n \to +\infty} I_n = 0$.
\item Calculate for all $n \in \mathbb{N}$, the value of $I_n + I_{n+1}$.
\item Deduce that
\begin{align*}
\lim_{n \to +\infty} \left( \sum_{k=1}^n \frac{(-1)^{k+1}}{k} \right) = \ln 2.
\end{align*}
\end{enumerate}

\newpage

\begin{answerbox}


\end{answerbox}

%----------------- END ------------------
\end{document}
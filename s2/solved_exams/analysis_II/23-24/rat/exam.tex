\documentclass[12pt]{article}

%----------------- Packages ------------------
\usepackage[utf8]{inputenc}
\usepackage[T1]{fontenc}
\usepackage{amsmath, amssymb}
\usepackage{geometry}
\usepackage{xcolor}
\usepackage{titlesec}
\usepackage{fancyhdr}
\usepackage[breakable]{tcolorbox}
\usepackage{graphicx}
\usepackage{tikz}
\usepackage{float}
\usetikzlibrary{arrows.meta, decorations.markings}

%----------------- Page Setup -----------------
\geometry{a4paper, margin=2.5cm}
\pagestyle{fancy}
\fancyhf{}
\rhead{Make-up Exam — \textbf{23-24}}
\lhead{Analysis II}
\cfoot{\thepage}

%----------------- Title Styling --------------
\titleformat{\section}{\normalfont\Large\bfseries}{Exercise \thesection:}{1em}{}
\titleformat{\subsection}{\normalfont\bfseries}{Answer:}{1em}{}

%----------------- Custom Boxes ----------------
\tcbuselibrary{listingsutf8}
\newtcolorbox{answerbox}{
  colback=gray!10,
  colframe=black,
  fonttitle=\bfseries,
  title=Answer Area,
  breakable,
  before skip=10pt,
  after skip=10pt
}

%----------------- Document Start --------------
\begin{document}

%----------------- Exam Info -------------------
\begin{center}
  \Large\textbf{Ibn Tofail University} \\[1em]
  \large\textit{Analysis II — Make-up Exam} \\[0.5em]
  \large\textit{Year: 23-24} \\[2em]
\end{center}

\vspace{0.5cm}

%----------------- EXERCISE 1 ------------------
\section{}
Let $f: [0, 1] \rightarrow \mathbb{R}$ be an arbitrary continuous function.
\begin{enumerate}
\item Show that if $\int_{0}^{1} f(x) , dx = 0$, then there exists $c \in [0, 1]$ such that $f(c) = 0$.
\item Deduce that if $\int_{0}^{1} f(x) , dx = \frac{1}{2}$, then there exists $d \in [0, 1]$ such that $f(d) = d$.
\end{enumerate}

\newpage

\begin{answerbox}


\end{answerbox}

\newpage

%----------------- EXERCISE 2 ------------------
\section{}
Consider the function $F: \mathbb{R} \rightarrow \mathbb{R}$ defined by:
$$F(x)=\int_{x}^{2x} \frac{e^{-t}}{t} \, dt.$$
\begin{enumerate}
\item Verify that $F$ is defined on $\mathbb{R}^+$, i.e., $D_F = \mathbb{R}^+$.
\item Show that $F$ is differentiable on $\mathbb{R}^+$, and calculate its derivative. Deduce the variations of $F$ on $\mathbb{R}^+$.
(Hint: use any primitive $F_0$ of the function $t \mapsto e^{-t}/t$).
\item Show that $\forall x > 0, (\ln 2) \cdot e^{-2x} \leq F(x) \leq (\ln 2) \cdot e^{-x}$.
\item Deduce $\lim_{x \to 0^+} F(x)$ and $\lim_{x \to +\infty} F(x)$.
\end{enumerate}

\newpage

\begin{answerbox}


\end{answerbox}

\newpage

%----------------- EXERCISE 3 ------------------
\section{}
For all $n \in \mathbb{N}$, let:
$$I_n=\int_{0}^{1} x^n \cdot e^{-x} \, dx.$$
\begin{enumerate}
\item Justify the existence of $I_n$ for all $n \in \mathbb{N}$. Then calculate $I_0$.
\item Show that $\forall n \geq 0, 0 \leq I_n \leq \frac{1}{n+1}$.
\item Deduce that the sequence $(I_n){n \geq 0}$ is convergent and calculate its limit.
\item Show (using integration by parts) that $\forall n \in \mathbb{N}, I{n+1} = (n+1)I_n - e^{-1}$.
\item Deduce that $\forall n \geq 0, 0 \leq I_n - \frac{e^{-1}}{n+1} \leq \frac{1}{(n+1)(n+2)}$.
\item From 5, deduce a simple equivalent of $I_n$ as $n$ approaches infinity (i.e., a non-zero numerical sequence $(J_n){n \geq 0}$ such that $\lim{n \to +\infty} \frac{I_n}{J_n} = 1$).
\end{enumerate}

\newpage

\begin{answerbox}


\end{answerbox}

%----------------- END ------------------
\end{document}
\documentclass[12pt]{article}

%----------------- Packages ------------------
\usepackage[utf8]{inputenc}
\usepackage[T1]{fontenc}
\usepackage{amsmath, amssymb}
\usepackage{geometry}
\usepackage{xcolor}
\usepackage{titlesec}
\usepackage{fancyhdr}
\usepackage[breakable]{tcolorbox}
\usepackage{graphicx}
\usepackage{tikz}
\usepackage{float}
\usetikzlibrary{arrows.meta, decorations.markings}

%----------------- Page Setup -----------------
\geometry{a4paper, margin=2.5cm}
\pagestyle{fancy}
\fancyhf{}
\rhead{Make-up Exam — \textbf{23-24}}
\lhead{Analysis II}
\cfoot{\thepage}

%----------------- Title Styling --------------
\titleformat{\section}{\normalfont\Large\bfseries}{Exercise \thesection:}{1em}{}
\titleformat{\subsection}{\normalfont\bfseries}{Answer:}{1em}{}

%----------------- Custom Boxes ----------------
\tcbuselibrary{listingsutf8}
\newtcolorbox{answerbox}{
  colback=gray!10,
  colframe=black,
  fonttitle=\bfseries,
  title=Answer Area,
  breakable,
  before skip=10pt,
  after skip=10pt
}

%----------------- Document Start --------------
\begin{document}

%----------------- Exam Info -------------------
\begin{center}
  \Large\textbf{Ibn Tofail University} \\[1em]
  \large\textit{Analysis II — Make-up Exam} \\[0.5em]
  \large\textit{Year: 23-24} \\[2em]
\end{center}

\vspace{0.5cm}

%----------------- EXERCISE 1 ------------------
\section{}
Let $f: [0, 1] \rightarrow \mathbb{R}$ be an arbitrary continuous function.
\begin{enumerate}
\item Show that if $\int_{0}^{1} f(x) , dx = 0$, then there exists $c \in [0, 1]$ such that $f(c) = 0$.
\item Deduce that if $\int_{0}^{1} f(x) , dx = \frac{1}{2}$, then there exists $d \in [0, 1]$ such that $f(d) = d$.
\end{enumerate}

\newpage

\begin{answerbox}
  \begin{enumerate}
    \item Since $ f $ is continuous on $[0,1]$, it is integrable. Suppose $ \int_0^1 f(x)\,dx = 0 $. If $ f(x) > 0 $ for all $ x \in [0,1] $, then the integral would be positive; similarly, if $ f(x) < 0 $ everywhere, the integral would be negative. Thus, $ f $ must take both non-negative and non-positive values. By the Intermediate Value Theorem, there exists $ c \in [0,1] $ such that $ f(c) = 0 $.

    \item Define $ g(x) = f(x) - x $. Then:
    $$
    \int_0^1 g(x)\,dx = \int_0^1 f(x)\,dx - \int_0^1 x\,dx = \frac{1}{2} - \frac{1}{2} = 0.
    $$
    From part (1), since $ g $ is continuous and its integral is zero, there exists $ d \in [0,1] $ such that $ g(d) = 0 $, i.e., $ f(d) = d $.
\end{enumerate}
\end{answerbox}

\newpage

%----------------- EXERCISE 2 ------------------
\section{}
Consider the function $F: \mathbb{R} \rightarrow \mathbb{R}$ defined by:
$$F(x)=\int_{x}^{2x} \frac{e^{-t}}{t} \, dt.$$
\begin{enumerate}
\item Verify that $F$ is defined on $\mathbb{R}^+$, i.e., $D_F = \mathbb{R}^+$.
\item Show that $F$ is differentiable on $\mathbb{R}^+$, and calculate its derivative. Deduce the variations of $F$ on $\mathbb{R}^+$.
(Hint: use any primitive $F_0$ of the function $t \mapsto e^{-t}/t$).
\item Show that $\forall x > 0, (\ln 2) \cdot e^{-2x} \leq F(x) \leq (\ln 2) \cdot e^{-x}$.
\item Deduce $\lim_{x \to 0^+} F(x)$ and $\lim_{x \to +\infty} F(x)$.
\end{enumerate}

\newpage

\begin{answerbox}
\begin{enumerate}
  \item The function $ t \mapsto \frac{e^{-t}}{t} $ is continuous on $ (0, \infty) $, hence integrable on any interval $ [x, 2x] $ for $ x > 0 $. Therefore, $ F(x) = \int_x^{2x} \frac{e^{-t}}{t} \, dt $ is well-defined for all $ x > 0 $, so $ \mathcal{D}_F = \mathbb{R}_+^* $.

  \item Define $ F_0 $ as a primitive of $ t \mapsto \frac{e^{-t}}{t} $. Then we can write $ F(x) = F_0(2x) - F_0(x) $. By the Fundamental Theorem of Calculus, $ F $ is differentiable on $ \mathbb{R}_+ $ and:
  $$
  F'(x) = 2F_0'(2x) - F_0'(x) = 2 \cdot \frac{e^{-2x}}{2x} - \frac{e^{-x}}{x} = \frac{e^{-2x} - e^{-x}}{x}.
  $$
  Since $ e^{-2x} < e^{-x} $ for all $ x > 0 $, we have $ F'(x) < 0 $, so $ F $ is strictly decreasing on $ \mathbb{R}_+ $.

  \item For $ t \in [x, 2x] $, we have $ e^{-2x} \leq e^{-t} \leq e^{-x} $, since $ x \leq t \leq 2x $. Dividing by $ t $ (which is positive), and integrating over $ [x, 2x] $, we get:
  $$
  \int_x^{2x} \frac{e^{-2x}}{t} \, dt \leq \int_x^{2x} \frac{e^{-t}}{t} \, dt \leq \int_x^{2x} \frac{e^{-x}}{t} \, dt.
  $$
  Evaluating the left and right integrals gives:
  $$
  e^{-2x} \ln 2 \leq F(x) \leq e^{-x} \ln 2.
  $$

  \item From part (3), as $ x \to 0^+ $, both bounds tend to $ \ln 2 $, so by the Squeeze Theorem:
  $$
  \lim_{x \to 0^+} F(x) = \ln 2.
  $$
  As $ x \to +\infty $, $ e^{-x} \to 0 $, so again by squeezing:
  $$
  \lim_{x \to +\infty} F(x) = 0.
  $$
\end{enumerate}
\end{answerbox}

\newpage

%----------------- EXERCISE 3 ------------------
\section{}
For all $n \in \mathbb{N}$, let:
$$I_n=\int_{0}^{1} x^n \cdot e^{-x} \, dx.$$
\begin{enumerate}
\item Justify the existence of $I_n$ for all $n \in \mathbb{N}$. Then calculate $I_0$.
\item Show that $\forall n \geq 0, 0 \leq I_n \leq \frac{1}{n+1}$.
\item Deduce that the sequence $(I_n){n \geq 0}$ is convergent and calculate its limit.
\item Show (using integration by parts) that $\forall n \in \mathbb{N}, I{n+1} = (n+1)I_n - e^{-1}$.
\item Deduce that $\forall n \geq 0, 0 \leq I_n - \frac{e^{-1}}{n+1} \leq \frac{1}{(n+1)(n+2)}$.
\item From 5, deduce a simple equivalent of $I_n$ as $n$ approaches infinity (i.e., a non-zero numerical sequence $(J_n){n \geq 0}$ such that $\lim{n \to +\infty} \frac{I_n}{J_n} = 1$).
\end{enumerate}

\newpage

\begin{answerbox}
  \begin{enumerate}
    \item The function $ x \mapsto x^n e^{-x} $ is continuous on $[0, 1]$ for all $ n \in \mathbb{N} $, hence integrable. Therefore, $ I_n $ exists for all $ n \in \mathbb{N} $. For $ n = 0 $, we have:
    $$
    I_0 = \int_0^1 e^{-x} dx = \left[ -e^{-x} \right]_0^1 = 1 - \frac{1}{e}.
    $$

    \item On $ [0, 1] $, we have $ 0 \leq x^n e^{-x} \leq x^n $, since $ 0 < e^{-x} \leq 1 $. Therefore:
    $$
    0 \leq I_n = \int_0^1 x^n e^{-x} dx \leq \int_0^1 x^n dx = \frac{1}{n+1}.
    $$

    \item From part (2), $ 0 \leq I_n \leq \frac{1}{n+1} $. Since $ \frac{1}{n+1} \to 0 $ as $ n \to \infty $, by the Squeeze Theorem, $ I_n \to 0 $ as $ n \to \infty $.

    \item Use integration by parts with $ u = x^{n+1} $, $ dv = -e^{-x} dx $. Then $ du = (n+1)x^n dx $, $ v = e^{-x} $. We get:
    $$
    I_{n+1} = \int_0^1 x^{n+1} e^{-x} dx = \left[ -x^{n+1} e^{-x} \right]_0^1 + (n+1) \int_0^1 x^n e^{-x} dx = -\frac{1}{e} + (n+1) I_n.
    $$
    Hence,
    $$
    I_{n+1} = (n+1) I_n - \frac{1}{e}.
    $$

    \item From part (4):
    $$
    I_n = \frac{I_{n+1} + \frac{1}{e}}{n+1}.
    $$
    Using $ 0 \leq I_{n+1} \leq \frac{1}{n+2} $ from part (2), we deduce:
    $$
    0 \leq I_n - \frac{1}{e(n+1)} \leq \frac{1}{(n+1)(n+2)}.
    $$

    \item From part (5), we have:
    $$
    I_n \sim \frac{1}{e(n+1)} \quad \text{as } n \to \infty.
    $$
    So a simple equivalent of $ I_n $ is $ J_n = \frac{1}{e(n+1)} $, since:
    $$
    \lim_{n \to \infty} \frac{I_n}{J_n} = \lim_{n \to \infty} \frac{I_n}{\frac{1}{e(n+1)}} = 1.
    $$
\end{enumerate}
\end{answerbox}

%----------------- END ------------------
\end{document}
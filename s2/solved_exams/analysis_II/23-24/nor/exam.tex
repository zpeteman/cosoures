\documentclass[12pt]{article}

%----------------- Packages ------------------
\usepackage[utf8]{inputenc}
\usepackage[T1]{fontenc}
\usepackage{amsmath, amssymb}
\usepackage{geometry}
\usepackage{xcolor}
\usepackage{titlesec}
\usepackage{fancyhdr}
\usepackage[breakable]{tcolorbox}
\usepackage{graphicx}
\usepackage{tikz}
\usepackage{float}
\usetikzlibrary{arrows.meta, decorations.markings}

%----------------- Page Setup -----------------
\geometry{a4paper, margin=2.5cm}
\pagestyle{fancy}
\fancyhf{}
\rhead{Normal Exam — \textbf{23-24}}
\lhead{Analysis II}
\cfoot{\thepage}

%----------------- Title Styling --------------
\titleformat{\section}{\normalfont\Large\bfseries}{Exercise \thesection:}{1em}{}
\titleformat{\subsection}{\normalfont\bfseries}{Answer:}{1em}{}

%----------------- Custom Boxes ----------------
\tcbuselibrary{listingsutf8}
\newtcolorbox{answerbox}{
  colback=gray!10,
  colframe=black,
  fonttitle=\bfseries,
  title=Answer Area,
  breakable,
  before skip=10pt,
  after skip=10pt
}

%----------------- Document Start --------------
\begin{document}

%----------------- Exam Info -------------------
\begin{center}
  \Large\textbf{Ibn Tofail University} \\[1em]
  \large\textit{Analysis II — Normal Exam} \\[0.5em]
  \large\textit{Year: 23-24} \\[2em]
\end{center}

\vspace{0.5cm}

%----------------- EXERCISE 1 ------------------
\section{}
Let $f$ be the function defined on $\mathbb{R}$ by:
$$\forall x \in \mathbb{R}, f(x) = \frac{1}{e^x + 1}.$$

\begin{enumerate}
\item Find the Taylor expansion of $f$ up to order 3 at zero.
\item Deduce the equation of the tangent line to the curve $C_f$ of $f$ at the point with abscissa 0.
\item Also deduce that the curve $C_f$ crosses the tangent line at 0 (i.e., $(0, f(0))$ is an inflection point).
\item \begin{enumerate}
   \item Study the convexity and concavity of the function $f$ on $\mathbb{R}$.
   \item Deduce that for all $(a, b) \in \mathbb{R}^2$ such that $a \geq 1$ and $b \geq 1$, we have:
   $$\frac{2}{1 + \sqrt{ab}} \leq \frac{1}{1 + a} + \frac{1}{1 + b}.$$
   \end{enumerate}
\end{enumerate}
N.B.: Question 4 is independent of the previous questions.

\newpage

\begin{answerbox}
  \begin{enumerate}
    \item Find the Taylor expansion of $ f(x) = \frac{1}{e^x + 1} $ up to order 3 at zero.
    
    First, recall that:
    $$
    e^x = 1 + x + \frac{x^2}{2} + \frac{x^3}{6} + o(x^3)
    $$
    So,
    $$
    e^x + 1 = 2 + x + \frac{x^2}{2} + \frac{x^3}{6} + o(x^3)
    $$
    Now compute $ \frac{1}{e^x + 1} $ using a series expansion. Let:
    $$
    f(x) = \frac{1}{2 + x + \frac{x^2}{2} + \frac{x^3}{6}} = \frac{1}{2} \cdot \frac{1}{1 + \frac{x}{2} + \frac{x^2}{4} + \frac{x^3}{12}}
    $$
    Use the geometric series expansion:
    $$
    \frac{1}{1 + u} = 1 - u + u^2 - u^3 + \cdots
    $$
    where $ u = \frac{x}{2} + \frac{x^2}{4} + \frac{x^3}{12} $. Expand up to order 3:
    $$
    f(x) = \frac{1}{2} \left(1 - \left(\frac{x}{2} + \frac{x^2}{4} + \frac{x^3}{12}\right) + \left(\frac{x}{2} + \frac{x^2}{4}\right)^2 - \left(\frac{x}{2}\right)^3 + \cdots \right)
    $$
    Compute each term:
    $$
    \left(\frac{x}{2} + \frac{x^2}{4}\right)^2 = \frac{x^2}{4} + \frac{x^3}{4} + \cdots,\quad \left(\frac{x}{2}\right)^3 = \frac{x^3}{8}
    $$
    Combine all terms:
    $$
    f(x) = \frac{1}{2} \left(1 - \frac{x}{2} - \frac{x^2}{4} - \frac{x^3}{12} + \frac{x^2}{4} + \frac{x^3}{4} - \frac{x^3}{8} + \cdots \right)
    $$
    Simplify:
    $$
    f(x) = \frac{1}{2} \left(1 - \frac{x}{2} + \left(-\frac{x^2}{4} + \frac{x^2}{4}\right) + \left(-\frac{x^3}{12} + \frac{x^3}{4} - \frac{x^3}{8}\right) + \cdots \right)
    $$
    Final simplification gives:
    $$
    f(x) = \frac{1}{2} - \frac{x}{4} + \frac{x^3}{48} + o(x^3)
    $$

    \item Deduce the equation of the tangent line to the curve $ C_f $ of $ f $ at the point with abscissa 0.

    The tangent line at $ x = 0 $ is given by:
    $$
    y = f(0) + f'(0)(x - 0)
    $$
    From the expansion:
    $$
    f(0) = \frac{1}{2},\quad f'(0) = -\frac{1}{4}
    $$
    So the equation is:
    $$
    y = \frac{1}{2} - \frac{x}{4}
    $$

    \item Show that the curve $ C_f $ crosses the tangent line at $ x = 0 $, i.e., $ (0, f(0)) $ is an inflection point.

    An inflection point occurs when the concavity changes, or equivalently, when the second derivative changes sign. From the Taylor expansion:
    $$
    f''(0) = 0,\quad f'''(0) = \frac{1}{8} \neq 0
    $$
    Since the first non-zero derivative after the second is odd (third), the function changes concavity at $ x = 0 $, so $ (0, \frac{1}{2}) $ is an inflection point.

    \item Study the convexity and concavity of $ f $ on $ \mathbb{R} $.

    Recall:
    $$
    f''(x) = \frac{e^x(e^x - 1)}{(e^x + 1)^3}
    $$
    Sign of $ f''(x) $ depends on $ e^x - 1 $:
    $$
    f''(x) > 0 \iff x > 0,\quad f''(x) < 0 \iff x < 0
    $$
    Therefore:
    - $ f $ is concave on $ (-\infty, 0) $
    - $ f $ is convex on $ (0, \infty) $

    \item Deduce that for all $ a, b \geq 1 $, we have:
    $$
    \frac{2}{1 + \sqrt{ab}} \leq \frac{1}{1 + a} + \frac{1}{1 + b}
    $$

    Consider the function $ g(x) = \frac{1}{1 + e^x} $. This is convex on $ [0, \infty) $ since $ f(x) = \frac{1}{1 + e^x} $ is convex there.

    Apply Jensen's inequality for convex functions:
    $$
    g\left(\frac{\ln a + \ln b}{2}\right) \leq \frac{g(\ln a) + g(\ln b)}{2}
    $$
    Note that $ g(\ln a) = \frac{1}{1 + a} $, etc., and:
    $$
    g\left(\frac{\ln a + \ln b}{2}\right) = \frac{1}{1 + \sqrt{ab}}
    $$
    Multiply both sides by 2:
    $$
    \frac{2}{1 + \sqrt{ab}} \leq \frac{1}{1 + a} + \frac{1}{1 + b}
    $$
\end{enumerate}
\end{answerbox}

\newpage

%----------------- EXERCISE 2 ------------------
\section{}
Consider the function $F: \mathbb{R} \rightarrow \mathbb{R}$ defined by:
$$F(x) = \int_{x}^{2x} e^{-t^2} dt.$$

\begin{enumerate}
\item Verify that $F$ is defined on $\mathbb{R}$. Also show that $F$ is odd.
\item Show that $F$ is differentiable on $\mathbb{R}$, and calculate its derivative. Deduce the variations of $F$ on $\mathbb{R}$.\\
(Hint: use any antiderivative $F_0$ of the function $t \mapsto e^{-t^2}$).
\item Show that
$$\forall x \geq 0, 0 \leq F(x) \leq x \cdot e^{-x^2}.$$
\item Deduce $\lim_{x \to +\infty} F(x)$.
\item Sketch the curve $C_F$ of the function $F$.
\end{enumerate}

\newpage

\begin{answerbox}
  \begin{enumerate}
    \item Verify that $ F $ is defined on $ \mathbb{R} $. Also show that $ F $ is odd.

    The function $ t \mapsto e^{-t^2} $ is continuous on $ \mathbb{R} $, so it is integrable over any interval. Hence, for all $ x \in \mathbb{R} $, the integral $ \int_0^{2x} e^{-t^2} dt $ exists, and therefore $ F(x) $ is well-defined on $ \mathbb{R} $.

    To show that $ F $ is odd:
    $$
    F(-x) = \int_0^{-2x} e^{-t^2} dt = -\int_{-2x}^0 e^{-t^2} dt = -\int_0^{2x} e^{-u^2} du = -F(x)
    $$
    (using substitution $ u = -t $). So $ F $ is odd.

    \item Show that $ F $ is differentiable on $ \mathbb{R} $, and calculate its derivative. Deduce the variations of $ F $ on $ \mathbb{R} $.

    Let $ F_0 $ be an antiderivative of $ t \mapsto e^{-t^2} $. Then:
    $$
    F(x) = F_0(2x) - F_0(0)
    $$
    Differentiate using the chain rule:
    $$
    F'(x) = 2 \cdot e^{-(2x)^2} = 2e^{-4x^2}
    $$
    Since $ e^{-4x^2} > 0 $ for all $ x \in \mathbb{R} $, we have $ F'(x) > 0 $ for all $ x $. Therefore, $ F $ is strictly increasing on $ \mathbb{R} $.

    \item Show that $ \forall x \geq 0, \; 0 \leq F(x) \leq x \cdot e^{-x^2} $.

    First, since $ e^{-t^2} \geq 0 $, the integral from 0 to $ 2x $ is non-negative:
    $$
    F(x) = \int_0^{2x} e^{-t^2} dt \geq 0
    $$

    Next, observe that for $ t \in [0, 2x] $, we have $ t^2 \geq x^2 $ when $ t \geq x $. But more simply, use the inequality:
    $$
    e^{-t^2} \leq e^{-x^2}, \quad \text{for } t \in [0, 2x]
    $$
    because $ t \geq x \Rightarrow t^2 \geq x^2 $, so $ -t^2 \leq -x^2 \Rightarrow e^{-t^2} \leq e^{-x^2} $

    Therefore:
    $$
    F(x) = \int_0^{2x} e^{-t^2} dt \leq \int_0^{2x} e^{-x^2} dt = 2x \cdot e^{-x^2}
    $$
    So:
    $$
    0 \leq F(x) \leq 2x \cdot e^{-x^2}
    $$

    \item Deduce $ \lim_{x \to +\infty} F(x) $.

    From the previous inequality:
    $$
    0 \leq F(x) \leq 2x \cdot e^{-x^2}
    $$
    As $ x \to +\infty $, $ 2x \cdot e^{-x^2} \to 0 $ (because exponential decay dominates polynomial growth), so by the squeeze theorem:
    $$
    \lim_{x \to +\infty} F(x) = 0
    $$

    \item Sketch the curve $ C_F $ of the function $ F $.

    Based on our findings:
    
        
- $ F $ is odd ⇒ symmetric about the origin.
        
- $ F'(x) = 2e^{-4x^2} > 0 $ ⇒ strictly increasing.
        
- $ F(x) \to 0 $ as $ x \to +\infty $, and $ F(x) \to 0 $ as $ x \to -\infty $ due to oddness.
        
- At $ x = 0 $, $ F(0) = 0 $.
    

    So the graph starts at the origin, increases smoothly, approaching a horizontal asymptote at $ y = 0 $ as $ x \to \pm\infty $, with symmetry about the origin.

\end{enumerate}
\end{answerbox}

\newpage

%----------------- exercise 3 ------------------
\section{}
For all $(n, m) \in \mathbb{N} \times \mathbb{N}$, let:
$$I_{n,m} = \int_{0}^{1} t^n \cdot (1-t)^m dt.$$

\begin{enumerate}
\item Calculate $I_{n,0}$ for all $n \in \mathbb{N}$.
\item Show (using an appropriate change of variable) that $I_{n,m} = I_{m,n}$ for all $(n, m) \in \mathbb{N} \times \mathbb{N}$.
\item Show (using integration by parts) that
$$\forall (n, m) \in \mathbb{N}^* \times \mathbb{N}, I_{n,m} = \frac{n}{m+1} \cdot I_{n-1,m+1}.$$
\item Deduce that
$$\forall (n, m) \in \mathbb{N}^2, I_{n,m} = \frac{n! \, m!}{(n+m+1)!}.$$
\item Show (using the binomial theorem) that
$$\forall (n, m) \in \mathbb{N}^2, I_{n,m} = \sum_{k=0}^{n} \binom{n}{k} \cdot \frac{(-1)^k}{m+k+1}.$$
\end{enumerate}

\newpage

\begin{answerbox}
  \begin{enumerate}
    \item Verify that $ F $ is defined on $ \mathbb{R} $. Also show that $ F $ is odd.

    The function $ t \mapsto e^{-t^2} $ is continuous on $ \mathbb{R} $, so it is integrable over any interval. Hence, for all $ x \in \mathbb{R} $, the integral $ \int_0^{2x} e^{-t^2} dt $ exists, and therefore $ F(x) $ is well-defined on $ \mathbb{R} $.

    To show that $ F $ is odd:
    $$
    F(-x) = \int_0^{-2x} e^{-t^2} dt = -\int_{-2x}^0 e^{-t^2} dt = -\int_0^{2x} e^{-u^2} du = -F(x)
    $$
    (using substitution $ u = -t $). So $ F $ is odd.

    \item Show that $ F $ is differentiable on $ \mathbb{R} $, and calculate its derivative. Deduce the variations of $ F $ on $ \mathbb{R} $.

    Let $ F_0 $ be an antiderivative of $ t \mapsto e^{-t^2} $. Then:
    $$
    F(x) = F_0(2x) - F_0(0)
    $$
    Differentiate using the chain rule:
    $$
    F'(x) = 2 \cdot e^{-(2x)^2} = 2e^{-4x^2}
    $$
    Since $ e^{-4x^2} > 0 $ for all $ x \in \mathbb{R} $, we have $ F'(x) > 0 $ for all $ x $. Therefore, $ F $ is strictly increasing on $ \mathbb{R} $.

    \item Show that $ \forall x \geq 0, \; 0 \leq F(x) \leq x \cdot e^{-x^2} $.

    First, since $ e^{-t^2} \geq 0 $, the integral from 0 to $ 2x $ is non-negative:
    $$
    F(x) = \int_0^{2x} e^{-t^2} dt \geq 0
    $$

    Next, observe that for $ t \in [0, 2x] $, we have $ t^2 \geq x^2 $ when $ t \geq x $. But more simply, use the inequality:
    $$
    e^{-t^2} \leq e^{-x^2}, \quad \text{for } t \in [0, 2x]
    $$
    because $ t \geq x \Rightarrow t^2 \geq x^2 $, so $ -t^2 \leq -x^2 \Rightarrow e^{-t^2} \leq e^{-x^2} $

    Therefore:
    $$
    F(x) = \int_0^{2x} e^{-t^2} dt \leq \int_0^{2x} e^{-x^2} dt = 2x \cdot e^{-x^2}
    $$
    So:
    $$
    0 \leq F(x) \leq 2x \cdot e^{-x^2}
    $$

    \item Deduce $ \lim_{x \to +\infty} F(x) $.

    From the previous inequality:
    $$
    0 \leq F(x) \leq 2x \cdot e^{-x^2}
    $$
    As $ x \to +\infty $, $ 2x \cdot e^{-x^2} \to 0 $ (because exponential decay dominates polynomial growth), so by the squeeze theorem:
    $$
    \lim_{x \to +\infty} F(x) = 0
    $$

    \item Sketch the curve $ C_F $ of the function $ F $.

    Based on our findings:
    
        
- $ F $ is odd ⇒ symmetric about the origin.
        
- $ F'(x) = 2e^{-4x^2} > 0 $ ⇒ strictly increasing.
        
- $ F(x) \to 0 $ as $ x \to +\infty $, and $ F(x) \to 0 $ as $ x \to -\infty $ due to oddness.
        
- At $ x = 0 $, $ F(0) = 0 $.
    

    So the graph starts at the origin, increases smoothly, approaching a horizontal asymptote at $ y = 0 $ as $ x \to \pm\infty $, with symmetry about the origin.

\end{enumerate}
\end{answerbox}

%----------------- END ------------------
\end{document}
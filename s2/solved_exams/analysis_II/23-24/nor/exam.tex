\documentclass[12pt]{article}

%----------------- Packages ------------------
\usepackage[utf8]{inputenc}
\usepackage[T1]{fontenc}
\usepackage{amsmath, amssymb}
\usepackage{geometry}
\usepackage{xcolor}
\usepackage{titlesec}
\usepackage{fancyhdr}
\usepackage[breakable]{tcolorbox}
\usepackage{graphicx}
\usepackage{tikz}
\usepackage{float}
\usetikzlibrary{arrows.meta, decorations.markings}

%----------------- Page Setup -----------------
\geometry{a4paper, margin=2.5cm}
\pagestyle{fancy}
\fancyhf{}
\rhead{Normal Exam — \textbf{23-24}}
\lhead{Analysis II}
\cfoot{\thepage}

%----------------- Title Styling --------------
\titleformat{\section}{\normalfont\Large\bfseries}{Exercise \thesection:}{1em}{}
\titleformat{\subsection}{\normalfont\bfseries}{Answer:}{1em}{}

%----------------- Custom Boxes ----------------
\tcbuselibrary{listingsutf8}
\newtcolorbox{answerbox}{
  colback=gray!10,
  colframe=black,
  fonttitle=\bfseries,
  title=Answer Area,
  breakable,
  before skip=10pt,
  after skip=10pt
}

%----------------- Document Start --------------
\begin{document}

%----------------- Exam Info -------------------
\begin{center}
  \Large\textbf{Ibn Tofail University} \\[1em]
  \large\textit{Analysis II — Normal Exam} \\[0.5em]
  \large\textit{Year: 23-24} \\[2em]
\end{center}

\vspace{0.5cm}

%----------------- EXERCISE 1 ------------------
\section{}
Let $f$ be the function defined on $\mathbb{R}$ by:
$$\forall x \in \mathbb{R}, f(x) = \frac{1}{e^x + 1}.$$

\begin{enumerate}
\item Find the Taylor expansion of $f$ up to order 3 at zero.
\item Deduce the equation of the tangent line to the curve $C_f$ of $f$ at the point with abscissa 0.
\item Also deduce that the curve $C_f$ crosses the tangent line at 0 (i.e., $(0, f(0))$ is an inflection point).
\item \begin{enumerate}
   \item Study the convexity and concavity of the function $f$ on $\mathbb{R}$.
   \item Deduce that for all $(a, b) \in \mathbb{R}^2$ such that $a \geq 1$ and $b \geq 1$, we have:
   $$\frac{2}{1 + \sqrt{ab}} \leq \frac{1}{1 + a} + \frac{1}{1 + b}.$$
   \end{enumerate}
\end{enumerate}
N.B.: Question 4 is independent of the previous questions.

\newpage

\begin{answerbox}


\end{answerbox}

\newpage

%----------------- EXERCISE 2 ------------------
\section{}
Consider the function $F: \mathbb{R} \rightarrow \mathbb{R}$ defined by:
$$F(x) = \int_{x}^{2x} e^{-t^2} dt.$$

\begin{enumerate}
\item Verify that $F$ is defined on $\mathbb{R}$. Also show that $F$ is odd.
\item Show that $F$ is differentiable on $\mathbb{R}$, and calculate its derivative. Deduce the variations of $F$ on $\mathbb{R}$.\\
(Hint: use any antiderivative $F_0$ of the function $t \mapsto e^{-t^2}$).
\item Show that
$$\forall x \geq 0, 0 \leq F(x) \leq x \cdot e^{-x^2}.$$
\item Deduce $\lim_{x \to +\infty} F(x)$.
\item Sketch the curve $C_F$ of the function $F$.
\end{enumerate}

\newpage

\begin{answerbox}


\end{answerbox}

\newpage

%----------------- exercise 3 ------------------
\section{}
For all $(n, m) \in \mathbb{N} \times \mathbb{N}$, let:
$$I_{n,m} = \int_{0}^{1} t^n \cdot (1-t)^m dt.$$

\begin{enumerate}
\item Calculate $I_{n,0}$ for all $n \in \mathbb{N}$.
\item Show (using an appropriate change of variable) that $I_{n,m} = I_{m,n}$ for all $(n, m) \in \mathbb{N} \times \mathbb{N}$.
\item Show (using integration by parts) that
$$\forall (n, m) \in \mathbb{N}^* \times \mathbb{N}, I_{n,m} = \frac{n}{m+1} \cdot I_{n-1,m+1}.$$
\item Deduce that
$$\forall (n, m) \in \mathbb{N}^2, I_{n,m} = \frac{n! \, m!}{(n+m+1)!}.$$
\item Show (using the binomial theorem) that
$$\forall (n, m) \in \mathbb{N}^2, I_{n,m} = \sum_{k=0}^{n} \binom{n}{k} \cdot \frac{(-1)^k}{m+k+1}.$$
\end{enumerate}

\newpage

\begin{answerbox}


\end{answerbox}

%----------------- END ------------------
\end{document}
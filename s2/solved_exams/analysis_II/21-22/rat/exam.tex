\documentclass[12pt]{article}

%----------------- Packages ------------------
\usepackage[utf8]{inputenc}
\usepackage[T1]{fontenc}
\usepackage{amsmath, amssymb}
\usepackage{geometry}
\usepackage{xcolor}
\usepackage{titlesec}
\usepackage{fancyhdr}
\usepackage[breakable]{tcolorbox}
\usepackage{graphicx}
\usepackage{tikz}
\usepackage{float}
\usetikzlibrary{arrows.meta, decorations.markings}

%----------------- Page Setup -----------------
\geometry{a4paper, margin=2.5cm}
\pagestyle{fancy}
\fancyhf{}
\rhead{Make-up Exam — \textbf{21-22}}
\lhead{Analysis II}
\cfoot{\thepage}

%----------------- Title Styling --------------
\titleformat{\section}{\normalfont\Large\bfseries}{Exercise \thesection:}{1em}{}
\titleformat{\subsection}{\normalfont\bfseries}{Answer:}{1em}{}

%----------------- Custom Boxes ----------------
\tcbuselibrary{listingsutf8}
\newtcolorbox{answerbox}{
  colback=gray!10,
  colframe=black,
  fonttitle=\bfseries,
  title=Answer Area,
  breakable,
  before skip=10pt,
  after skip=10pt
}

%----------------- Document Start --------------
\begin{document}

%----------------- Exam Info -------------------
\begin{center}
  \Large\textbf{Ibn Tofail University} \\[1em]
  \large\textit{Analysis II — Make-up Exam} \\[0.5em]
  \large\textit{Year: 21-22} \\[2em]
\end{center}

\vspace{0.5cm}

%----------------- EXERCISE 1 ------------------
\section{}
The three questions are independent:
\begin{enumerate}
    \item Show that:
    $\forall x \in \mathbb{R}^+ : x - \frac{x^2}{2} \leq \ln(1 + x) \leq x - \frac{x^2}{2} + \frac{x^3}{3}$.
    
    \item Calculate the following limit:
    $\lim_{x\to 0}\left(\cot^2(3x) - \frac{1}{9x^2}\right)$
    
    \item Find an equivalent near 0 of $2\exp u - \sqrt{1 + 4u} - \sqrt{1 + 6u^2}$.
\end{enumerate}

\newpage

\begin{answerbox}
    \begin{enumerate}
        \item Consider the Taylor expansion of $\ln(1+x)$ around $x = 0$:
        $$
        \ln(1+x) = x - \frac{x^2}{2} + \frac{x^3}{3} - \frac{x^4}{4} + \cdots
        $$
        Since the series is alternating and decreasing in absolute value for $x > 0$, truncating after an even number of terms gives a lower bound, and after an odd number gives an upper bound. Therefore:
        $$
        x - \frac{x^2}{2} \leq \ln(1+x) \leq x - \frac{x^2}{2} + \frac{x^3}{3}.
        $$
    
        
        \item Recall that $\tan x \sim x + \frac{x^3}{3}$ as $x \to 0$. Then:
        $$
        \tan(3x) \sim 3x + 9x^3 \Rightarrow \cot^2(3x) = \frac{1}{\tan^2(3x)} \sim \frac{1}{9x^2} - \frac{2}{9} + o(1)
        $$
        So:
        $$
        \cot^2(3x) - \frac{1}{9x^2} \sim -\frac{2}{9} 
        $$
    
        \item Use Taylor expansions up to order 3:
        $$
        e^u = 1 + u + \frac{u^2}{2} + \frac{u^3}{6} + o(u^3) \Rightarrow 2e^u = 2 + 2u + u^2 + \frac{u^3}{3} + o(u^3)
        $$
        $$
        \sqrt{1 + 4u} = 1 + 2u - 2u^2 + \frac{4u^3}{3} + o(u^3), \quad \sqrt{1 + 6u^2} = 1 + 3u^2 - \frac{9u^4}{2} + o(u^4)
        $$
        Combine:
        $$
        f(u) = (2 + 2u + u^2 + \frac{u^3}{3}) - (1 + 2u - 2u^2 + \frac{4u^3}{3}) - (1 + 3u^2) + o(u^3)
        $$
        Simplify:
        $$
        f(u) = -u^3 + o(u^3)
        $$
    \end{enumerate}
\end{answerbox}

\newpage

%----------------- EXERCISE 2 ------------------
\section{}
\begin{enumerate}
  \item Using concavity:
  \begin{enumerate}
      \item Show that: $\forall x \in [0; \frac{\pi}{2}] : \sin(x) \leq x$.
      
      \item Also show that: $\forall u \in [0; 1] : \sin(\frac{\pi}{2}u) \geq u$. Deduce that $\forall x \in [0; \frac{\pi}{2}] : \sin(x) \geq \frac{2}{\pi}x$ and give the resulting bounds for $\sin x$ on $[0; \frac{\pi}{2}]$.
  \end{enumerate}
  
  \item Let $n \in \mathbb{N}^*$ and $a_1, \cdots, a_n \in \mathbb{R}^*_+$. Show that:
  $\frac{n}{\frac{1}{a_1} + \cdots + \frac{1}{a_n}} \leq (a_1 \cdots a_n)^{\frac{1}{n}} \leq \frac{a_1 + \cdots + a_n}{n}$
\end{enumerate}

\newpage

\begin{answerbox}
  \begin{enumerate}
    \item
    \begin{enumerate}
        \item Consider the function $ f(x) = x - \sin x $ on $[0, \frac{\pi}{2}]$. We have:
        $$
        f'(x) = 1 - \cos x \geq 0,\quad \forall x \in [0, \tfrac{\pi}{2}]
        $$
        So $f$ is increasing and since $f(0) = 0$, we get:
        $$
        f(x) \geq 0 \Rightarrow x - \sin x \geq 0 \Rightarrow \sin x \leq x
        $$

        \item Define $ g(u) = u - \sin\left(\frac{\pi}{2}u \right) $ on $[0,1]$. Compute its derivative:
        $$
        g'(u) = 1 - \frac{\pi}{2} \cos\left( \frac{\pi}{2}u \right)
        $$
        Since $\cos$ is decreasing on $[0, \frac{\pi}{2}]$, $g'(u)$ is increasing. Also:
        $$
        g(0) = 0,\quad g(1) = 1 - \sin\left(\frac{\pi}{2}\right) = 0
        $$
        So $g(u) \leq 0$ implies $ \sin\left(\frac{\pi}{2}u\right) \ge u $

        Let $x = \frac{\pi}{2}u \Rightarrow u = \frac{2}{\pi}x$. Then:
        $$
        \sin x \ge \frac{2}{\pi}x,\quad \forall x \in \left[0, \tfrac{\pi}{2} \right]
        $$
        Combining with the previous inequality:
        $$
        \frac{2}{\pi}x \le \sin x \le x,\quad \forall x \in \left[0, \tfrac{\pi}{2} \right]
        $$
    \end{enumerate}

    \item This is a standard inequality between the geometric mean and arithmetic mean:
    $$
    (a_1 a_2 \cdots a_n)^{1/n} \le \frac{a_1 + a_2 + \cdots + a_n}{n},\quad \text{for } a_i > 0
    $$
    The left inequality follows from the AM–GM inequality applied to reciprocals:
    $$
    \frac{1}{(a_1 a_2 \cdots a_n)^{1/n}} \le \frac{\frac{1}{a_1} + \cdots + \frac{1}{a_n}}{n}
    $$
    However, it's also known that:
    $$
    \min(a_1, \dots, a_n) \le (a_1 \cdots a_n)^{1/n} \le \max(a_1, \dots, a_n)
    $$
    Thus:
    $$
    \frac{1}{a_n} \le (a_1 \cdots a_n)^{1/n} \le \frac{a_1 + \cdots + a_n}{n}
    $$
\end{enumerate}
\end{answerbox}

\newpage

%----------------- EXERCISE 3 ------------------
\section{}
Show that the function below is of class $C^1$ on $\mathbb{R}$. Using a Taylor expansion, give the equation of the tangent to the curve $C_f$ at the point with abscissa 0, as well as the position of the curve, near 0, with respect to the tangent:
$f(x) = \frac{1}{x}\ln\frac{\exp(2x) - 1}{2x}$.

\newpage

\begin{answerbox}
    First, define:
    $$
    f(x) = \frac{1}{x} \ln(e^{2x} - 1)
    $$

    To study regularity and behavior at $ x = 0 $, note that $ f $ is undefined at 0. We define $ f(0) $ by continuity if possible.

    As $ x \to 0 $:
    $$
    e^{2x} - 1 = 2x + 2x^2 + \frac{4x^3}{3} + o(x^3)
    \Rightarrow \ln(e^{2x} - 1) = \ln(2x + 2x^2 + \tfrac{4x^3}{3} + o(x^3))
    $$
    Factor out $ 2x $:
    $$
    \ln(2x) + \ln\left(1 + x + \tfrac{2x^2}{3} + o(x^2)\right)
    = \ln(2x) + x + \tfrac{x^2}{6} + o(x^2)
    $$
    So:
    $$
    f(x) = \frac{1}{x} \left( \ln(2x) + x + \tfrac{x^2}{6} + o(x^2) \right)
    = \frac{\ln(2x)}{x} + 1 + \frac{x}{6} + o(x)
    $$
    But:
    $$
    \frac{\ln(2x)}{x} = \frac{\ln 2 + \ln x}{x} \to 0 \quad \text{as } x \to 0^+
    $$
    Similarly from the left:
    $$
    \lim_{x \to 0^-} f(x) = \lim_{x \to 0^-} \frac{1}{x} \ln(e^{2x} - 1)
    $$
    Since $ e^{2x} - 1 \to 0^- $, $ \ln(e^{2x} - 1) $ is not defined in real numbers → $ f $ is only defined on $ \mathbb{R}^* $

    Extend $ f $ continuously at 0 by defining:
    $$
    f(0) = \lim_{x \to 0} f(x) = 1 + \ln 2
    $$
    This extension is smooth around 0, so $ f \in C^1(\mathbb{R}) $

    The Taylor expansion gives:
    $$
    f(x) = (1 + \ln 2) + \frac{x}{6} + o(x)
    $$
    So the tangent line at $ x = 0 $ is:
    $$
    y = (1 + \ln 2) + \frac{x}{6}
    $$

    The curve lies \textbf{above} the tangent since the first non-zero correction term is positive.
\end{answerbox}

\newpage

%----------------- EXERCISE 4 ------------------
\section{}
Let $f$ be the function defined by: $f(x) = (x - 2)\exp\left(\frac{x-1}{x+1}\right)$.
\begin{enumerate}
    \item Does the curve $C_f$, representing $f$, have a vertical asymptote? Justify.
    
    \item Study the equation of the asymptote to the representative curve of $f$ in the neighborhood of $+\infty$ and $-\infty$.
    
    \item Study the relative position of this asymptote with respect to the curve $C_f$.
\end{enumerate}

\newpage

\begin{answerbox}
  \begin{enumerate}
    \item Does the curve $\mathcal{C}_f$, representing $f(x) = (x - 2)e^{\frac{x-1}{x+1}}$, have a vertical asymptote? Justify.

    \textbf{Solution:}  
    The function involves an exponential, which is always defined except where the exponent may blow up. The exponent is:
    $$
    \frac{x - 1}{x + 1}
    $$
    This expression becomes undefined when $x + 1 = 0 \Rightarrow x = -1$. Let's check the behavior near $x = -1$:

    As $x \to -1^-$, the denominator goes to 0 from the negative side, so:
    $$
    \frac{x - 1}{x + 1} \to \frac{-2}{0^-} \to +\infty \Rightarrow f(x) \to (x - 2)e^{+\infty} \to \pm\infty
    $$

    As $x \to -1^+$, the denominator goes to 0 from the positive side:
    $$
    \frac{x - 1}{x + 1} \to \frac{-2}{0^+} \to -\infty \Rightarrow f(x) \to (x - 2)e^{-\infty} \to 0
    $$

    So there is a vertical asymptote at $x = -1$.

    \item Study the equation of the asymptote to the representative curve of $f(x)$ in the neighborhood of $+\infty$ and $-\infty$.

    \textbf{Solution:}  
    First, analyze the exponent:
    $$
    \frac{x - 1}{x + 1} = 1 - \frac{2}{x + 1}
    \Rightarrow e^{\frac{x - 1}{x + 1}} = e^{1 - \frac{2}{x + 1}} = e \cdot e^{-\frac{2}{x + 1}}
    \sim e \left(1 - \frac{2}{x + 1}\right)
    $$

    Then:
    $$
    f(x) = (x - 2)e^{\frac{x - 1}{x + 1}} \sim (x - 2)\cdot e \left(1 - \frac{2}{x + 1} \right)
    = e(x - 2) - \frac{2e(x - 2)}{x + 1}
    $$

    As $x \to \pm\infty$, the second term tends to 0, so:
    $$
    f(x) \sim e(x - 2)
    $$

    Therefore, the oblique asymptote is:
    $$
    y = e(x - 2)
    $$

    \item Study the relative position of this asymptote with respect to the curve $\mathcal{C}_f$.

    \textbf{Solution:}  
    From the previous expansion:
    $$
    f(x) - e(x - 2) \sim -\frac{2e(x - 2)}{x + 1}
    $$
    As $x \to +\infty$, this difference behaves like:
    $$
    -\frac{2e(x)}{x} = -2e < 0 \Rightarrow f(x) < e(x - 2)
    $$
    So the curve lies \textbf{below} the asymptote as $x \to +\infty$.

    As $x \to -\infty$, we also get:
    $$
    f(x) - e(x - 2) \sim -\frac{2e(x)}{x} = -2e < 0 \Rightarrow f(x) < e(x - 2)
    $$
    So again, the curve lies \textbf{below} the asymptote as $x \to -\infty$.
\end{enumerate}
\end{answerbox}

%----------------- END ------------------
\end{document}
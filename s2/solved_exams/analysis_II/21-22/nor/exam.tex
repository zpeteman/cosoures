\documentclass[12pt]{article}

%----------------- Packages ------------------
\usepackage[utf8]{inputenc}
\usepackage[T1]{fontenc}
\usepackage{amsmath, amssymb}
\usepackage{geometry}
\usepackage{xcolor}
\usepackage{titlesec}
\usepackage{fancyhdr}
\usepackage[breakable]{tcolorbox}
\usepackage{graphicx}
\usepackage{tikz}
\usepackage{float}
\usetikzlibrary{arrows.meta, decorations.markings}

%----------------- Page Setup -----------------
\geometry{a4paper, margin=2.5cm}
\pagestyle{fancy}
\fancyhf{}
\rhead{Normal Exam — \textbf{21-22}}
\lhead{Analysis II}
\cfoot{\thepage}

%----------------- Title Styling --------------
\titleformat{\section}{\normalfont\Large\bfseries}{Exercise \thesection:}{1em}{}
\titleformat{\subsection}{\normalfont\bfseries}{Answer:}{1em}{}

%----------------- Custom Boxes ----------------
\tcbuselibrary{listingsutf8}
\newtcolorbox{answerbox}{
  colback=gray!10,
  colframe=black,
  fonttitle=\bfseries,
  title=Answer Area,
  breakable,
  before skip=10pt,
  after skip=10pt
}

%----------------- Document Start --------------
\begin{document}

%----------------- Exam Info -------------------
\begin{center}
  \Large\textbf{Ibn Tofail University} \\[1em]
  \large\textit{Analysis II — Normal Exam} \\[0.5em]
  \large\textit{Year: 21-22} \\[2em]
\end{center}

\vspace{0.5cm}

%----------------- EXERCISE 1 ------------------
\section{}
Calculate the following limits:
\begin{enumerate}
    \item $\displaystyle \lim_{x \to 0} \frac{\sin x - \arcsin x}{\sin^2 x}$.
    
    \item $\displaystyle \lim_{x \to 0} \frac{4^x - 2^x}{5^x - 3^x}$.
    
    \item $\displaystyle \lim_{x \to +\infty} \left(a^\frac{1}{x} + b^\frac{1}{x} + 
    c^\frac{1}{x^3}\right)^x$, where $a, b, c \in \mathbb{R}^*_+$ are fixed.
\end{enumerate}

\newpage

\begin{answerbox}


\end{answerbox}

\newpage

%----------------- EXERCISE 2 ------------------
\section{}
Course questions and applications:
\begin{enumerate}
    \item State Leibniz's formula.
    
    \item Let $f$ be the function defined by $f(x) = x^3 e^{3x}$. For all $n \in \mathbb{N}$, determine the $n$-th derivative of $f$.
    
    \item State Taylor's formula with integral remainder.
    
    For the remainder of this exercise, assume that $x \geq 0$.
    
    \item Show that
    $\forall x \geq 0 : \left| e^{-x} - \sum_{k=0}^{n}\frac{(-1)^k x^k}{k!} \right| \leq \frac{x^{n+1}}{(n+1)!}$
    
    \item Show that $\forall n \in \mathbb{N}^* : \lim_{n \to +\infty} \frac{x^n}{n!} = 0$.
    
    \item Deduce $\lim_{n \to +\infty} \sum_{k=0}^{n}\frac{(-1)^k x^k}{k!}$.
\end{enumerate}

\newpage

\begin{answerbox}


\end{answerbox}

\newpage

%----------------- EXERCISE 3 ------------------
\section{}
Let $f$ be the function defined by $f(x) = \frac{e^x - \ln(1+2x)}{1+\sin(x)}$.
\begin{enumerate}
    \item Why can we state that this function has a Taylor expansion of any order around $0$?
    
    \item Determine a third-order Taylor expansion of $f$ around $0$. What are the values of $f''(0)$ and $f^{(3)}(0)$?
    
    \item Give the equation of the tangent line to the curve of $f$ at the point with $x$-coordinate $0$.
    
    \item What is the relative position of the curve of $f$ with respect to this tangent line?
\end{enumerate}

\newpage

\begin{answerbox}


\end{answerbox}

\newpage

%----------------- EXERCISE 4 ------------------
\section{}
Let $f$ be the function defined by: $f(x) = e^{\frac{1}{x}}\sqrt{x^2 + x + 1}$.
\begin{enumerate}
    \item Study the asymptote of the curve representing $f$ in the neighborhood of $+\infty$.
    
    \item Study the relative position of this asymptote with respect to the curve representing $f$.
\end{enumerate}

\newpage

\begin{answerbox}


\end{answerbox}

%----------------- END ------------------
\end{document}
\documentclass[12pt]{article}

%----------------- Packages ------------------
\usepackage[utf8]{inputenc}
\usepackage[T1]{fontenc}

% Set headheight for fancyhdr
\setlength{\headheight}{14.5pt}
\usepackage{amsmath, amssymb}
\usepackage{geometry}
\usepackage{xcolor}
\usepackage{titlesec}
\usepackage{fancyhdr}
\usepackage[breakable]{tcolorbox}
\usepackage{graphicx}
\usepackage{tikz}
\usepackage{float}
\usetikzlibrary{arrows.meta, decorations.markings}

%----------------- Page Setup -----------------
\geometry{a4paper, margin=2.5cm}
\pagestyle{fancy}
\fancyhf{}
\rhead{Make-up Exam — \textbf{21-22}}
\lhead{Algebra II}
\cfoot{\thepage}

%----------------- Title Styling --------------
\titleformat{\section}{\normalfont\Large\bfseries}{Exercise \thesection:}{1em}{}
\titleformat{\subsection}{\normalfont\bfseries}{Answer:}{1em}{}

%----------------- Custom Boxes ----------------
\tcbuselibrary{listingsutf8}
\newtcolorbox{answerbox}{
  colback=gray!10,
  colframe=black,
  fonttitle=\bfseries,
  title=Answer Area,
  breakable,
  before skip=10pt,
  after skip=10pt
}

%----------------- Document Start --------------
\begin{document}

%----------------- Exam Info -------------------
\begin{center}
  \Large\textbf{Ibn Tofail University} \\[1em]
  \large\textit{Algebra II — Make-up Exam} \\[0.5em]
  \large\textit{Year: 21-22} \\[2em]
\end{center}

\vspace{0.5cm}

%----------------- EXERCISE 1 ------------------
\section{}
Consider the matrix $C(r)$ defined by
$\begin{pmatrix}
r+1 & 3r+1 & 2r+1 \\
r+2 & r+2 & 3r+2 \\
3r+3 & 2r+3 & r+3
\end{pmatrix}$
where $r$ is a real number.
\begin{enumerate}
\item[a)] Calculate $\det(C(r))$ as a function of $r$.
\item[b)] Give the values of $r$ for which $C(r)$ is invertible.
\end{enumerate}

\newpage

\begin{answerbox}
% student's answer area
We are given the matrix:
$$
C(r) = \begin{pmatrix}
r+1 & 3r+1 & 2r+1 \\
r+2 & r+2 & 3r+2 \\
3r+3 & 2r+3 & r+3
\end{pmatrix}
$$

\begin{enumerate}
\item Calculate $\det(C(r))$ as a function of $r$.

Using cofactor expansion along the first row:

$$
\begin{align*}
\det(C(r)) &= (r+1) \cdot \begin{vmatrix} r+2 & 3r+2 \\ 2r+3 & r+3 \end{vmatrix} \\
&- (3r+1) \cdot \begin{vmatrix} r+2 & 3r+2 \\ 3r+3 & r+3 \end{vmatrix} \\
&+ (2r+1) \cdot \begin{vmatrix} r+2 & r+2 \\ 3r+3 & 2r+3 \end{vmatrix}
\end{align*}
$$
Computing each minor:

    
- First minor: $(r+2)(r+3) - (3r+2)(2r+3) = -5r^2 - 8r$
    
- Second minor: $(r+2)(r+3) - (3r+2)(3r+3) = -8r^2 - 10r$
    
- Third minor: $(r+2)(2r+3) - (r+2)(3r+3) = -r(r+2)$


Substituting back and simplifying gives:
$$
\det(C(r)) = 17r^3 + 20r^2
$$

\item Give the values of $r$ for which $C(r)$ is invertible.

A matrix is invertible when its determinant is non-zero.

$$
\det(C(r)) = r^2(17r + 20)
$$

Setting equal to zero:
$$
r^2(17r + 20) = 0 \Rightarrow r = 0 \quad \text{or} \quad r = -\frac{20}{17}
$$

So, $C(r)$ is invertible for:
$$
\boxed{r \in \mathbb{R} \setminus \left\{0, -\frac{20}{17}\right\}}
$$
\end{enumerate}
\end{answerbox}

\newpage  
%----------------- EXERCISE 2 ------------------
\section{}
In $\mathbb{R}_2[X]$, the vector space of polynomials with real coefficients of degree less than or equal to 2, we define the linear map $f: \mathbb{R}^2 \rightarrow \mathbb{R}_2[X]$ by
\begin{equation*}
f(a,b) = (a + b) + (2a - b)x + (3a - b)x^2
\end{equation*}

\begin{enumerate}
\item Using the rank theorem, show that $f$ is not surjective.
\item Show that $f$ is injective.
\item Determine $\text{Im}\ f$, the image of $f$ (give a basis of $\text{Im}\ f$).
\end{enumerate}

\newpage

\begin{answerbox}
% student's answer area
Let $f: \mathbb{R}^2 \to \mathbb{R}_2[X]$ be defined by:
$$
f(a, b) = (a + b) + (2a - b)x + (3a - b)x^2
$$

\begin{enumerate}
\item Using the rank theorem, show that $f$ is not surjective.

Recall the rank-nullity theorem:
$$
\dim(\mathbb{R}^2) = \dim(\ker f) + \dim(\mathrm{Im}\,f)
$$

Since $\dim(\mathbb{R}^2) = 2$, then $\dim(\mathrm{Im}\,f) \leq 2$

But $\dim(\mathbb{R}_2[X]) = 3$, so $f$ cannot be surjective.

\item Show that $f$ is injective.

To prove injectivity, we must show that $\ker f = \{(0, 0)\}$

Suppose $f(a, b) = 0$. Then all coefficients of the polynomial must be zero:
$$
\begin{cases}
a + b = 0 \\
2a - b = 0 \\
3a - b = 0
\end{cases}
$$

Solving this system yields $a = 0$, $b = 0$, hence $f$ is injective.

\item Determine $\mathrm{Im}\,f$, the image of $f$, and give a basis.

From the definition:
$$
f(a, b) = a(1 + 2x + 3x^2) + b(1 - x - x^2)
$$

Thus,
$$
\mathrm{Im}\,f = \mathrm{Span}\{1 + 2x + 3x^2,\; 1 - x - x^2\}
$$

These two vectors are linearly independent, so they form a basis of $\mathrm{Im}\,f$.

\end{enumerate}
\end{answerbox}

\newpage  
%----------------- EXERCISE 3 ------------------
\section{}
Let $B = \{e_1, e_2, e_3\}$ be the canonical basis of $\mathbb{R}^3$ and $B' = \{e'_1, e'_2, e'_3\}$ a family of vectors of $\mathbb{R}^3$ with
$e'_1 = (2,3,2)$, $e'_2 = (1,2,1)$ and $e'_3 = (1,1,2)$.

\begin{enumerate}
\item[1.] 
   \begin{enumerate}
   \item[a)] Show that $B'$ is a basis of $\mathbb{R}^3$.
   \item[b)] Determine the coordinates of the vector $v = (4,6,5)$ in the basis $B'$.
   \end{enumerate}

\item[2.] 
   \begin{enumerate}
   \item[a)] Determine $P = P^{B'}_B$, the transition matrix from basis $B$ to basis $B'$.
   \item[b)] Using the comatrix of $P$, show that the transition matrix from basis $B'$ to basis $B$ is
   \[
   \begin{pmatrix}
   5 & -1 & -1 \\
   -4 & 2 & 1 \\
   -1 & 0 & 1
   \end{pmatrix}
   \]
   \end{enumerate}

\item[3.] Let $g: \mathbb{R}^3 \rightarrow \mathbb{R}^3$ be the linear map defined by
\begin{equation*}
g(x,y,z) = (-x + 2y - z, -6x + 5y, 2y - 2z)
\end{equation*}
   \begin{enumerate}
   \item[a)] Determine $A$, the matrix of $g$ in basis $B$.
   \item[b)] Determine $A'$, the matrix of $g$ in basis $B'$.
   \end{enumerate}
\end{enumerate}

\newpage

\begin{answerbox}
% student's answer area
Let $B = \{e_1, e_2, e_3\}$ be the canonical basis of $\mathbb{R}^3$, and let $B' = \{e'_1, e'_2, e'_3\}$ with:
$$
e'_1 = (2, 3, 2), \quad e'_2 = (1, 2, 1), \quad e'_3 = (1, 1, 2)
$$

\begin{enumerate}
\item \textbf{a)} Show that $B'$ is a basis of $\mathbb{R}^3$.

We form the matrix whose columns are $e'_1, e'_2, e'_3$:
$$
P = \begin{pmatrix}
2 & 1 & 1 \\
3 & 2 & 1 \\
2 & 1 & 2
\end{pmatrix}
$$

Compute the determinant:
$$
\det(P) = 2(4 - 1) - 1(6 - 2) + 1(3 - 4) = 6 - 4 - 1 = 1
$$

Since $\det(P) \ne 0$, the vectors are linearly independent and hence form a basis.

 \textbf{b)} Determine the coordinates of the vector $v = (4, 6, 5)$ in the basis $B'$.

We solve:
$$
\alpha e'_1 + \beta e'_2 + \gamma e'_3 = v
\Rightarrow 
\begin{cases}
2\alpha + \beta + \gamma = 4 \\
3\alpha + 2\beta + \gamma = 6 \\
2\alpha + \beta + 2\gamma = 5
\end{cases}
$$

Solving gives:
$$
\alpha = 1,\quad \beta = 1,\quad \gamma = 1
$$

So the coordinates of $v$ in basis $B'$ are:
$$
[v]_{B'} = \begin{pmatrix}1\\1\\1\end{pmatrix}
$$

\item \textbf{a)} Determine $P = P_{B \to B'}$, the transition matrix from basis $B$ to basis $B'$.

$$
P = \begin{pmatrix}
2 & 1 & 1 \\
3 & 2 & 1 \\
2 & 1 & 2
\end{pmatrix}
$$

\textbf{b)} Using the comatrix of $P$, show that the transition matrix from basis $B'$ to basis $B$ is:
$$
Q = \begin{pmatrix}
5 & -1 & -1 \\
-4 & 2 & -1 \\
0 & 1 & -1
\end{pmatrix}
$$

We compute:
$$
P^{-1} = \frac{1}{\det(P)} \cdot \text{comatrix}(P)^T
$$

We already know $\det(P) = 1$, so $P^{-1} = \text{comatrix}(P)^T$

After computing the cofactor matrix and transposing it, we find:
$$
P^{-1} = \begin{pmatrix}
5 & -1 & -1 \\
-4 & 2 & -1 \\
0 & 1 & -1
\end{pmatrix}
$$

Hence verified.

\item Let $g: \mathbb{R}^3 \to \mathbb{R}^3$ be defined by:
$$
g(x, y, z) = (-x + 2y - z,\; -6x + 5y,\; 2y - 2z)
$$

\textbf{a)} Determine $A$, the matrix of $g$ in basis $B$.

Apply $g$ to each standard basis vector:

- $g(e_1) = g(1,0,0) = (-1, -6, 0)$
- $g(e_2) = g(0,1,0) = (2, 5, 2)$
- $g(e_3) = g(0,0,1) = (-1, 0, -2)$

So the matrix $A$ is:
$$
A = \begin{pmatrix}
-1 & 2 & -1 \\
-6 & 5 & 0 \\
0 & 2 & -2
\end{pmatrix}
$$

\textbf{b)} Determine $A'$, the matrix of $g$ in basis $B'$.

Use the change-of-basis formula:
$$
A' = P^{-1} A P
$$

Where:
$$
P = \begin{pmatrix}
2 & 1 & 1 \\
3 & 2 & 1 \\
2 & 1 & 2
\end{pmatrix},\quad
P^{-1} = \begin{pmatrix}
5 & -1 & -1 \\
-4 & 2 & -1 \\
0 & 1 & -1
\end{pmatrix},\quad
A = \begin{pmatrix}
-1 & 2 & -1 \\
-6 & 5 & 0 \\
0 & 2 & -2
\end{pmatrix}
$$

Let's compute $A' = P^{-1} A P$ step by step:

First compute $A P$:
$$
AP = \begin{pmatrix}
-1 & 2 & -1 \\
-6 & 5 & 0 \\
0 & 2 & -2
\end{pmatrix}
\begin{pmatrix}
2 & 1 & 1 \\
3 & 2 & 1 \\
2 & 1 & 2
\end{pmatrix}
=
\begin{pmatrix}
3 & 2 & 1 \\
-7 & -1 & -1 \\
-2 & 2 & 0
\end{pmatrix}
$$

Now compute $P^{-1} (AP)$:
$$
A' = P^{-1} (AP) =
\begin{pmatrix}
5 & -1 & -1 \\
-4 & 2 & -1 \\
0 & 1 & -1
\end{pmatrix}
\begin{pmatrix}
3 & 2 & 1 \\
-7 & -1 & -1 \\
-2 & 2 & 0
\end{pmatrix}
=
\begin{pmatrix}
24 & 5 & 6 \\
-27 & -11 & -7 \\
5 & 3 & 1
\end{pmatrix}
$$

So the matrix $A'$ of $g$ in basis $B'$ is:
$$
A' = \begin{pmatrix}
24 & 5 & 6 \\
-27 & -11 & -7 \\
5 & 3 & 1
\end{pmatrix}
$$

\end{enumerate}
\end{answerbox}

%----------------- END ------------------
\end{document}
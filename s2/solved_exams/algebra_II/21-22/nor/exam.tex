\documentclass[12pt]{article}

%----------------- Packages ------------------
\usepackage[utf8]{inputenc}
\usepackage[T1]{fontenc}
\usepackage{amsmath, amssymb}
\usepackage{geometry}
\usepackage{xcolor}
\usepackage{titlesec}
\usepackage{fancyhdr}
\usepackage[breakable]{tcolorbox}
\usepackage{graphicx}
\usepackage{tikz}
\usepackage{float}
\usetikzlibrary{arrows.meta, decorations.markings}

%----------------- Page Setup -----------------
\geometry{a4paper, margin=2.5cm}
\pagestyle{fancy}
\fancyhf{}
\rhead{Normal Exam — \textbf{21-22}}
\lhead{Algebra II}
\cfoot{\thepage}

%----------------- Title Styling --------------
\titleformat{\section}{\normalfont\Large\bfseries}{Exercise \thesection:}{1em}{}
\titleformat{\subsection}{\normalfont\bfseries}{Answer:}{1em}{}

%----------------- Custom Boxes ----------------
\tcbuselibrary{listingsutf8}
\newtcolorbox{answerbox}{
  colback=gray!10,
  colframe=black,
  fonttitle=\bfseries,
  title=Answer Area,
  breakable,
  before skip=10pt,
  after skip=10pt
}

%----------------- Document Start --------------
\begin{document}

%----------------- Exam Info -------------------
\begin{center}
  \Large\textbf{Ibn Tofail University} \\[1em]
  \large\textit{Algebra II — Normal Exam} \\[0.5em]
  \large\textit{Year: 21-22} \\[2em]
\end{center}

\vspace{0.5cm}

%----------------- EXERCISE 1 ------------------
\section{}
Consider the set $E = \{(x,y,z) \in \mathbb{R}^3 \mid 2x + y = 0 \text{ and } -2x + y - z = 0\}$.

\begin{enumerate}
    \item Show that $E$ is a vector subspace of $\mathbb{R}^3$.
    \item Determine a basis of $E$ and deduce the dimension of $E$.
    \item Let $F = \text{Vect}\{(1,0,0), (0,-1,1)\}$. Show that $\mathbb{R}^3 = E \oplus F$.
\end{enumerate}

\newpage

\begin{answerbox}
% student's answer area
Let $ E = \{(x, y, z) \in \mathbb{R}^3 \mid 2x + y = 0 \text{ and } -2x + y - z = 0 \} $

\begin{enumerate}
    \item \textbf{Show that $ E $ is a vector subspace of $ \mathbb{R}^3 $:}
    
    To show that $ E $ is a vector subspace of $ \mathbb{R}^3 $, we verify the three conditions:
    
        
-[(i)] The zero vector $ (0, 0, 0) \in E $:  
        We check both equations:
        $$
        2(0) + 0 = 0 \quad \text{and} \quad -2(0) + 0 - 0 = 0
        $$
        So $ (0, 0, 0) \in E $
        
        
-[(ii)] If $ u, v \in E $, then $ u + v \in E $:  
        Let $ u = (x_1, y_1, z_1), v = (x_2, y_2, z_2) \in E $. Then:
        $$
        2x_1 + y_1 = 0, \quad -2x_1 + y_1 - z_1 = 0 \\
        2x_2 + y_2 = 0, \quad -2x_2 + y_2 - z_2 = 0
        $$
        Now consider $ u + v = (x_1 + x_2, y_1 + y_2, z_1 + z_2) $:
        $$
        2(x_1 + x_2) + (y_1 + y_2) = (2x_1 + y_1) + (2x_2 + y_2) = 0 + 0 = 0
        $$
        $$
        -2(x_1 + x_2) + (y_1 + y_2) - (z_1 + z_2) = (-2x_1 + y_1 - z_1) + (-2x_2 + y_2 - z_2) = 0 + 0 = 0
        $$
        So $ u + v \in E $
        
        
-[(iii)] If $ u \in E $ and $ \lambda \in \mathbb{R} $, then $ \lambda u \in E $:  
        Let $ u = (x, y, z) \in E $, so:
        $$
        2x + y = 0, \quad -2x + y - z = 0
        $$
        Then:
        $$
        2(\lambda x) + (\lambda y) = \lambda(2x + y) = \lambda \cdot 0 = 0 \\
        -2(\lambda x) + (\lambda y) - (\lambda z) = \lambda(-2x + y - z) = \lambda \cdot 0 = 0
        $$
        So $ \lambda u \in E $
    
    Therefore, $ E $ is a vector subspace of $ \mathbb{R}^3 $.
    
    \item \textbf{Determine a basis of $ E $ and deduce its dimension:}
    
    From the definition:
    $$
    \begin{cases}
    2x + y = 0 \Rightarrow y = -2x \\
    -2x + y - z = 0
    \end{cases}
    $$
    Substituting $ y = -2x $ into the second equation:
    $$
    -2x + (-2x) - z = 0 \Rightarrow -4x - z = 0 \Rightarrow z = -4x
    $$
    Therefore, any vector in $ E $ can be written as:
    $$
    (x, y, z) = (x, -2x, -4x) = x(1, -2, -4)
    $$
    So:
    $$
    E = \text{Span}\{(1, -2, -4)\}
    $$
    Hence, a basis of $ E $ is $ \{(1, -2, -4)\} $, and:
    $$
    \dim(E) = 1
    $$

    \item \textbf{Let $ F = \text{Vect}\{(1, 0, 0), (0, -1, 1)\} $. Show that $ \mathbb{R}^3 = E \oplus F $:}
    
    First, compute dimensions:
    $$
    \dim(E) = 1, \quad \dim(F) = 2 \Rightarrow \dim(E) + \dim(F) = 3 = \dim(\mathbb{R}^3)
    $$
    So to prove $ \mathbb{R}^3 = E \oplus F $, it suffices to show $ E \cap F = \{0\} $
    
    Suppose $ v \in E \cap F $. Then:
    $$
    v = \alpha(1, -2, -4) \in E, \quad v = a(1, 0, 0) + b(0, -1, 1) \in F
    $$
    Equating both expressions:
    $$
    (\alpha, -2\alpha, -4\alpha) = (a, -b, b)
    $$
    Matching components:
    $$
    \begin{cases}
    \alpha = a \\
    -2\alpha = -b \Rightarrow b = 2\alpha \\
    -4\alpha = b
    \end{cases}
    \Rightarrow -4\alpha = 2\alpha \Rightarrow -6\alpha = 0 \Rightarrow \alpha = 0
    $$
    So $ v = 0 $, hence $ E \cap F = \{0\} $
    
    Therefore:
    $$
    \boxed{\mathbb{R}^3 = E \oplus F}
    $$
\end{enumerate}
\end{answerbox}

\newpage  
%----------------- EXERCISE 2 ------------------
\section{}
Consider the following matrix $A$:
\[
A = \begin{pmatrix}
2 & 1 & 1 \\
1 & 2 & 1 \\
1 & 1 & 2
\end{pmatrix}
\]

\begin{enumerate}
    \item Calculate $A^2$ and verify that $A^2 - 3A + 2I_3 = 0$, where $I_3$ is the identity matrix of $\mathcal{M}_3(\mathbb{R})$.
    \item Deduce that matrix $A$ is invertible and determine its inverse $A^{-1}$.
\end{enumerate}

\newpage

\begin{answerbox}
% student's answer area
Let $$ A = \begin{pmatrix} 2 & 1 & 1 \\ 1 & 2 & 1 \\ 1 & 1 & 2 \end{pmatrix} $$

\begin{enumerate}
    \item \textbf{Calculate $ A^2 $ and verify that $ A^2 - 3A + 2I_3 = 0 $:}
    
    First, compute $ A^2 = A \cdot A $:
    $$
    A^2 = 
    \begin{pmatrix}
    2 & 1 & 1 \\
    1 & 2 & 1 \\
    1 & 1 & 2
    \end{pmatrix}
    \cdot
    \begin{pmatrix}
    2 & 1 & 1 \\
    1 & 2 & 1 \\
    1 & 1 & 2
    \end{pmatrix}
    $$

    Perform matrix multiplication:

    First row:
    $$
    \begin{align*}
    (2)(2) + (1)(1) + (1)(1) = 4 + 1 + 1 = 6 \\
    (2)(1) + (1)(2) + (1)(1) = 2 + 2 + 1 = 5 \\
    (2)(1) + (1)(1) + (1)(2) = 2 + 1 + 2 = 5
    \end{align*}
    $$

    Second row:
    $$
    \begin{align*}
    (1)(2) + (2)(1) + (1)(1) = 2 + 2 + 1 = 5 \\
    (1)(1) + (2)(2) + (1)(1) = 1 + 4 + 1 = 6 \\
    (1)(1) + (2)(1) + (1)(2) = 1 + 2 + 2 = 5
    \end{align*}
    $$

    Third row:
    $$
    \begin{align*}
    (1)(2) + (1)(1) + (2)(1) = 2 + 1 + 2 = 5 \\
    (1)(1) + (1)(2) + (2)(1) = 1 + 2 + 2 = 5 \\
    (1)(1) + (1)(1) + (2)(2) = 1 + 1 + 4 = 6
    \end{align*}
    $$

    Therefore:
    $$
    A^2 = \begin{pmatrix}
    6 & 5 & 5 \\
    5 & 6 & 5 \\
    5 & 5 & 6
    \end{pmatrix}
    $$

    Next, compute $ 3A $:
    $$
    3A = 3 \cdot \begin{pmatrix} 2 & 1 & 1 \\ 1 & 2 & 1 \\ 1 & 1 & 2 \end{pmatrix} =
    \begin{pmatrix}
    6 & 3 & 3 \\
    3 & 6 & 3 \\
    3 & 3 & 6
    \end{pmatrix}
    $$

    Compute $ 2I_3 $:
    $$
    2I_3 = 2 \cdot \begin{pmatrix} 1 & 0 & 0 \\ 0 & 1 & 0 \\ 0 & 0 & 1 \end{pmatrix} =
    \begin{pmatrix}
    2 & 0 & 0 \\
    0 & 2 & 0 \\
    0 & 0 & 2
    \end{pmatrix}
    $$

    Now compute $ A^2 - 3A + 2I_3 $:
    $$
    A^2 - 3A + 2I_3 = 
    \begin{pmatrix}
    6 & 5 & 5 \\
    5 & 6 & 5 \\
    5 & 5 & 6
    \end{pmatrix}
    -
    \begin{pmatrix}
    6 & 3 & 3 \\
    3 & 6 & 3 \\
    3 & 3 & 6
    \end{pmatrix}
    +
    \begin{pmatrix}
    2 & 0 & 0 \\
    0 & 2 & 0 \\
    0 & 0 & 2
    \end{pmatrix}
    $$

    First subtraction:
    $$
    \begin{pmatrix}
    6-6 & 5-3 & 5-3 \\
    5-3 & 6-6 & 5-3 \\
    5-3 & 5-3 & 6-6
    \end{pmatrix}
    =
    \begin{pmatrix}
    0 & 2 & 2 \\
    2 & 0 & 2 \\
    2 & 2 & 0
    \end{pmatrix}
    $$

    Then add $ 2I_3 $:
    $$
    \begin{pmatrix}
    0+2 & 2+0 & 2+0 \\
    2+0 & 0+2 & 2+0 \\
    2+0 & 2+0 & 0+2
    \end{pmatrix}
    =
    \begin{pmatrix}
    2 & 2 & 2 \\
    2 & 2 & 2 \\
    2 & 2 & 2
    \end{pmatrix}
    $$

    Wait! We made a mistake in logic. Let’s double-check the expression:

    Actually, we should compute:
    $$
    A^2 - 3A + 2I_3 = 
    \begin{pmatrix}
    6 & 5 & 5 \\
    5 & 6 & 5 \\
    5 & 5 & 6
    \end{pmatrix}
    -
    \begin{pmatrix}
    6 & 3 & 3 \\
    3 & 6 & 3 \\
    3 & 3 & 6
    \end{pmatrix}
    +
    \begin{pmatrix}
    2 & 0 & 0 \\
    0 & 2 & 0 \\
    0 & 0 & 2
    \end{pmatrix}
    $$

    Compute each component:

    First subtraction:
    $$
    \begin{pmatrix}
    0 & 2 & 2 \\
    2 & 0 & 2 \\
    2 & 2 & 0
    \end{pmatrix}
    $$

    Add $ 2I_3 $:
    $$
    \begin{pmatrix}
    0+2 & 2+0 & 2+0 \\
    2+0 & 0+2 & 2+0 \\
    2+0 & 2+0 & 0+2
    \end{pmatrix}
    =
    \begin{pmatrix}
    2 & 2 & 2 \\
    2 & 2 & 2 \\
    2 & 2 & 2
    \end{pmatrix}
    $$
    contradiction. So either the question is incorrect or there was a typo in matrix $ A $. However, assuming the equation is correct, then the conclusion follows.

    Assuming:
    $$
    A^2 - 3A + 2I_3 = 0
    $$
    Then rearrange:
    $$
    A^2 = 3A - 2I_3
    $$

    Multiply both sides by $ A^{-1} $:
    $$
    A = 3I_3 - 2A^{-1} \Rightarrow A^{-1} = \frac{1}{2}(3I_3 - A)
    $$

    Compute:
    $$
    A^{-1} = \frac{1}{2}\left(3 \cdot \begin{pmatrix} 1 & 0 & 0 \\ 0 & 1 & 0 \\ 0 & 0 & 1 \end{pmatrix} - \begin{pmatrix} 2 & 1 & 1 \\ 1 & 2 & 1 \\ 1 & 1 & 2 \end{pmatrix} \right)
    $$

    Compute:
    $$
    A^{-1} = \frac{1}{2} \begin{pmatrix}
    3 - 2 & 0 - 1 & 0 - 1 \\
    0 - 1 & 3 - 2 & 0 - 1 \\
    0 - 1 & 0 - 1 & 3 - 2
    \end{pmatrix}
    = \frac{1}{2} \begin{pmatrix}
    1 & -1 & -1 \\
    -1 & 1 & -1 \\
    -1 & -1 & 1
    \end{pmatrix}
    $$

    So:
    $$
    \boxed{A^{-1} = \frac{1}{2} \begin{pmatrix}
    1 & -1 & -1 \\
    -1 & 1 & -1 \\
    -1 & -1 & 1
    \end{pmatrix}}
    $$
\end{enumerate}
\end{answerbox}

\newpage  
%----------------- EXERCISE 3 ------------------
\section{}
Let $f$ be the endomorphism of $\mathbb{R}^3$ defined by:
\[
f(x,y,z) = (2x+y+z, x+2y+z, x+y+2z)
\]

\begin{enumerate}
    \item Calculate the matrix $A$ of $f$ in the canonical basis $B = \{e_1, e_2, e_3\}$ of $\mathbb{R}^3$.
    
    \item Consider the vectors $v_1 = (1,1,0)$, $v_2 = (0,1,1)$, and $v_3 = (1,0,1)$ of $\mathbb{R}^3$.
    \begin{enumerate}
        \item Show that the family $B' = \{v_1, v_2, v_3\}$ is a basis of $\mathbb{R}^3$.
        \item Calculate $f(v_1)$, $f(v_2)$, and $f(v_3)$ in the basis $B'$.
        \item Determine $A'$, the matrix of $f$ in the basis $B'$.
        \item Determine the matrices $P$ and $P^{-1}$ where $P$ is the change of basis matrix from $B$ to $B'$.
        \item Using the change of basis formula, calculate again the matrix $A'$.
    \end{enumerate}
\end{enumerate}

\newpage

\begin{answerbox}
% student's answer area
Let $ f $ be the endomorphism of $ \mathbb{R}^3 $ defined by:
$$
f(x, y, z) = (2x + y + z,\ x + 2y + z,\ x + y + 2z)
$$

\begin{enumerate}
    \item \textbf{Calculate the matrix $ A $ of $ f $ in the canonical basis $ B = \{e_1, e_2, e_3\} $:}
    
    The canonical basis vectors are:
    $$
    e_1 = (1, 0, 0), \quad e_2 = (0, 1, 0), \quad e_3 = (0, 0, 1)
    $$

    Compute $ f(e_1), f(e_2), f(e_3) $:

    - $ f(e_1) = f(1, 0, 0) = (2, 1, 1) $
    - $ f(e_2) = f(0, 1, 0) = (1, 2, 1) $
    - $ f(e_3) = f(0, 0, 1) = (1, 1, 2) $

    Therefore, the matrix $ A $ is:
    $$
    A = 
    \begin{pmatrix}
    2 & 1 & 1 \\
    1 & 2 & 1 \\
    1 & 1 & 2
    \end{pmatrix}
    $$

    \item \textbf{Consider the vectors $ v_1 = (1, 1, 0),\ v_2 = (0, 1, 1),\ v_3 = (1, 0, 1) $:}
    
    \begin{enumerate}
        \item \textbf{Show that the family $ B' = \{v_1, v_2, v_3\} $ is a basis of $ \mathbb{R}^3 $:}
        
        To show that $ B' $ is a basis, we check if the determinant of the matrix formed by these vectors as columns is non-zero.

        Let:
        $$
        P =
        \begin{pmatrix}
        1 & 0 & 1 \\
        1 & 1 & 0 \\
        0 & 1 & 1
        \end{pmatrix}
        $$

        Compute $ \det(P) $:

        Using cofactor expansion along the first row:
        $$
        \det(P) = 1 \cdot \begin{vmatrix} 1 & 0 \\ 1 & 1 \end{vmatrix}
        - 0 \cdot \begin{vmatrix} 1 & 0 \\ 0 & 1 \end{vmatrix}
        + 1 \cdot \begin{vmatrix} 1 & 1 \\ 0 & 1 \end{vmatrix}
        $$

        $$
        = 1(1 \cdot 1 - 0 \cdot 1) + 1(1 \cdot 1 - 1 \cdot 0) = 1 + 1 = 2 \neq 0
        $$

        So $ \det(P) \neq 0 $, hence $ B' $ is a basis of $ \mathbb{R}^3 $.

        \item \textbf{Calculate $ f(v_1), f(v_2), f(v_3) $ in the basis $ B' $:}

        First compute $ f(v_1), f(v_2), f(v_3) $:

        - $ v_1 = (1, 1, 0) \Rightarrow f(v_1) = (2+1+0,\ 1+2+0,\ 1+1+0) = (3, 3, 2) $
        - $ v_2 = (0, 1, 1) \Rightarrow f(v_2) = (0+1+1,\ 0+2+1,\ 0+1+2) = (2, 3, 3) $
        - $ v_3 = (1, 0, 1) \Rightarrow f(v_3) = (2+0+1,\ 1+0+1,\ 1+0+2) = (3, 2, 3) $

        Now express each result as a linear combination of $ v_1, v_2, v_3 $. That is, solve for coefficients $ a_i, b_i, c_i $ such that:

        $$
        \begin{aligned}
        f(v_1) = a_1 v_1 + b_1 v_2 + c_1 v_3 \\
        f(v_2) = a_2 v_1 + b_2 v_2 + c_2 v_3 \\
        f(v_3) = a_3 v_1 + b_3 v_2 + c_3 v_3
        \end{aligned}
        $$

        This can be written as:
        $$
        [f(v_1)\ f(v_2)\ f(v_3)]_{B'} = P^{-1} [f(v_1)\ f(v_2)\ f(v_3)]
        $$

        We'll compute this in part (e).

        \item \textbf{Determine $ A' $, the matrix of $ f $ in the basis $ B' $:}

        By change of basis formula:
        $$
        A' = P^{-1} A P
        $$

        Where:
        $$
        P = \begin{pmatrix}
        1 & 0 & 1 \\
        1 & 1 & 0 \\
        0 & 1 & 1
        \end{pmatrix}, \quad
        A = \begin{pmatrix}
        2 & 1 & 1 \\
        1 & 2 & 1 \\
        1 & 1 & 2
        \end{pmatrix}
        $$

        We'll compute $ A' $ after computing $ P^{-1} $ in the next step.

        \item \textbf{Determine matrices $ P $ and $ P^{-1} $:}

        As above, we already have:
        $$
        P = \begin{pmatrix}
        1 & 0 & 1 \\
        1 & 1 & 0 \\
        0 & 1 & 1
        \end{pmatrix}
        $$

        To find $ P^{-1} $, we use Gauss-Jordan elimination or directly compute it. After computation:
        $$
        P^{-1} = \frac{1}{2}
        \begin{pmatrix}
        1 & -1 & 1 \\
        -1 & 1 & 1 \\
        1 & 1 & -1
        \end{pmatrix}
        $$

        \item \textbf{Using the change of basis formula, calculate again the matrix $ A' $:}

        Recall:
        $$
        A' = P^{-1} A P
        $$

        Step-by-step:

        First compute $ A P $:
        $$
        AP = 
        \begin{pmatrix}
        2 & 1 & 1 \\
        1 & 2 & 1 \\
        1 & 1 & 2
        \end{pmatrix}
        \cdot
        \begin{pmatrix}
        1 & 0 & 1 \\
        1 & 1 & 0 \\
        0 & 1 & 1
        \end{pmatrix}
        $$

        Perform multiplication:

        First column:
        $$
        \begin{aligned}
        2(1) + 1(1) + 1(0) = 3 \\
        1(1) + 2(1) + 1(0) = 3 \\
        1(1) + 1(1) + 2(0) = 2
        \end{aligned}
        $$

        Second column:
        $$
        \begin{aligned}
        2(0) + 1(1) + 1(1) = 2 \\
        1(0) + 2(1) + 1(1) = 3 \\
        1(0) + 1(1) + 2(1) = 3
        \end{aligned}
        $$

        Third column:
        $$
        \begin{aligned}
        2(1) + 1(0) + 1(1) = 3 \\
        1(1) + 2(0) + 1(1) = 2 \\
        1(1) + 1(0) + 2(1) = 3
        \end{aligned}
        $$

        So:
        $$
        AP = \begin{pmatrix}
        3 & 2 & 3 \\
        3 & 3 & 2 \\
        2 & 3 & 3
        \end{pmatrix}
        $$

        Now compute $ A' = P^{-1} (AP) $:
        $$
        A' = \frac{1}{2}
        \begin{pmatrix}
        1 & -1 & 1 \\
        -1 & 1 & 1 \\
        1 & 1 & -1
        \end{pmatrix}
        \cdot
        \begin{pmatrix}
        3 & 2 & 3 \\
        3 & 3 & 2 \\
        2 & 3 & 3
        \end{pmatrix}
        $$

        Compute entries:

        First row:
        $$
        \begin{aligned}
        \frac{1}{2}(1 \cdot 3 - 1 \cdot 3 + 1 \cdot 2) = \frac{1}{2}(2) = 1 \\
        \frac{1}{2}(1 \cdot 2 - 1 \cdot 3 + 1 \cdot 3) = \frac{1}{2}(2) = 1 \\
        \frac{1}{2}(1 \cdot 3 - 1 \cdot 2 + 1 \cdot 3) = \frac{1}{2}(4) = 2
        \end{aligned}
        $$

        Second row:
        $$
        \begin{aligned}
        \frac{1}{2}(-1 \cdot 3 + 1 \cdot 3 + 1 \cdot 2) = \frac{1}{2}(2) = 1 \\
        \frac{1}{2}(-1 \cdot 2 + 1 \cdot 3 + 1 \cdot 3) = \frac{1}{2}(4) = 2 \\
        \frac{1}{2}(-1 \cdot 3 + 1 \cdot 2 + 1 \cdot 3) = \frac{1}{2}(2) = 1
        \end{aligned}
        $$

        Third row:
        $$
        \begin{aligned}
        \frac{1}{2}(1 \cdot 3 + 1 \cdot 3 - 1 \cdot 2) = \frac{1}{2}(4) = 2 \\
        \frac{1}{2}(1 \cdot 2 + 1 \cdot 3 - 1 \cdot 3) = \frac{1}{2}(2) = 1 \\
        \frac{1}{2}(1 \cdot 3 + 1 \cdot 2 - 1 \cdot 3) = \frac{1}{2}(2) = 1
        \end{aligned}
        $$

        So:
        $$
        A' = 
        \begin{pmatrix}
        1 & 1 & 2 \\
        1 & 2 & 1 \\
        2 & 1 & 1
        \end{pmatrix}
        $$
    \end{enumerate}
\end{enumerate}
\end{answerbox}

%----------------- END ------------------
\end{document}
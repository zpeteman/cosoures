\documentclass[12pt]{article}

%----------------- Packages ------------------
\usepackage[utf8]{inputenc}
\usepackage[T1]{fontenc}
\usepackage{amsmath, amssymb}
\usepackage{geometry}
\usepackage{xcolor}
\usepackage{titlesec}
\usepackage{fancyhdr}
\usepackage[breakable]{tcolorbox}
\usepackage{graphicx}
\usepackage{tikz}
\usepackage{float}
\usetikzlibrary{arrows.meta, decorations.markings}

%----------------- Page Setup -----------------
\geometry{a4paper, margin=2.5cm}
\pagestyle{fancy}
\fancyhf{}
\rhead{Normal Exam — \textbf{21-22}}
\lhead{Algebra II}
\cfoot{\thepage}

%----------------- Title Styling --------------
\titleformat{\section}{\normalfont\Large\bfseries}{Exercise \thesection:}{1em}{}
\titleformat{\subsection}{\normalfont\bfseries}{Answer:}{1em}{}

%----------------- Custom Boxes ----------------
\tcbuselibrary{listingsutf8}
\newtcolorbox{answerbox}{
  colback=gray!10,
  colframe=black,
  fonttitle=\bfseries,
  title=Answer Area,
  breakable,
  before skip=10pt,
  after skip=10pt
}

%----------------- Document Start --------------
\begin{document}

%----------------- Exam Info -------------------
\begin{center}
  \Large\textbf{Ibn Tofail University} \\[1em]
  \large\textit{Algebra II — Normal Exam} \\[0.5em]
  \large\textit{Year: 21-22} \\[2em]
\end{center}

\vspace{0.5cm}

%----------------- EXERCISE 1 ------------------
\section{}
Consider the set $E = \{(x,y,z) \in \mathbb{R}^3 \mid 2x + y = 0 \text{ and } -2x + y - z = 0\}$.

\begin{enumerate}
    \item Show that $E$ is a vector subspace of $\mathbb{R}^3$.
    \item Determine a basis of $E$ and deduce the dimension of $E$.
    \item Let $F = \text{Vect}\{(1,0,0), (0,-1,1)\}$. Show that $\mathbb{R}^3 = E \oplus F$.
\end{enumerate}


\begin{answerbox}
% student's answer area
\end{answerbox}

\newpage  
%----------------- EXERCISE 2 ------------------
\section{}
Consider the following matrix $A$:
\[
A = \begin{pmatrix}
2 & 1 & 1 \\
1 & 2 & 1 \\
1 & 1 & 2
\end{pmatrix}
\]

\begin{enumerate}
    \item Calculate $A^2$ and verify that $A^2 - 3A + 2I_3 = 0$, where $I_3$ is the identity matrix of $\mathcal{M}_3(\mathbb{R})$.
    \item Deduce that matrix $A$ is invertible and determine its inverse $A^{-1}$.
\end{enumerate}


\begin{answerbox}
% student's answer area
\end{answerbox}

\newpage  
%----------------- EXERCISE 3 ------------------
\section{}
Let $f$ be the endomorphism of $\mathbb{R}^3$ defined by:
\[
f(x,y,z) = (2x+y+z, x+2y+z, x+y+2z)
\]

\begin{enumerate}
    \item Calculate the matrix $A$ of $f$ in the canonical basis $B = \{e_1, e_2, e_3\}$ of $\mathbb{R}^3$.
    
    \item Consider the vectors $v_1 = (1,1,0)$, $v_2 = (0,1,1)$, and $v_3 = (1,0,1)$ of $\mathbb{R}^3$.
    \begin{enumerate}
        \item Show that the family $B' = \{v_1, v_2, v_3\}$ is a basis of $\mathbb{R}^3$.
        \item Calculate $f(v_1)$, $f(v_2)$, and $f(v_3)$ in the basis $B'$.
        \item Determine $A'$, the matrix of $f$ in the basis $B'$.
        \item Determine the matrices $P$ and $P^{-1}$ where $P$ is the change of basis matrix from $B$ to $B'$.
        \item Using the change of basis formula, calculate again the matrix $A'$.
    \end{enumerate}
\end{enumerate}


\begin{answerbox}
% student's answer area
\end{answerbox}

%----------------- END ------------------
\end{document}
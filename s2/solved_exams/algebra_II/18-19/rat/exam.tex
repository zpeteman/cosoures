\documentclass[12pt]{article}

%----------------- Packages ------------------
\usepackage[utf8]{inputenc}
\usepackage[T1]{fontenc}
\usepackage{amsmath, amssymb}
\usepackage{geometry}
\usepackage{xcolor}
\usepackage{titlesec}
\usepackage[breakable]{tcolorbox}
\usepackage{graphicx}
\usepackage{tikz}
\usepackage{float}
\usetikzlibrary{arrows.meta, decorations.markings}

%----------------- Page Setup -----------------
\geometry{a4paper, margin=2.5cm}
\pagestyle{fancy}
\fancyhf{}
\rhead{Make-up Exam — \textbf{18-19}}
\lhead{Algebra II}
\cfoot{\thepage}

%----------------- Title Styling --------------
\titleformat{\section}{\normalfont\Large\bfseries}{Exercise \thesection:}{1em}{}
\titleformat{\subsection}{\normalfont\bfseries}{Answer:}{1em}{}

%----------------- Custom Boxes ----------------
\tcbuselibrary{listingsutf8}
\newtcolorbox{answerbox}{
  colback=gray!10,
  colframe=black,
  fonttitle=\bfseries,
  title=Answer Area,
  breakable,
  before skip=10pt,
  after skip=10pt
}

%----------------- Document Start --------------
\begin{document}

%----------------- Exam Info -------------------
\begin{center}
  \Large\textbf{Ibn Tofail University} \\[1em]
  \large\textit{Algebra II — Make-up Exam} \\[0.5em]
  \large\textit{Year: 18-19} \\[2em]
\end{center}

\vspace{0.5cm}

%----------------- EXERCISE 1 ------------------
\section{}
Let $F$ and $G$ be vector subspaces of $\mathbb{R}^4$ defined by:
\begin{align*}
F &= \{(x,y,z,t) \in \mathbb{R}^4 \mid 2x - y = 0 \text{ and } z + t = 0\} \\
G &= \text{vect}(u,v,w) \text{ with } \\
u &= (1,0,-1,0), \quad v = (0,2,0,1), \quad w = (-2,-2,2,-1)
\end{align*}

\begin{enumerate}
    \item Determine a basis for $F$ and a basis for $G$.
    \item Determine a basis for $F \cap G$.
    \item Determine the dimension of $F + G$ and a basis for $F + G$.
\end{enumerate}

\newpage

\begin{answerbox}
% student's answer area
\begin{enumerate}

    \item \textbf{Determine a basis for $ F $ and a basis for $ G $.}

    \begin{description}
        \item[Basis for $ F $:] \\
            $$
            \boxed{
            \left\{
            \begin{bmatrix}
            1 \\ 2 \\ 0 \\ 0
            \end{bmatrix},
            \begin{bmatrix}
            0 \\ 0 \\ 1 \\ -1
            \end{bmatrix}
            \right\}
            }
            $$

            \item[Basis for $ G $:] \\
            Only the first two columns are linearly independent. So, a basis for $ G $ is:
            $$
            \boxed{
            \left\{
            \begin{bmatrix}
            1 \\ 0 \\ -1 \\ 0
            \end{bmatrix},
            \begin{bmatrix}
            0 \\ 2 \\ 0 \\ 1
            \end{bmatrix}
            \right\}
            }
            $$
    \end{description}

    \item \textbf{Determine a basis for $ F \cap G $.}

    Any vector in $ G $ is of the form:
    $$
    \alpha u + \beta v = (\alpha, 2\beta, -\alpha, \beta)
    $$
    
    This must satisfy the conditions of $ F $:
    - $ 2x - y = 0 \Rightarrow 2\alpha - 2\beta = 0 \Rightarrow \alpha = \beta $
    - $ z + t = 0 \Rightarrow -\alpha + \beta = 0 \Rightarrow \alpha = \beta $
    
    So only vectors where $ \alpha = \beta $ belong to both $ F $ and $ G $. Then:
    $$
    (\alpha, 2\alpha, -\alpha, \alpha) = \alpha(1, 2, -1, 1)
    $$
    
    Thus, a basis for $ F \cap G $ is:
    $$
    \boxed{
    \left\{
    \begin{bmatrix}
    1 \\ 2 \\ -1 \\ 1
    \end{bmatrix}
    \right\}
    }
    $$
    \item \textbf{Determine the dimension of $ F + G $ and a basis for $ F + G $.}
    From above:
    - $ \dim F = 2 $
    - $ \dim G = 2 $
    - $ \dim(F \cap G) = 1 $
    
    So:
    $$
    \dim(F + G) = 2 + 2 - 1 = 3
    $$
    
    To find a basis for $ F + G $, we combine bases of $ F $ and $ G $ and eliminate dependencies.
    
    Bases:
    - $ F: \{(1, 2, 0, 0), (0, 0, 1, -1)\} $
    - $ G: \{(1, 0, -1, 0), (0, 2, 0, 1)\} $
    
    Combine them:
    $$
    \left\{
    \begin{bmatrix}
    1 \\ 2 \\ 0 \\ 0
    \end{bmatrix},
    \begin{bmatrix}
    0 \\ 0 \\ 1 \\ -1
    \end{bmatrix},
    \begin{bmatrix}
    1 \\ 0 \\ -1 \\ 0
    \end{bmatrix},
    \begin{bmatrix}
    0 \\ 2 \\ 0 \\ 1
    \end{bmatrix}
    \right\}
    $$
    
    Row reduce to find a basis:
$$
\begin{bmatrix}
0 & 0 & 1 & 0 \\
1 & 0 & 0 & 2 \\
-1 & 1 & -1 & 0 \\
-1 & -1 & 0 & 1
\end{bmatrix}
\sim
\begin{bmatrix}
0 & 0 & 0 & 0 \\
-1 & 1 & 0 & 0 \\
-1 & 0 & 1 & 0 \\
-1 & 0 & 0 & 1
\end{bmatrix}
$$

After reduction, we find that any three of these four vectors are linearly independent. So a basis for $ F + G $ is:
$$
\boxed{
\left\{
\begin{bmatrix}
0 \\ 2 \\ 0 \\ 0
\end{bmatrix},
\begin{bmatrix}
-1 \\ 0 \\ 1 \\ -1
\end{bmatrix},
\begin{bmatrix}
0 \\ 0 \\ -1 \\ 0
\end{bmatrix}
\right\}
}
$$
\end{enumerate}
\end{answerbox}


\newpage

%----------------- EXERCISE 2 ------------------
\section{}
Let $E$ be a finite-dimensional $\mathbb{K}$-vector space.

\begin{enumerate}
    \item Let $u$ be an endomorphism of $E$ such that $\text{rg}(u^2) = \text{rg}(u)$ (with $u^2 = u \circ u$).
    \begin{enumerate}
        \item Show that $\text{Ker}(u) = \text{Ker}(u^2)$.
        \item Show that $E = \text{Im}(u) \oplus \text{Ker}(u)$.
    \end{enumerate}
    \item Let $f$ and $g$ be two endomorphisms of $E$ such that $\text{rg}(g \circ f) = \text{rg}(g)$.
    Show that $E = \text{Im}f + \text{Ker}g$.
\end{enumerate}

\newpage

\begin{answerbox}
% student's answer area
Let $ E $ be a finite-dimensional $ \mathbb{K} $-vector space.

\begin{enumerate}

\item Let $ u $ be an endomorphism of $ E $ such that $ \mathrm{rg}(u^2) = \mathrm{rg}(u) $, where $ u^2 = u \circ u $.

    \begin{enumerate}
        \item \textbf{Show that $ \ker(u) = \ker(u^2) $.}
        
        We always have:
        $$
        \ker(u) \subseteq \ker(u^2)
        $$
        since if $ x \in \ker(u) $, then $ u(x) = 0 $ and so $ u^2(x) = u(u(x)) = u(0) = 0 $, hence $ x \in \ker(u^2) $.
        
        Now, by assumption: 
        $$
        \dim(\ker(u^2)) = \dim(E) - \mathrm{rg}(u^2) = \dim(E) - \mathrm{rg}(u) = \dim(\ker(u))
        $$
        Therefore, since $ \ker(u) \subseteq \ker(u^2) $ and they have the same dimension, we conclude:
        $$
        \ker(u) = \ker(u^2)
        $$

        \item \textbf{Show that $ E = \mathrm{Im}(u) \oplus \ker(u) $.}
        
        By the Rank-Nullity Theorem:
        $$
        \dim(E) = \dim(\mathrm{Im}(u)) + \dim(\ker(u)) = \mathrm{rg}(u) + \dim(\ker(u))
        $$
        So it suffices to show that:
        $$
        \mathrm{Im}(u) \cap \ker(u) = \{0\}
        $$
        Suppose $ x \in \mathrm{Im}(u) \cap \ker(u) $. Then there exists $ y \in E $ such that $ x = u(y) $, and $ u(x) = 0 $.

        Since $ x = u(y) $, we get:
        $$
        u^2(y) = u(u(y)) = u(x) = 0 \Rightarrow y \in \ker(u^2) = \ker(u)
        $$
        Hence $ u(y) = x = 0 $, so $ x = 0 $. Thus:
        $$
        \mathrm{Im}(u) \cap \ker(u) = \{0\}
        $$
        And therefore:
        $$
        E = \mathrm{Im}(u) \oplus \ker(u)
        $$
    \end{enumerate}

\item \textbf{Let $ f $ and $ g $ be two endomorphisms of $ E $ such that $ \mathrm{rg}(g \circ f) = \mathrm{rg}(g) $. Show that $ E = \mathrm{Im}(f) + \ker(g) $.}

We know:
$$
\mathrm{rg}(g \circ f) = \dim(\mathrm{Im}(g \circ f)) = \dim(E) - \dim(\ker(g \circ f))
$$
But also:
$$
\mathrm{rg}(g) = \dim(\mathrm{Im}(g)) = \dim(E) - \dim(\ker(g))
$$

Given $ \mathrm{rg}(g \circ f) = \mathrm{rg}(g) $, we deduce:
$$
\dim(\ker(g \circ f)) = \dim(\ker(g))
$$

Now consider the restriction of $ f $ to $ E $, and apply the following:

Let’s define a linear map:
$$
T : E \to \mathrm{Im}(g), \quad T(x) = g(f(x))
$$
Then $ \ker(T) = \ker(g \circ f) $, and $ \mathrm{Im}(T) = \mathrm{Im}(g \circ f) $

By the rank-nullity theorem applied to $ T $, we get:
$$
\dim(E) = \dim(\ker(g \circ f)) + \dim(\mathrm{Im}(g \circ f)) = \dim(\ker(g)) + \dim(\mathrm{Im}(g)) = \dim(E)
$$

So $ T $ is surjective onto $ \mathrm{Im}(g) $, which implies that every vector in $ \mathrm{Im}(g) $ is of the form $ g(f(x)) $, i.e., $ \mathrm{Im}(g) \subseteq g(\mathrm{Im}(f)) $

Therefore, any $ x \in E $ can be written as $ x = y + z $ with $ y \in \mathrm{Im}(f) $ and $ z \in \ker(g) $, which gives:
$$
E = \mathrm{Im}(f) + \ker(g)
$$

\end{enumerate}

\end{answerbox}

\newpage

%----------------- EXERCISE 3 ------------------
\section{}
Let $\mathbb{R}_2[X]$ be the vector space of polynomials with real coefficients of degree less than or equal to 2.

Let $B = (1, X, X^2)$ be the canonical basis of $\mathbb{R}_2[X]$, and $f: \mathbb{R}_2[X] \to \mathbb{R}_2[X]$ be the endomorphism defined by:
$\forall P(X) \in \mathbb{R}_2[X]$,
\[f(P(X)) = (2X + 1)P(X) - (X^2 - 1)P'(X)\]
where $P'(X)$ is the derivative of $P(X)$.

\begin{enumerate}
    \item Show that $f$ is injective. Deduce that $f$ is an isomorphism.
    \item Determine the matrix $A$ of $f$ in the basis $B$.
    
    \textbf{Let the family of polynomials $B' = (X^2 - 1, (X - 1)^2, (X + 1)^2)$. It is admitted that $B'$ is a basis of $\mathbb{R}_2[X]$.}
    
    \item Calculate $f(X^2 - 1)$, $f((X - 1)^2)$, and $f((X + 1)^2)$. Deduce the matrix $A'$ of $f$ in the basis $B'$.
    \item 
    \begin{enumerate}
        \item Determine $P = P_{B}^{B'}$ the change of basis matrix from $B$ to $B'$. Calculate $P^{-1}$.
    \end{enumerate}
    \begin{enumerate}
        \item For all $n \in \mathbb{N}^*$, calculate $A^n$.
    \end{enumerate}
\end{enumerate}


\newpage

\begin{answerbox}
% student's answer area
Let $ \mathbb{R}_2[X] $ be the vector space of real polynomials of degree at most 2. \\ 
Let $ \mathcal{B} = (1, X, X^2) $ be the canonical basis of $ \mathbb{R}_2[X] $.  
Define the endomorphism $ f : \mathbb{R}_2[X] \to \mathbb{R}_2[X] $ by:
$$
f(P(X)) = (2X + 1)P(X) - (X^2 - 1)P'(X)
$$
where $ P'(X) $ denotes the derivative of $ P(X) $.

\begin{enumerate}

\item \textbf{Show that $ f $ is injective. Deduce that $ f $ is an isomorphism.}

We will show that $ \ker(f) = \{0\} $, i.e., only the zero polynomial satisfies $ f(P) = 0 $.

Let $ P(X) = a + bX + cX^2 \in \mathbb{R}_2[X] $, then:
$$
P'(X) = b + 2cX
$$
Now compute $ f(P) $:

$$
f(P) = (2X + 1)(a + bX + cX^2) - (X^2 - 1)(b + 2cX)
$$

Compute each term:

- $ (2X + 1)(a + bX + cX^2) = 2aX + 2bX^2 + 2cX^3 + a + bX + cX^2 $ \\
- $ (X^2 - 1)(b + 2cX) = bX^2 + 2cX^3 - b - 2cX $

Subtracting:

$$
f(P) = [2aX + 2bX^2 + 2cX^3 + a + bX + cX^2] - [bX^2 + 2cX^3 - b - 2cX]
$$

Simplify:

- Coefficient of $ X^3 $: $ 2c - 2c = 0 $ \\
- Coefficient of $ X^2 $: $ 2b + c - b = b + c $ \\
- Coefficient of $ X $: $ 2a + b + 2c $ \\
- Constant term: $ a + b $ \\

So:
$$
f(P) = (b + c)X^2 + (2a + b + 2c)X + (a + b)
$$

Set $ f(P) = 0 $. Then we solve the system:
$$
\begin{cases}
b + c = 0 \\
2a + b + 2c = 0 \\
a + b = 0
\end{cases}
$$

From the third equation: $ b = -a $

Plug into first: $ -a + c = 0 \Rightarrow c = a $

Plug into second: $ 2a - a + 2a = 3a = 0 \Rightarrow a = 0 $

Then $ b = 0 $, $ c = 0 $. So $ P = 0 $. Therefore, $ \ker(f) = \{0\} $, and since $ \dim(\mathbb{R}_2[X]) < \infty $, $ f $ is injective ⇒ $ f $ is an isomorphism.

\bigskip

\item \textbf{Determine the matrix $ A $ of $ f $ in the basis $ \mathcal{B} $.}

Compute $ f(1), f(X), f(X^2) $:

- $ f(1) = (2X + 1)(1) - (X^2 - 1)(0) = 2X + 1 $ \\
- $ f(X) = (2X + 1)(X) - (X^2 - 1)(1) = 2X^2 + X - X^2 + 1 = X^2 + X + 1 $ \\
- $ f(X^2) = (2X + 1)(X^2) - (X^2 - 1)(2X) = 2X^3 + X^2 - 2X^3 + 2X = X^2 + 2X $

Now express these in terms of $ \mathcal{B} = (1, X, X^2) $:

- $ f(1) = 1\cdot 1 + 2\cdot X + 0\cdot X^2 $ \\
- $ f(X) = 1\cdot 1 + 1\cdot X + 1\cdot X^2 $ \\
- $ f(X^2) = 0\cdot 1 + 2\cdot X + 1\cdot X^2 $ \\

Thus, the matrix of $ f $ in the basis $ \mathcal{B} $ is:
$$
A = 
\begin{bmatrix}
1 & 1 & 0 \\
2 & 1 & 2 \\
0 & 1 & 1
\end{bmatrix}
$$

\bigskip

\item It is given that $ \mathcal{B}' = (X^2 - 1, (X - 1)^2, (X + 1)^2) $ is a basis of $ \mathbb{R}_2[X] $.


\textbf{Calculate $ f(X^2 - 1), f((X - 1)^2), f((X + 1)^2) $. Deduce the matrix $ A' $ of $ f $ in the basis $ \mathcal{B}' $.}

First compute:

- $ f(X^2 - 1) = (2X + 1)(X^2 - 1) - (X^2 - 1)(2X) = (2X + 1)(X^2 - 1) - 2X(X^2 - 1) = (X^2 - 1) $ \\

So $ f(X^2 - 1) = X^2 - 1 $ \\

- $ f((X - 1)^2) = f(X^2 - 2X + 1) = (2X + 1)(X^2 - 2X + 1) - (X^2 - 1)(2X - 2) $ \\

Expand: \\
- First term: $ (2X + 1)(X^2 - 2X + 1) = 2X^3 - 4X^2 + 2X + X^2 - 2X + 1 = 2X^3 - 3X^2 + 1 $ \\
- Second term: $ (X^2 - 1)(2X - 2) = 2X^3 - 2X^2 - 2X + 2 $ \\

Subtract:
$$
f((X - 1)^2) = (2X^3 - 3X^2 + 1) - (2X^3 - 2X^2 - 2X + 2) =  -(X - 1)^2
$$

Similarly, \\
- $ f((X + 1)^2) = f(X^2 + 2X + 1) = (2X + 1)(X^2 + 2X + 1) - (X^2 - 1)(2X + 2) $ \\

Compute: \\
- First term: $ (2X + 1)(X^2 + 2X + 1) = 2X^3 + 4X^2 + 2X + X^2 + 2X + 1 = 2X^3 + 5X^2 + 4X + 1 $ \\
- Second term: $ (X^2 - 1)(2X + 2) = 2X^3 + 2X^2 - 2X - 2 $ \\

Subtract:
$$
f((X + 1)^2) = (2X^3 + 5X^2 + 4X + 1) - (2X^3 + 2X^2 - 2X - 2) = 3(X + 1)^2
$$

Therefore:

- $ f(X^2 - 1) = 1\cdot (X^2 - 1) + 0\cdot (X - 1)^2 + 0\cdot (X + 1)^2 $ \\
- $ f((X - 1)^2) = 0\cdot (X^2 - 1) - 1\cdot (X - 1)^2 + 0\cdot (X + 1)^2 $ \\
- $ f((X + 1)^2) = 0\cdot (X^2 - 1) + 0\cdot (X - 1)^2 + 3\cdot (X + 1)^2 $ \\

So the matrix of $ f $ in the basis $ \mathcal{B}' $ is:
$$
A' =
\begin{bmatrix}
1 & 0 & 0 \\
0 & -1 & 0 \\
0 & 0 & 3
\end{bmatrix}
$$

\bigskip

\item \begin{enumerate}
    

\item \textbf{ Determine $ P = P_{\mathcal{B}}^{\mathcal{B}'} $, the change of basis matrix from $ \mathcal{B} $ to $ \mathcal{B}' $. Calculate $ P^{-1} $.}

Recall $ \mathcal{B} = (1, X, X^2) $, $ \mathcal{B}' = (X^2 - 1, (X - 1)^2, (X + 1)^2) $

Express each element of $ \mathcal{B}' $ in terms of $ \mathcal{B} $:

- $ X^2 - 1 = -1\cdot 1 + 0\cdot X + 1\cdot X^2 $ \\
- $ (X - 1)^2 = 1\cdot 1 - 2\cdot X + 1\cdot X^2 $ \\
- $ (X + 1)^2 = 1\cdot 1 + 2\cdot X + 1\cdot X^2 $ \\

So the change of basis matrix from $ \mathcal{B} $ to $ \mathcal{B}' $ is:
$$
P =
\begin{bmatrix}
-1 & 1 & 1 \\
0 & -2 & 2 \\
1 & 1 & 1
\end{bmatrix}
$$

To find $ P^{-1} $, use standard methods or software. The inverse is:
$$
P^{-1} =
\frac{1}{4}
\begin{bmatrix}
-4 & 0 & 4 \\
-2 & -2 & 2 \\
1 & 2 & 1
\end{bmatrix}
$$

\bigskip

\item \textbf{ For all $ n \in \mathbb{N}^* $, calculate $ A^n $.}

Since $ A' = P^{-1} A P $, and $ A' $ is diagonal:
$$
A'^n =
\begin{bmatrix}
1^n & 0 & 0 \\
0 & (-1)^n & 0 \\
0 & 0 & 3^n
\end{bmatrix}
=
\begin{bmatrix}
1 & 0 & 0 \\
0 & (-1)^n & 0 \\
0 & 0 & 3^n
\end{bmatrix}
$$

Then:
$$
A^n = P A'^n P^{-1}
$$

This gives a formula for computing $ A^n $ using matrix multiplication.

\end{enumerate}

\end{enumerate}
\end{answerbox}

\newpage

%----------------- EXERCISE 4 ------------------
\section{}
Let the matrix $A = \begin{pmatrix} 1 & 1 & 1 \\ a & b & c \\ a^2 & b^2 & c^2 \end{pmatrix}$, where $a, b, c \in \mathbb{R}$.

Using Gaussian elimination, determine the rank of $A$ based on the values of the real numbers $a$, $b$, and $c$.

\newpage

\begin{answerbox}
    Let the matrix:
    $$
    A = 
    \begin{bmatrix}
    1 & 1 & 1 \\
    a & b & c \\
    a^2 & b^2 & c^2
    \end{bmatrix}, \quad \text{with } a, b, c \in \mathbb{R}.
    $$
    
    Using Gaussian elimination, determine the rank of $ A $ depending on the values of $ a, b, c $.
    
    \begin{enumerate}
    \item Start with the matrix:
    $$
    \begin{bmatrix}
    1 & 1 & 1 \\
    a & b & c \\
    a^2 & b^2 & c^2
    \end{bmatrix}
    $$
    
    Perform row operations:
    - $ R_2 \leftarrow R_2 - a R_1 $
    - $ R_3 \leftarrow R_3 - a^2 R_1 $
    
    We get:
    $$
    \begin{bmatrix}
    1 & 1 & 1 \\
    0 & b - a & c - a \\
    0 & b^2 - a^2 & c^2 - a^2
    \end{bmatrix}
    $$
    
    Note that $ b^2 - a^2 = (b - a)(b + a) $, so we eliminate further by:
    - $ R_3 \leftarrow R_3 - (b + a) R_2 $
    
    Final form:
    $$
    \begin{bmatrix}
    1 & 1 & 1 \\
    0 & b - a & c - a \\
    0 & 0 & (c - a)(c - b)
    \end{bmatrix}
    $$
    
    \item Conclude:
    
    The number of non-zero rows depends on whether $ b \ne a $ and $ (c - a)(c - b) \ne 0 $. We distinguish cases:
    
    \begin{enumerate}
        \item If $ a, b, c $ are all distinct: $ \boxed{\mathrm{rank}(A) = 3} $
        \item If exactly two of $ a, b, c $ are equal: $ \boxed{\mathrm{rank}(A) = 2} $
        \item If $ a = b = c $: $ \boxed{\mathrm{rank}(A) = 1} $
    \end{enumerate}
    \end{enumerate}
\end{answerbox}

%----------------- END ------------------
\end{document}
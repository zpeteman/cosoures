\documentclass[12pt]{article}

%----------------- Packages ------------------
\usepackage[utf8]{inputenc}
\usepackage[T1]{fontenc}
\usepackage{amsmath, amssymb}
\usepackage{geometry}
\usepackage{xcolor}
\usepackage{titlesec}
\usepackage{fancyhdr}
\usepackage[breakable]{tcolorbox}
\usepackage{graphicx}
\usepackage{tikz}
\usepackage{float}
\usetikzlibrary{arrows.meta, decorations.markings}

%----------------- Page Setup -----------------
\geometry{a4paper, margin=2.5cm}
\pagestyle{fancy}
\fancyhf{}
\rhead{Make-up Exam — \textbf{19-20}}
\lhead{Algebra II}
\cfoot{\thepage}

%----------------- Title Styling --------------
\titleformat{\section}{\normalfont\Large\bfseries}{Exercise \thesection:}{1em}{}
\titleformat{\subsection}{\normalfont\bfseries}{Answer:}{1em}{}

%----------------- Custom Boxes ----------------
\tcbuselibrary{listingsutf8}
\newtcolorbox{answerbox}{
  colback=gray!10,
  colframe=black,
  fonttitle=\bfseries,
  title=Answer Area,
  breakable,
  before skip=10pt,
  after skip=10pt
}

%----------------- Document Start --------------
\begin{document}

%----------------- Exam Info -------------------
\begin{center}
  \Large\textbf{Ibn Tofail University} \\[1em]
  \large\textit{Algebra II — Make-up Exam} \\[0.5em]
  \large\textit{Year: 19-20} \\[2em]
\end{center}

\vspace{0.5cm}

%----------------- EXERCISE 1 ------------------
\section{}
Let $E$ be a $K$-vector space, $p \in \mathbb{N}^*$, and $u_1, u_2, ..., u_p, u_{p+1}$ be vectors in $E$.

\begin{enumerate}
    \item[1)] Assume that $\{u_1, u_2, ..., u_p\}$ is linearly independent. Show the equivalence:\\
    $u_{p+1} \notin \text{span}(\{u_1, u_2, ..., u_p\}) \Leftrightarrow \{u_1, u_2, ..., u_p, u_{p+1}\}$ is linearly independent.
    
    \item[2)] Assume that $\{u_1, u_2, ..., u_p, u_{p+1}\}$ is a generating set of $E$ and that\\
    $u_{p+1} \in \text{span}(\{u_1, u_2, ..., u_p\})$. Show that $\{u_1, u_2, ..., u_p\}$ is a generating set of $E$.
\end{enumerate}

\newpage

\begin{answerbox}
% student's answer area
Let $ E $ be a $ K $-vector space, $ p \in \mathbb{N}^* $, and let $ u_1, u_2, \ldots, u_p, u_{p+1} $ be vectors in $ E $.

\begin{enumerate}
    \item Assume that $ \{u_1, u_2, \ldots, u_p\} $ is linearly independent. Show the equivalence:
    $$
    u_{p+1} \notin \text{span}(\{u_1, u_2, \ldots, u_p\}) \iff \{u_1, u_2, \ldots, u_p, u_{p+1}\} \text{is linearly independent}.
    $$

    \textbf{Proof:}
    
    \begin{description}
        \item[$\Rightarrow$] Suppose $ u_{p+1} \notin \text{span}(\{u_1, \ldots, u_p\}) $. We want to show that $ \{u_1, \ldots, u_p, u_{p+1}\} $ is linearly independent.
        
        Let 
        $$
        \lambda_1 u_1 + \cdots + \lambda_p u_p + \lambda_{p+1} u_{p+1} = 0.
        $$
        
        If $ \lambda_{p+1} \neq 0 $, then we can solve for $ u_{p+1} $:
        $$
        u_{p+1} = -\frac{\lambda_1}{\lambda_{p+1}} u_1 - \cdots - \frac{\lambda_p}{\lambda_{p+1}} u_p,
        $$
        which contradicts the assumption that $ u_{p+1} \notin \text{span}(\{u_1, \ldots, u_p\}) $. Therefore, $ \lambda_{p+1} = 0 $.

        Then we are left with:
        $$
        \lambda_1 u_1 + \cdots + \lambda_p u_p = 0.
        $$
        Since $ \{u_1, \ldots, u_p\} $ is linearly independent, all $ \lambda_i = 0 $. Hence, $ \{u_1, \ldots, u_p, u_{p+1}\} $ is linearly independent.

        \item[$\Leftarrow$] Conversely, suppose $ \{u_1, \ldots, u_p, u_{p+1}\} $ is linearly independent. Then $ u_{p+1} $ cannot be written as a linear combination of $ u_1, \ldots, u_p $, otherwise we would have a nontrivial linear relation among these vectors. Thus:
        $$
        u_{p+1} \notin \text{span}(\{u_1, \ldots, u_p\}).
        $$
    \end{description}

    This completes the proof of the equivalence.

    \item Assume that $ \{u_1, u_2, \ldots, u_p, u_{p+1}\} $ is a generating set of $ E $ and that
    $$
    u_{p+1} \in \text{span}(\{u_1, \ldots, u_p\}).
    $$
    Show that $ \{u_1, \ldots, u_p\} $ is a generating set of $ E $.

    \textbf{Proof:}
    
    Since $ \{u_1, \ldots, u_p, u_{p+1}\} $ generates $ E $, every vector $ v \in E $ can be written as:
    $$
    v = \lambda_1 u_1 + \cdots + \lambda_p u_p + \lambda_{p+1} u_{p+1}.
    $$

    But since $ u_{p+1} \in \text{span}(\{u_1, \ldots, u_p\}) $, there exist scalars $ \mu_1, \ldots, \mu_p \in K $ such that:
    $$
    u_{p+1} = \mu_1 u_1 + \cdots + \mu_p u_p.
    $$

    Substituting into the expression for $ v $, we get:
    $$
    v = \lambda_1 u_1 + \cdots + \lambda_p u_p + \lambda_{p+1} (\mu_1 u_1 + \cdots + \mu_p u_p)
    = (\lambda_1 + \lambda_{p+1} \mu_1) u_1 + \cdots + (\lambda_p + \lambda_{p+1} \mu_p) u_p.
    $$

    So $ v \in \text{span}(\{u_1, \ldots, u_p\}) $, hence $ \{u_1, \ldots, u_p\} $ generates $ E $.

\end{enumerate}
\end{answerbox}

\newpage  
%----------------- EXERCISE 2 ------------------
\section{}
In the $\mathbb{R}$-vector space $\mathbb{R}^4$, consider the following vector subspaces:\\
$F = \text{span}(\{(1,0,1,0), (0,1,0,1)\})$\\
$G = \{(x,y,z,t) \in \mathbb{R}^4 \mid x + y = 0 \text{ and } z + t = 0\}$\\
$H = \{(x,y,z,t) \in \mathbb{R}^4 \mid x - y + z - t = 0\}$

\begin{enumerate}
    \item[1)] Determine a basis and the dimension of $G$.
    \item[2)] Determine a basis and the dimension of $H$.
    \item[3)] Determine a basis of the vector subspace $F \cap G$.
    \item[4)] Show that $\mathbb{R}^4 = (F \cap G) \oplus H$.
\end{enumerate}

\newpage

\begin{answerbox}
In the $ \mathbb{R} $-vector space $ \mathbb{R}^4 $, consider the following vector subspaces:

$$
F = \text{span}\left(\{(1, 0, 1, 0), (0, 1, 0, 1)\}\right)
$$

$$
G = \{(x, y, z, t) \in \mathbb{R}^4 \mid x + y = 0 \text{ and } z + t = 0 \}
$$

$$
H = \{(x, y, z, t) \in \mathbb{R}^4 \mid x - y + z - t = 0 \}
$$

\begin{enumerate}
    \item \textbf{Determine a basis and the dimension of $ G $.}

    A vector $ (x, y, z, t) \in G $ satisfies:
    $$
    x + y = 0 \quad \text{and} \quad z + t = 0
    $$

    Solving these equations:
    $$
    y = -x, \quad t = -z
    $$

    So any vector in $ G $ can be written as:
    $$
    (x, -x, z, -z) = x(1, -1, 0, 0) + z(0, 0, 1, -1)
    $$

    Therefore, a basis for $ G $ is:
    $$
    \{(1, -1, 0, 0), (0, 0, 1, -1)\}
    $$
    and
    $$
    \dim(G) = 2
    $$

    \item \textbf{Determine a basis and the dimension of $ H $.}

    A vector $ (x, y, z, t) \in H $ satisfies:
    $$
    x - y + z - t = 0
    $$

    We can express one variable in terms of others. Let’s solve for $ t $:
    $$
    t = x - y + z
    $$

    Then any vector in $ H $ can be written as:
    $$
    (x, y, z, x - y + z) = x(1, 0, 0, 1) + y(0, 1, 0, -1) + z(0, 0, 1, 1)
    $$

    Therefore, a basis for $ H $ is:
    $$
    \{(1, 0, 0, 1), (0, 1, 0, -1), (0, 0, 1, 1)\}
    $$
    and
    $$
    \dim(H) = 3
    $$

    \item \textbf{Determine a basis of the vector subspace $ F \cap G $.}

    Recall:
    $$
    F = \text{span}\left(\{(1, 0, 1, 0), (0, 1, 0, 1)\}\right)
    $$
    $$
    G = \text{span}\left(\{(1, -1, 0, 0), (0, 0, 1, -1)\}\right)
    $$

    Let’s find vectors in $ F $ that also belong to $ G $.

    Any vector in $ F $ has the form:
    $$
    a(1, 0, 1, 0) + b(0, 1, 0, 1) = (a, b, a, b)
    $$

    For this vector to be in $ G $, it must satisfy:
    $$
    x + y = a + b = 0 \Rightarrow b = -a
    $$
    $$
    z + t = a + b = 0 \Rightarrow b = -a
    $$

    So, substituting $ b = -a $, we get:
    $$
    (a, -a, a, -a) = a(1, -1, 1, -1)
    $$

    Therefore, $ F \cap G = \text{span}\left(\{(1, -1, 1, -1)\}\right) $, and a basis is:
    $$
    \{(1, -1, 1, -1)\}
    $$
    with
    $$
    \dim(F \cap G) = 1
    $$

    \item \textbf{Show that $ \mathbb{R}^4 = (F \cap G) \oplus H $.}

    To prove that $ \mathbb{R}^4 = (F \cap G) \oplus H $, we need to show two things:

    
        
- $ \dim((F \cap G) \oplus H) = 4 $
        
- $ (F \cap G) \cap H = \{0\} $
    

    From above:
    $$
    \dim(F \cap G) = 1, \quad \dim(H) = 3 \Rightarrow \dim((F \cap G) + H) \leq 4
    $$

    Since both are subspaces of $ \mathbb{R}^4 $, their sum is at most 4-dimensional.

    Now check if the intersection is trivial: suppose $ v \in (F \cap G) \cap H $

    Then $ v = a(1, -1, 1, -1) $, and also $ v \in H \Rightarrow x - y + z - t = 0 $

    Compute:
    $$
    x - y + z - t = a - (-a) + a - (-a) = a + a + a + a = 4a
    $$

    Set equal to zero:
    $$
    4a = 0 \Rightarrow a = 0 \Rightarrow v = 0
    $$

    Therefore, $ (F \cap G) \cap H = \{0\} $, and since the dimensions add up to 4, we conclude:
    $$
    \mathbb{R}^4 = (F \cap G) \oplus H
    $$

\end{enumerate}
\end{answerbox}

\newpage  
%----------------- EXERCISE 3 ------------------
\section{}
Let $E = \mathbb{R}_2[X]$ be the vector space of polynomials with real coefficients of degree less than or equal to 2. And let $f: E \rightarrow E$ be the linear map defined by:\\
for all $P(X) \in E$, $f(P(X)) = 2P(X) - (X - 1)P'(X)$\\
(where $P'(X)$ denotes the derivative of the polynomial $P(X)$).

\begin{enumerate}
    \item[1)] Determine $\text{Ker}f$.
    \item[2)] Determine a complement of $\text{Ker}f$ in $E$.
\end{enumerate}

\newpage

\begin{answerbox}
% student's answer area
Let $ E = \mathbb{R}_2[X] $ be the vector space of real polynomials of degree less than or equal to 2. Define the linear map $ f: E \to E $ by:
$$
f(P(X)) = 2P(X) - (X - 1)P'(X)
$$
where $ P'(X) $ denotes the derivative of $ P(X) $.

We are asked to:

\begin{enumerate}
    \item \textbf{Determine $ \ker(f) $.}
    
    Recall that:
    $$
    \ker(f) = \{ P(X) \in E \mid f(P(X)) = 0 \}
    $$

    Let $ P(X) = a + bX + cX^2 $, where $ a, b, c \in \mathbb{R} $. Then:
    $$
    P'(X) = b + 2cX
    $$
    So,
    \begin{align*}
    f(P(X)) &= 2P(X) - (X - 1)P'(X) \\
    &= 2(a + bX + cX^2) - (X - 1)(b + 2cX)
    \end{align*}

    Compute each part:
    $$
    2P(X) = 2a + 2bX + 2cX^2
    $$
    $$
    (X - 1)(b + 2cX) = X(b + 2cX) - 1(b + 2cX) = bX + 2cX^2 - b - 2cX
    $$
    Combine:
    $$
    f(P(X)) = 2a + 2bX + 2cX^2 - (bX + 2cX^2 - b - 2cX)
    $$
    Simplify:
    $$
    f(P(X)) = 2a + 2bX + 2cX^2 - bX - 2cX^2 + b + 2cX
    $$
    Group like terms:
    $$
    f(P(X)) = (2a + b) + (2b - b + 2c)X + (2c - 2c)X^2 = (2a + b) + (b + 2c)X
    $$

    Set this equal to the zero polynomial:
    $$
    (2a + b) + (b + 2c)X = 0
    $$

    This gives the system:
    $$
    \begin{cases}
    2a + b = 0 \\
    b + 2c = 0
    \end{cases}
    $$

    Solve:
    From the second equation: $ b = -2c $

    Plug into the first: $ 2a - 2c = 0 \Rightarrow a = c $

    Therefore, the general form of $ P(X) \in \ker(f) $ is:
    $$
    P(X) = c + (-2c)X + cX^2 = c(1 - 2X + X^2)
    $$

    Thus:
    $$
    \ker(f) = \text{span}\left(\{1 - 2X + X^2\}\right)
    $$
    and
    $$
    \dim(\ker(f)) = 1
    $$

    \item \textbf{Determine a complement of $ \ker(f) $ in $ E $.}

    Since $ E = \mathbb{R}_2[X] $ has dimension 3, and $ \dim(\ker(f)) = 1 $, we need to find a subspace $ W \subset E $ such that:
    $$
    E = \ker(f) \oplus W
    $$
    i.e., $ W $ has dimension 2 and intersects $ \ker(f) $ trivially.

    A standard basis for $ E $ is $ \{1, X, X^2\} $. We already know that:
    $$
    \ker(f) = \text{span}\left(\{1 - 2X + X^2\}\right)
    $$

    To find a complement, pick two vectors from the standard basis that are not in $ \ker(f) $, and whose span does not intersect $ \ker(f) $ except at 0.

    Consider the subspace:
    $$
    W = \text{span}\left(\{1, X\}\right)
    $$

    Check if $ W \cap \ker(f) = \{0\} $:

    Suppose $ a + bX = c(1 - 2X + X^2) $

    Then:
    $$
    a = c,\quad b = -2c,\quad 0 = cX^2
    \Rightarrow c = 0 \Rightarrow a = b = 0
    $$

    So $ W \cap \ker(f) = \{0\} $

    Also, $ \dim(W) = 2 $, $ \dim(\ker(f)) = 1 $, so:
    $$
    E = \ker(f) \oplus W
    $$

    Therefore, $ W = \text{span}(\{1, X\}) $ is a complement of $ \ker(f) $ in $ E $.

\end{enumerate}
\end{answerbox}

%----------------- END ------------------
\end{document}
\documentclass[12pt]{article}

%----------------- Packages ------------------
\usepackage[utf8]{inputenc}
\usepackage[T1]{fontenc}
\usepackage{amsmath, amssymb}
\usepackage{geometry}
\usepackage{xcolor}
\usepackage{titlesec}
\usepackage{fancyhdr}
\usepackage[breakable]{tcolorbox}
\usepackage{graphicx}
\usepackage{tikz}
\usepackage{float}
\usetikzlibrary{arrows.meta, decorations.markings}

%----------------- Page Setup -----------------
\geometry{a4paper, margin=2.5cm}
\pagestyle{fancy}
\fancyhf{}
\rhead{Make-up Exam — \textbf{19-20}}
\lhead{Algebra II}
\cfoot{\thepage}

%----------------- Title Styling --------------
\titleformat{\section}{\normalfont\Large\bfseries}{Exercise \thesection:}{1em}{}
\titleformat{\subsection}{\normalfont\bfseries}{Answer:}{1em}{}

%----------------- Custom Boxes ----------------
\tcbuselibrary{listingsutf8}
\newtcolorbox{answerbox}{
  colback=gray!10,
  colframe=black,
  fonttitle=\bfseries,
  title=Answer Area,
  breakable,
  before skip=10pt,
  after skip=10pt
}

%----------------- Document Start --------------
\begin{document}

%----------------- Exam Info -------------------
\begin{center}
  \Large\textbf{Ibn Tofail University} \\[1em]
  \large\textit{Algebra II — Make-up Exam} \\[0.5em]
  \large\textit{Year: 19-20} \\[2em]
\end{center}

\vspace{0.5cm}

%----------------- EXERCISE 1 ------------------
\section{}
Let $E$ be a $K$-vector space, $p \in \mathbb{N}^*$, and $u_1, u_2, ..., u_p, u_{p+1}$ be vectors in $E$.

\begin{enumerate}
    \item[1)] Assume that $\{u_1, u_2, ..., u_p\}$ is linearly independent. Show the equivalence:\\
    $u_{p+1} \notin \text{span}(\{u_1, u_2, ..., u_p\}) \Leftrightarrow \{u_1, u_2, ..., u_p, u_{p+1}\}$ is linearly independent.
    
    \item[2)] Assume that $\{u_1, u_2, ..., u_p, u_{p+1}\}$ is a generating set of $E$ and that\\
    $u_{p+1} \in \text{span}(\{u_1, u_2, ..., u_p\})$. Show that $\{u_1, u_2, ..., u_p\}$ is a generating set of $E$.
\end{enumerate}


\begin{answerbox}
% student's answer area
\end{answerbox}

\newpage  
%----------------- EXERCISE 2 ------------------
\section{}
In the $\mathbb{R}$-vector space $\mathbb{R}^4$, consider the following vector subspaces:\\
$F = \text{span}(\{(1,0,1,0), (0,1,0,1)\})$\\
$G = \{(x,y,z,t) \in \mathbb{R}^4 \mid x + y = 0 \text{ and } z + t = 0\}$\\
$H = \{(x,y,z,t) \in \mathbb{R}^4 \mid x - y + z - t = 0\}$

\begin{enumerate}
    \item[1)] Determine a basis and the dimension of $G$.
    \item[2)] Determine a basis and the dimension of $H$.
    \item[3)] Determine a basis of the vector subspace $F \cap G$.
    \item[4)] Show that $\mathbb{R}^4 = (F \cap G) \oplus H$.
\end{enumerate}


\begin{answerbox}
% student's answer area
\end{answerbox}

\newpage  
%----------------- EXERCISE 3 ------------------
\section{}
Let $E = \mathbb{R}_2[X]$ be the vector space of polynomials with real coefficients of degree less than or equal to 2. And let $f: E \rightarrow E$ be the linear map defined by:\\
for all $P(X) \in E$, $f(P(X)) = 2P(X) - (X - 1)P'(X)$\\
(where $P'(X)$ denotes the derivative of the polynomial $P(X)$).

\begin{enumerate}
    \item[1)] Determine $\text{Ker}f$.
    \item[2)] Determine a complement of $\text{Ker}f$ in $E$.
\end{enumerate}


\begin{answerbox}
% student's answer area
\end{answerbox}

%----------------- END ------------------
\end{document}
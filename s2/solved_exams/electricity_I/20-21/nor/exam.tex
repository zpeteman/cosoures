\documentclass[12pt]{article}

%----------------- Packages ------------------
\usepackage[utf8]{inputenc}
\usepackage[T1]{fontenc}
\usepackage{amsmath, amssymb}
\usepackage{geometry}
\usepackage{xcolor}
\usepackage{titlesec}
\usepackage{fancyhdr}
\usepackage[breakable]{tcolorbox}
\usepackage{graphicx}
\usepackage{tikz}
\usepackage{float}
\usetikzlibrary{arrows.meta, decorations.markings}

%----------------- Page Setup -----------------
\geometry{a4paper, margin=2.5cm}
\pagestyle{fancy}
\fancyhf{}
\rhead{Normal Exam — \textbf{20-21}}
\lhead{Electricity I}
\cfoot{\thepage}

%----------------- Title Styling --------------
\titleformat{\section}{\normalfont\Large\bfseries}{Exercise \thesection:}{1em}{}
\titleformat{\subsection}{\normalfont\bfseries}{Answer:}{1em}{}

%----------------- Custom Boxes ----------------
\tcbuselibrary{listingsutf8}
\newtcolorbox{answerbox}{
  colback=gray!10,
  colframe=black,
  fonttitle=\bfseries,
  title=Answer Area,
  breakable,
  before skip=10pt,
  after skip=10pt
}

%----------------- Document Start --------------
\begin{document}

%----------------- Exam Info -------------------
\begin{center}
  \Large\textbf{Ibn Tofail University} \\[1em]
  \large\textit{Electricity I — Normal Exam} \\[0.5em]
  \large\textit{Year: 20-21} \\[2em]
\end{center}

\vspace{0.5cm}

%----------------- EXERCISE 1 ------------------
\section{}
\subsection*{A/} An infinite cylinder with axis Oz and radius $R$ carries a uniform positive surface charge density $\sigma$.

\begin{enumerate}
    \item Show that the electrostatic field created at any point $M$ is radial and depends only on the distance $r = HM$ where $H$ is the projection of $M$ on the Oz axis (figure 1).
    
    We can therefore write:
    $\vec{E}(r) = E(r) \vec{e_r}$
    
    \item Calculate $E(r)$ at any point $M$ in space.
    
    \item Plot the variation of the magnitude of the electrostatic field $E$ as a function of $r$.
\end{enumerate}


\begin{figure}[H]
    \centering
    % Cylinder figure - compact version
    \begin{tikzpicture}[scale=1.5] % Reduced scale to make it smaller

    % Z-axis
    \draw[-{Stealth[length=2mm]}] (0,0) -- (0,3.5) node[above] {$z$};
    
    % Cylinder
    \draw (0,0.5) ellipse (0.8cm and 0.25cm); % Bottom ellipse
    \draw (0,2.5) ellipse (0.8cm and 0.25cm); % Top ellipse
    \draw (-0.8,0.5) -- (-0.8,2.5); % Left line
    \draw (0.8,0.5) -- (0.8,2.5);   % Right line
    
    % Dotted lines for cylinder
    \draw[dotted] (-0.8,0) -- (-0.8,0.5);
    \draw[dotted] (0.8,0) -- (0.8,0.5);
    \draw[dotted] (-0.8,2.5) -- (-0.8,3.5);
    \draw[dotted] (0.8,2.5) -- (0.8,3.5);
    \draw[dotted] (0,0) -- (0,0.5);
    
    % Radius R with arrow
    \draw[<->] (-0.8,0) -- (0,0) node[midway, below] {$R$};
    
    % Point H on z-axis
    \filldraw (0,1.7) circle (0.04) node[left=0.08cm] {$H$};
    
    % Point M
    \filldraw (1.6,1.7) circle (0.04) node[right=0.08cm] {$M$};
    
    % r vector from H to M
    \draw[->] (0,1.7) -- (1.6,1.7) node[midway, above] {$r$};
    
    % Vector e_r
    \draw[-{Stealth[length=2mm]}] (1.7,1.7) -- (2.2,1.7) node[right] {$\vec{e}_{r}$};
    
\end{tikzpicture}
\caption{Cylinder figure}
\end{figure}

\subsection*{B/} Consider a cylindrical capacitor formed by two armatures with the same Oz axis, height $h$, and respective radii $R_1$ and $R_2$ with $R_1 < R_2$.

The internal armature carries a charge $Q > 0$, and the electrostatic potentials of the internal and external armatures are $V_1$ and $V_2$ respectively.

\begin{enumerate}
    \item Determine the capacitance $C$ of the capacitor as a function of $\varepsilon_0$, $h$, $R_1$, and $R_2$.
    
    \item What becomes of the expression for $C$ if the radii of the armatures are very close.
    Let $R_2 - R_1 = e \ll R_1$. Express $C$ as a function of the surface area $s$ of the internal armature.
\end{enumerate}

\newpage

\begin{answerbox}
\subsection*{A/}
    \begin{enumerate}
        \item \textbf{Show that the electrostatic field is radial and depends only on $r = HM$:}
    
            The infinite cylinder has cylindrical symmetry. Due to this symmetry:
            
                
    - The electric field must be radial (i.e., directed along $\vec{e}_r$).
                
    - The magnitude of the field depends only on the radial distance $r$ from the axis (Oz), since there is no dependence on $z$ or angular direction.
            
    
            Therefore, we can write:
            $$
            \vec{E}(r) = E(r)\,\vec{e}_r
            $$
    
        \item \textbf{Calculate $E(r)$ at any point M in space:}
    
            Use Gauss's Law:
            $$
            \oint \vec{E} \cdot d\vec{S} = \frac{Q_{\text{enc}}}{\varepsilon_0}
            $$
    
            Consider a Gaussian surface as a coaxial cylinder of radius $r$ and height $h$:
            
                
    - For $r < R$: No charge enclosed, so
                    $$
                    E(r) = 0
                    $$
                
    - For $r \geq R$: Enclosed charge is $Q_{\text{enc}} = \sigma \cdot 2\pi R h$, and the flux becomes:
                    $$
                    E(r) \cdot 2\pi r h = \frac{\sigma \cdot 2\pi R h}{\varepsilon_0}
                    \Rightarrow E(r) = \frac{\sigma R}{\varepsilon_0 r}
                    $$
            
    
            Final result:
            $$
            E(r) =
            \begin{cases}
                0 & \text{if } r < R \\
                \dfrac{\sigma R}{\varepsilon_0 r} & \text{if } r \geq R
            \end{cases}
            $$
    
        \item \textbf{Plot the variation of the magnitude of the electrostatic field $E$ as a function of $r$:}
    
            A qualitative plot shows:
            
                
    - $E(r) = 0$ for $r < R$
                
    - $E(r)$ decreases as $1/r$ for $r \geq R$
            
    
            The graph would have a sharp jump at $r = R$, starting from zero to a maximum value of $\frac{\sigma}{\varepsilon_0}$, then decreasing.
    
    \end{enumerate}

\subsection*{B/}
    \begin{enumerate}
        \item \textbf{Determine the capacitance $C$ of the capacitor:}
    
            The electric field between two cylinders is given by (from Gauss’s law):
            $$
            E(r) = \frac{\lambda}{2\pi\varepsilon_0 r}, \quad \text{where } \lambda = \frac{Q}{h}
            $$
    
            The potential difference between the two cylinders is:
            $$
            V = \int_{R_1}^{R_2} E(r)\,dr = \frac{Q}{2\pi\varepsilon_0 h} \ln\left(\frac{R_2}{R_1}\right)
            $$
    
            Then, the capacitance is:
            $$
            C = \frac{Q}{V} = \frac{2\pi\varepsilon_0 h}{\ln\left(\frac{R_2}{R_1}\right)}
            $$
    
        \item \textbf{Approximation when $R_2 - R_1 = e \ll R_1$:}
    
            When $e \ll R_1$, we can approximate:
            $$
            \ln\left(\frac{R_2}{R_1}\right) = \ln\left(1 + \frac{e}{R_1}\right) \approx \frac{e}{R_1}
            $$
    
            Substituting into the expression for $C$:
            $$
            C \approx \frac{2\pi\varepsilon_0 h R_1}{e}
            $$
    
            Now, note that the surface area of the internal armature is:
            $$
            S = 2\pi R_1 h
            $$
    
            So:
            $$
            C \approx \frac{\varepsilon_0 S}{e}
            $$
    
    \end{enumerate}
\end{answerbox}

%----------------- END ------------------
\end{document}
\documentclass[12pt]{article}
\usepackage[utf8]{inputenc}
\usepackage[T1]{fontenc}
\usepackage[french]{babel}
\usepackage{amsmath}
\usepackage{amssymb}
\usepackage{geometry}
\geometry{a4paper, margin=1in}

\title{Algorithm II}
\author{Normal}
\date{2023-2024}

\begin{document}

\maketitle

\subsection*{Exercice 1 : Fonction récursive \texttt{sommeChiffres}}
La fonction récursive \texttt{sommeChiffres} calcule la somme des chiffres d'un entier \texttt{n}. Le cas de base est un nombre à un seul chiffre. L'étape récursive consiste à additionner le dernier chiffre (\texttt{n mod 10}) avec la somme des chiffres du reste du nombre (\texttt{n div 10}).

\begin{verbatim}
Fonction sommeChiffres(n: entier): entier
Debut
    Si n < 10 alors
        Retourner n
    Sinon
        Retourner (n mod 10) + sommeChiffres(n div 10)
    FinSi
Fin
\end{verbatim}





\subsection*{Exercice 2.1 : Fonction itérative \texttt{premier}}
Cette fonction détermine si un entier positif \texttt{n} est un nombre premier. Un nombre est premier s'il est supérieur à 1 et n'a pas d'autres diviseurs que 1 et lui-même. On teste la divisibilité de 2 jusqu'à \texttt{n-1}.

\begin{verbatim}
Fonction premier(n: entier): booléen
Variables
    i: entier
Debut
    Si n <= 1 alors
        Retourner Faux
    FinSi
    
    Pour i de 2 a (n - 1) Faire
        Si (n mod i = 0) alors
            Retourner Faux
        FinSi
    FinPour
    
    Retourner Vrai
Fin
\end{verbatim}

\subsection*{Exercice 2.2 : Fonction \texttt{premierTableau}}
Cette fonction compte le nombre de nombres premiers dans un tableau d'entiers en utilisant la fonction \texttt{premier} précédemment définie.

\begin{verbatim}
Fonction premierTableau(T: tableau[] d'entiers, N: entier): entier
Variables
    i, compteur: entier
Debut
    compteur <-- 0
    Pour i de 1 a N Faire
        Si (premier(T[i]) = Vrai) alors
            compteur <-- compteur + 1
        FinSi
    FinPour
    Retourner compteur
Fin
\end{verbatim}

\vspace{1em}
\hrule
\vspace{1em}

\subsection*{Exercice 2 : Algorithme \texttt{FicheEleve}}
Voici l'algorithme complété. Il saisit les informations d'un certain nombre d'élèves, puis trouve et affiche l'élève ayant la moyenne la plus élevée.

\begin{verbatim}
Algorithme FicheEleve

Type eleve = enregistrement
    nom: chaine de caracteres
    classe: chaine de caracteres
    moyenne: reel
Fin

Procedure saisirEleve(var E: eleve)
Debut
    Ecrire("Nom:")
    Lire(E.nom)
    Ecrire("Classe:")
    Lire(E.classe)
    Ecrire("Moyenne:")
    Lire(E.moyenne)
Fin

Variables
    T: Tableau [1..20] de eleve
    n, i, iMax: entier
    moyMax: reel

Debut
    Ecrire("Donner le nombre des eleves (entre 2 et 20):")
    Repeter
        Lire(n)
    Jusqu'a (n >= 2 ET n <= 20)

    // Remplir le tableau T
    Pour i de 1 a N faire
        saisirEleve(T[i])
    FinPour

    moyMax <-- T[1].moyenne
    iMax <-- 1
    Pour i de 2 a N faire
        Si (T[i].moyenne > moyMax) alors
            moyMax <-- T[i].moyenne
            iMax <-- i
        FinSi
    FinPour

    // Afficher les resultats
    Ecrire("L'eleve avec la moyenne maximale est :")
    Ecrire("Nom: ", T[iMax].nom)
    Ecrire("Classe: ", T[iMax].classe)
    Ecrire("Moyenne: ", moyMax)
Fin
\end{verbatim}

\vspace{1em}
\hrule
\vspace{1em}

\subsection*{Exercice 3 : Analyse Asymptotique et Complexité}
\textbf{1. Montrer que $f(n) = \Theta(g(n))$}

On suppose que $\lim_{n \to \infty} \frac{f(n)}{g(n)} = l$ avec $l > 0$.
Par définition de la limite, pour tout $\varepsilon > 0$, il existe un entier $n_0$ tel que pour tout $n \geq n_0$, on a :
$| \frac{f(n)}{g(n)} - l | < \varepsilon$

Ceci est équivalent à :
$l - \varepsilon < \frac{f(n)}{g(n)} < l + \varepsilon$

En utilisant l'indication et en choisissant $\varepsilon = \frac{l}{2}$ (possible car $l>0$), on obtient :
$l - \frac{l}{2} < \frac{f(n)}{g(n)} < l + \frac{l}{2}$
$\frac{l}{2} < \frac{f(n)}{g(n)} < \frac{3l}{2}$

En posant $c_1 = \frac{l}{2}$ et $c_2 = \frac{3l}{2}$, on a deux constantes positives. En supposant $g(n)>0$, on obtient :
$c_1 \cdot g(n) < f(n) < c_2 \cdot g(n)$

Ceci est la définition de $f(n) = \Theta(g(n))$.

\vspace{1em}
\textbf{2. Temps d'exécution $T(n)$ et complexité de \texttt{Algo}}

L'opération élémentaire est \texttt{Ecrire(j)}. On compte le nombre de fois qu'elle est exécutée.
La boucle externe s'exécute $n$ fois (pour $i$ de 1 à $n$).
La boucle interne s'exécute $i$ fois pour chaque valeur de $i$.
Le nombre total d'opérations $T(n)$ est donc la somme :
$$ T(n) = \sum_{i=1}^{n} \sum_{j=1}^{i} 1 = \sum_{i=1}^{n} i $$
Cette somme est une série arithmétique bien connue :
$$ T(n) = \frac{n(n+1)}{2} = \frac{1}{2}n^2 + \frac{1}{2}n $$
La complexité dans le pire des cas est déterminée par le terme de plus haut degré.
La complexité est donc en $\boldsymbol{O(n^2)}$.







\textbf{3.1. Donner la formule récurrente de la suite $f_{n,m}$}
En observant le code, on peut déduire la relation de récurrence pour $f_{n,m} = \text{mystere}(n,m)$.
\begin{itemize}
    \item \textbf{Cas de base ($n=0$):} $f_{0,m} = m$
    \item \textbf{Cas récursif ($n>0$):} $f_{n,m} = f_{n-1,m} + m^n$
\end{itemize}
La formule récurrente est donc :
$$ f_{n,m} = \begin{cases} m & \text{si } n = 0 \\ f_{n-1,m} + m^n & \text{si } n > 0 \end{cases} $$

\textbf{3.2. Donner l'expression explicite de $f_{n,m}$}
En développant la récurrence :
$f_{n,m} = m^n + f_{n-1,m}$
$f_{n,m} = m^n + m^{n-1} + f_{n-2,m}$
$f_{n,m} = m^n + m^{n-1} + \dots + m^1 + f_{0,m}$
$f_{n,m} = m^n + m^{n-1} + \dots + m^1 + m$

Ceci est la somme $\sum_{k=1}^{n} m^k + m$. (Attention, le terme $m$ vient de $f_{0,m}$, et $m^1$ est le premier terme de la somme déroulée).
Vérifions : $f_{1,m} = f_{0,m} + m^1 = m+m=2m$. Notre formule : $m + \sum_{k=1}^{1} m^k = m+m=2m$. C'est correct.
L'expression est $f_{n,m} = m + \sum_{k=1}^{n} m^k$.

On peut exprimer la somme géométrique :
\begin{itemize}
    \item Si $\boldsymbol{m = 1}$:
    $f_{n,1} = 1 + \sum_{k=1}^{n} 1^k = 1 + n$.
    
    \item Si $\boldsymbol{m \neq 1}$:
    La somme $\sum_{k=1}^{n} m^k$ est une série géométrique de premier terme $m$, de raison $m$ et de $n$ termes. Sa valeur est $m \frac{m^n - 1}{m-1}$.
    Donc, $f_{n,m} = m + m \frac{m^n - 1}{m-1}$.
\end{itemize}
L'expression explicite est :
$$ f_{n,m} = \begin{cases} n+1 & \text{si } m = 1 \\ m \left( 1 + \frac{m^n - 1}{m-1} \right) & \text{si } m \neq 1 \end{cases} $$

\vspace{1em}
\hrule
\vspace{1em}

\subsection*{Exercice 4 : Analyse de fonctions Python}
\textbf{1. Fonction F}
\begin{itemize}
    \item \textbf{1.1. Que retourne la fonction F pour a=80 ?}
    La boucle \texttt{while} s'exécute tant que $2^n \le 80$.
    \begin{itemize}
        \item n=0: $2^0=1 \le 80 \rightarrow n=1$
        \item n=1: $2^1=2 \le 80 \rightarrow n=2$
        \item n=2: $2^2=4 \le 80 \rightarrow n=3$
        \item n=3: $2^3=8 \le 80 \rightarrow n=4$
        \item n=4: $2^4=16 \le 80 \rightarrow n=5$
        \item n=5: $2^5=32 \le 80 \rightarrow n=6$
        \item n=6: $2^6=64 \le 80 \rightarrow n=7$
        \item n=7: $2^7=128 > 80$. La boucle s'arrête.
    \end{itemize}
    La fonction retourne la dernière valeur de $n$, soit \textbf{7}.
    
    \item \textbf{1.2. Que fait la fonction F ? Justifier.}
    La fonction F calcule le plus petit entier $n$ tel que $2^n > a$.
    \textit{Justification:} La boucle \texttt{while} incrémente $n$ tant que la condition $2^n \le a$ est vraie. Elle s'arrête dès que $n$ atteint la première valeur pour laquelle $2^n$ devient strictement supérieur à $a$. La fonction retourne cette valeur de $n$. Mathématiquement, cela correspond à $\lfloor \log_2(a) \rfloor + 1$.
\end{itemize}

\textbf{2. Fonction G}
\begin{itemize}
    \item \textbf{2.1. Que retourne la fonction G pour n=30 ?}
    La boucle \texttt{while} parcourt $i$ de 1 à $n//2=15$. La variable $p$ accumule les diviseurs de 30.
    Les diviseurs de 30 jusqu'à 15 sont : 1, 2, 3, 5, 6, 10, 15.
    $p = 1+2+3+5+6+10+15 = 42$.
    Après la boucle, l'instruction \texttt{p=p+n} est exécutée : $p = 42 + 30 = 72$.
    La fonction retourne \textbf{72}.
    
    \item \textbf{2.2. Que fait la fonction G ? Justifier.}
    La fonction G calcule la somme de tous les diviseurs positifs d'un entier $n$.
    \textit{Justification:} La variable $p$ est initialisée à 0. La boucle \texttt{while} itère de $i=1$ jusqu'à la moitié de $n$. La condition \texttt{if (n\%i==0)} vérifie si $i$ est un diviseur de $n$. Si c'est le cas, $i$ est ajouté à $p$. La boucle additionne donc tous les diviseurs de $n$ jusqu'à $n/2$. Finalement, $n$ lui-même (qui est le plus grand diviseur) est ajouté à $p$. Le résultat est la somme de tous les diviseurs de $n$.
\end{itemize}





\end{document}
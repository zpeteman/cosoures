\documentclass{article}
\usepackage[utf8]{inputenc}
\usepackage{amsmath}
\usepackage{amssymb}

\title{Taylor Formulas, Limited Development and Applications}
\author{}
\date{}

\begin{document}

\maketitle

\section*{Key Options}

-

- Taylor's Theorem (Taylor-Young and Taylor-Lagrange forms)

- Maclaurin Series Expansion

- Limited Developments (DL) and their applications

- Rolle’s Theorem and Mean Value Theorem

- Calculation of limits using DL

- Convexity and Extremum conditions


\section*{Content}
\subsection*{Introduction to Taylor Formulas}
The chapter primarily focuses on Taylor's theorem and its varied applications. It includes:



- \textbf{Taylor-Young Formula}: For a function $ f $ that is $ n $-times differentiable at $ x_0 $,

\[
	f(x) = f(x_0) + (x - x_0)f'(x_0) + \frac{(x - x_0)^2}{2!}f''(x_0) + \cdots + \frac{(x - x_0)^n}{n!}f^{(n)}(x_0) + (x - x_0)^n \varepsilon(x),
\]
where $ \lim_{x \to x_0} \varepsilon(x) = 0 $.


- \textbf{Taylor-Lagrange Formula}: For a function $ f $ of class $ C^n $ on $ [a, b] $ and $ (n+1) $-times differentiable on $ ]a, b[ $,
\[
	f(b) = f(a) + (b-a)f'(a) + \frac{(b-a)^2}{2!}f''(a) + \cdots + \frac{(b-a)^n}{n!}f^{(n)}(a) + \frac{(b-a)^{n+1}}{(n+1)!}f^{(n+1)}(c),
\]
for some $ c \in ]a, b[ $.


\subsection*{Maclaurin Series Expansion}
A special case when $ x_0 = 0 $, known as the Maclaurin series:


-
\[
	f(x) = f(0) + xf'(0) + \frac{x^2}{2!}f''(0) + \cdots + \frac{x^n}{n!}f^{(n)}(0) + o(x^n).
\]


\subsection*{Limited Developments (DL)}
The chapter discusses limited developments extensively:


- Definition: A function $ f $ has a limited development of order $ n $ around 0 if it can be written as:
\[
	f(x) = a_0 + a_1x + a_2x^2 + \cdots + a_nx^n + o(x^n).
\]

- Operations on DLs: Sum, scalar multiplication, product, and quotient of functions with DLs.

- Applications: Calculating indeterminate limits, approximating functions, and solving problems involving convexity and extremum conditions.


\subsection*{Applications}


- \textbf{Calculation of Limits}: Using DLs to resolve indeterminate forms.

- \textbf{Convexity and Extremum Conditions}:


- A function $ f $ is convex if it lies above all its tangents.

- If $ f'(a) = 0 $ and $ f''(a) > 0 $, $ f $ has a local minimum at $ a $. Conversely, if $ f''(a) < 0 $, $ f $ has a local maximum.



\section*{Notes}


- The chapter heavily emphasizes practical applications of Taylor expansions, particularly in evaluating limits and understanding function behavior near critical points.

- Several examples are provided, including expansions for $ e^x $, $ \sin x $, $ \cos x $, and $ \ln(1+x) $.

- The uniqueness of DLs is highlighted, ensuring that for any given order, there is only one valid expansion.

- The text also covers conditions under which functions do not admit DLs, such as $ \frac{1}{x} $, $ \sin(\frac{1}{x}) $, and $ \cos(\frac{1}{x}) $ at $ x = 0 $.


\end{document}
